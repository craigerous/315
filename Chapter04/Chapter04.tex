\chapter{Coupled Oscillations}
\section{Two Spring-Coupled Masses}\label{s4.1}
Consider a mechanical system consisting of two identical masses $m$
which are free to slide over a frictionless horizontal surface. Suppose that
the masses are attached to  one another, and to two immovable
walls, by means of three identical light horizontal springs of spring constant $k$, as
shown in Figure~\ref{f4.1}. The instantaneous state of the system
is conveniently specified by the displacements of the left and
right masses, $x_1(t)$ and $x_2(t)$, respectively. The extensions
of the left, middle, and right springs are  $x_1$, $x_2-x_1$, and $-x_2$,
respectively, assuming that   $x_1=x_2=0$ corresponds to the equilibrium configuration in which the springs are all
unextended. The equations of motion of the two masses
are thus
\begin{eqnarray}\label{e4.1}
m\,\ddot{x}_1 &=&  -k\,x_1 +k\,(x_2-x_1),\\[0.5ex]
m\,\ddot{x}_2 &=&  -k\,(x_2-x_1) +k\,(-x_2).\label{e4.2}
\end{eqnarray}
Here, we have made use of the fact that a mass attached to the left end of a
spring of extension $x$ and spring constant $k$ experiences a horizontal force $+k\,x$,
whereas a mass attached to the right end of the same spring experiences an
equal and opposite force $-k\,x$. 

\begin{figure}
\epsfysize=1.4in
\centerline{\epsffile{Chapter04/fig01.eps}}
\caption{\em Two degree of freedom mass-spring system.}\label{f4.1}   
\end{figure}

Equations~(\ref{e4.1})--(\ref{e4.2}) can be rewritten in the form
\begin{eqnarray}\label{e4.3}
\ddot{x}_1 &=& -2\,\omega_0^{\,2}\,x_1+ \omega_0^{\,2}\,x_2,\\[0.5ex]
\ddot{x}_2&=& \omega_0^{\,2}\,x_1 - 2\,\omega_0^{\,2}\,x_2,\label{e4.4}
\end{eqnarray}
where $\omega_0=\sqrt{k/m}$. Let us search for a solution in which the two
masses oscillate {\em in phase}\/ at the {\em same angular frequency}, $\omega$. In other words,
\begin{eqnarray}\label{e4.5}
x_1(t) &=& \hat{x}_1\,\cos(\omega\,t-\phi),\\[0.5ex]
x_2(t) &=&\hat{x}_2\,\cos(\omega\,t-\phi),\label{e4.6}
\end{eqnarray}
where $\hat{x}_1$, $\hat{x}_2$, and $\phi$ are constants. Equations~(\ref{e4.3}) and
(\ref{e4.4}) yield
\begin{eqnarray}
-\omega^2\,\hat{x}_1\,\cos(\omega\,t-\phi) &=&\left( -2\,\omega_0^{\,2}\,\hat{x}_1+ \omega_0^{\,2}\,\hat{x}_2\right)\,\cos(\omega\,t-\phi),\\[0.5ex]
-\omega^2\,\hat{x}_2\,\cos(\omega\,t-\phi) &=&\left( \omega_0^{\,2}\,\hat{x}_1-2\,\omega_0^{\,2}\,\hat{x}_2\right)\,\cos(\omega\,t-\phi),
\end{eqnarray}
or
\begin{eqnarray}\label{e4.9}
(\hat{\omega}^{\,2}-2)\,\hat{x}_1 +\hat{x}_2 &=&0,\\[0.5ex]
\hat{x}_1 + (\hat{\omega}^{\,2}-2)\,\hat{x}_2&=&0,\label{e4.10}
\end{eqnarray}
where $\hat{\omega}=\omega/\omega_0$. Note that by searching for a solution
of the form (\ref{e4.5})--(\ref{e4.6}) we have effectively converted the system of two coupled
 {\em linear differential equations}\/ (\ref{e4.3})--(\ref{e4.4}) into the  much simpler system of two coupled  {\em linear algebraic
equations}\/ (\ref{e4.9})--(\ref{e4.10}). The latter equations have the trivial solutions $\hat{x}_1=\hat{x}_2=0$, but also yield
\begin{equation}\label{e4.11}
\frac{\hat{x}_1}{\hat{x}_2}=-\frac{1}{(\hat{\omega}^{\,2}-2)} =-( \hat{\omega}^{\,2}-2).
\end{equation}
Hence, the condition for a nontrivial solution is 
\begin{equation}\label{e4.12}
(\hat{\omega}^{\,2}-2)\,(\hat{\omega}^{\,2}-2)-1 = 0.
\end{equation}
In fact, if we write Equations~(\ref{e4.9})--(\ref{e4.10}) in the form of a homogenous ({\em i.e.}, with a null right-hand side) $2\times 2$ matrix equation, so that
\begin{equation}
\left(
\begin{array}{cc}
\hat{\omega}^{\,2}-2 & 1\\[0.5ex]
1&\hat{\omega}^{\,2}-2
\end{array}\right)\left(
\begin{array}{c}
\hat{x}_1\\[0.5ex] \hat{x}_2
\end{array}\right) = \left(
\begin{array}{c}
0\\[0.5ex] 0
\end{array}\right) ,
\end{equation}
then it becomes apparent  that the criterion (\ref{e4.12}) can also be obtained by
setting the {\em determinant}\/ of the associated $2\times 2$ matrix to {\em zero}.

Equation~(\ref{e4.12}) can be rewritten
\begin{equation}\label{e4.14}
\hat{\omega}^{\,4} - 4\,\hat{\omega}^2+3 = (\hat{\omega}^{\,2}-1)\,(\hat{\omega}^{\,2}-3)=0.
\end{equation}
It follows that
\begin{equation}
\hat{\omega} = 1 \mbox{~or~~} \sqrt{3}.
\end{equation}
 Here, we have neglected
the two  negative frequency roots of (\ref{e4.14})---{\em i.e.}, $\hat{\omega}=-1$ and $\hat{\omega}=-\sqrt{3}$---since a negative frequency oscillation
is equivalent to an oscillation with an equal and opposite positive frequency, and
an equal and opposite phase: {\em i.e.},
$\cos(\omega\,t-\phi)\equiv \cos(-\omega\,t+\phi)$. 
 It is thus apparent that
the dynamical system  pictured in Figure~\ref{e4.1} has {\em two}\/  unique  frequencies of oscillation: {\em i.e.},  $\omega=\omega_0$
and $\omega=\sqrt{3}\,\omega_0$. These are called the {\em normal frequencies}\/ of the
system.
Since the system possesses {\em two degrees of
freedom}\/ ({\em i.e.}, two independent coordinates are needed to specify its
instantaneous configuration) it is not entirely surprising that it possesses {\em two}\/
 normal frequencies. In fact, it is a general rule that a dynamical system
with $N$ degrees of freedom possesses $N$  normal frequencies. 

The patterns of motion associated with the two normal frequencies 
can easily be deduced  from Equation~(\ref{e4.11}). Thus, for $\omega=\omega_0$ ({\em i.e.}, $\hat{\omega}=1$), we
get $\hat{x}_1=\hat{x}_2$, so that
\begin{eqnarray}\label{e4.16}
x_1(t) &=& \hat{\eta}_1\,\cos(\omega_0\,t-\phi_1),\\[0.5ex]
x_2(t)&=&\hat{\eta}_1\,\cos(\omega_0\,t-\phi_1),\label{e4.17}
\end{eqnarray}
where $\hat{\eta}_1$ and $\phi_1$ are constants. This first pattern of motion corresponds to the two masses
 executing simple harmonic oscillation with the {\em same amplitude and phase}. 
Note that such an oscillation does not stretch the middle spring.  
On the other hand, for
$\omega=\sqrt{3}\,\omega_0$ ({\em i.e.}, $\hat{\omega}=\sqrt{3}$), we get $\hat{x}_1=-\hat{x}_2$, so that
\begin{eqnarray}\label{e4.18}
x_1(t) &=& \hat{\eta}_2\,\cos\left(\sqrt{3}\,\omega_0\,t-\phi_2\right),\\[0.5ex]
x_2(t)&=&-\hat{\eta}_2\,\cos\left(\sqrt{3}\,\omega_0\,t-\phi_2\right),\label{e4.19}
\end{eqnarray}
where $\hat{\eta}_2$ and $\phi_2$ are constants. This second pattern of motion
corresponds to the two masses executing simple harmonic oscillation with the
{\em same amplitude but in anti-phase}: {\em i.e.}, with a phase shift of $\pi$ radians. Such oscillations do stretch the
middle spring, implying that the  restoring force associated with
similar amplitude displacements is greater for the second
pattern of motion than for the first. This accounts for the higher
oscillation frequency in the second case.  (The inertia is the same in both cases, so the
oscillation frequency is proportional to the square root of the restoring force
associated with similar amplitude displacements.) The two distinctive
patterns of motion which we have found are called the {\em normal modes of
oscillation}\/ of the system. Incidentally, it is a general rule that a dynamical system
possessing $N$ degrees of freedom has $N$ unique normal modes of oscillation.

Now, the most general motion of the system is a
{\em linear combination}\/ of the two normal modes. This immediately follows because 
Equations~(\ref{e4.1}) and ({\ref{e4.2}) are {\em linear equations}. [In other
words, if $x_1(t)$ and $x_2(t)$ are solutions then so are $a\,x_1(t)$ and $a\,x_2(t)$,
where $a$ is an arbitrary constant.] Thus, we can write
\begin{eqnarray}\label{e4.20}
x_1(t) &=& \hat{\eta}_1\,\cos(\omega_0\,t-\phi_1)+ \hat{\eta}_2\,\cos\left(\sqrt{3}\,\omega_0\,t-\phi_2\right),\\[0.5ex]
x_2(t) &=&  \hat{\eta}_1\,\cos(\omega_0\,t-\phi_1)-\hat{ \eta}_2\,\cos\left(\sqrt{3}\,\omega_0\,t-\phi_2\right).\label{e4.21}
\end{eqnarray}
Note that we can be sure that this represents the most general solution
to Equations~(\ref{e4.1}) and (\ref{e4.2}) because it contains {\em four}\/
arbitrary constants: {\em i.e.}, $\hat{\eta}_1$, $\phi_1$, $\hat{\eta}_2$, and $\phi_2$. (In general,
we expect the solution of a second-order ordinary differential equation to
contain two arbitrary constants. It, thus, follows that the solution of a system of two
coupled ordinary differential equations should contain four arbitrary constants.)
Of course, these constants are determined by the {\em initial conditions}. 

For instance, suppose that $x_1=a$, $\dot{x}_1=0$, $x_2=0$, and $\dot{x}_2=0$
at $t=0$. It follows, from (\ref{e4.20}) and (\ref{e4.21}), that
\begin{eqnarray}
a &=&\hat{\eta}_1\,\cos\phi_1 + \hat{\eta}_2\,\cos\phi_2,\\[0.5ex]
0&=&\hat{\eta}_1\,\sin\phi_1+\sqrt{3}\,\hat{\eta}_2\,\sin\phi_2,\\[0.5ex]
0&=&\hat{\eta}_1\,\cos\phi_1 -\hat{\eta}_2\,\cos\phi_2,\\[0.5ex]
0&=&\hat{\eta}_1\,\sin\phi_1-\sqrt{3}\,\hat{\eta}_2\,\sin\phi_2,
\end{eqnarray}
which implies that $\phi_1=\phi_2=0$ and $\hat{\eta}_1=\hat{\eta}_2=a/2$. Thus, the
system evolves in time as 
\begin{eqnarray}
x_1(t)&=& a\,\cos(\omega_-\,t)\,\cos(\omega_+\,t),\\[0.5ex]
x_2(t)&=&a\,\sin(\omega_-\,t)\,\sin(\omega_+\,t),
\end{eqnarray}
where $\omega_\pm = [(\sqrt{3}\pm 1)/2]\,\omega_0$, and
use has been made of the trigonometric identities $\cos a + \cos b\equiv 2\,\cos[(a+b)/2]\,\cos[(a-b)/2]$ and $\cos a - \cos b \equiv -2\,\sin[(a+b)/2]\,\sin[(a-b)/2]$. This evolution is
illustrated in Figure~\ref{f4.2}. [Here, $T_0=2\pi/\omega_0$. The solid curve
corresponds to $x_1$, and the dashed curve to $x_2$.]

\begin{figure}
\epsfysize=3in
\centerline{\epsffile{Chapter04/fig02.eps}}
\caption{\em Coupled oscillations in a two degree of freedom mass-spring system.}\label{f4.2}   
\end{figure}

Finally, let us  define the so-called {\em normal coordinates},
\begin{eqnarray}
\eta_1(t) &=& [x_1(t) + x_2(t)]/2,\\[0.5ex]
\eta_2(t) &=& [x_1(t)-x_2(t)]/2.
\end{eqnarray}
It follows from (\ref{e4.20}) and (\ref{e4.21}) that, in the presence of both normal
modes, 
\begin{eqnarray}\label{e4.30}
\eta_1(t) &=&\hat{\eta}_1\,\cos(\omega_0\,t-\phi_1),\\[0.5ex]
\eta_2(t) &=&\hat{\eta}_2\,\cos(\sqrt{3}\,\omega_0\,t-\phi_2).\label{e4.31}
\end{eqnarray}
Thus, in general, the two normal coordinates oscillate sinusoidally with 
{\em unique frequencies}, unlike the regular coordinates, $x_1(t)$ and $x_2(t)$---see Figure~\ref{f4.2}. 
This suggests that the equations of motion of the system should look particularly simple when expressed in terms of the normal coordinates. In fact, it is easily seen that the
sum of Equations~(\ref{e4.3}) and (\ref{e4.4}) reduces to
\begin{equation}\label{e4.32}
\ddot{\eta}_1 = -\omega_0^{\,2}\,\eta_1,
\end{equation}
whereas the difference gives
\begin{equation}\label{e4.33}
\ddot{\eta}_2 = -3\,\omega_0^{\,2}\,\eta_2.
\end{equation}
Thus, when expressed in terms of the normal coordinates, the equations of motion
of the system reduce to two {\em uncoupled}\/ simple harmonic oscillator
equations. 
Of course, the most general solution to Equation~(\ref{e4.32}) is (\ref{e4.30}),
whereas the most  general solution to Equation~(\ref{e4.33}) is (\ref{e4.31}). 
Hence, if we can guess the normal coordinates of a coupled oscillatory
system then the determination of the normal modes of oscillation is considerably simplified. 

\section{Two Coupled $LC$ Circuits}
Consider the $LC$ circuit pictured in Figure~\ref{f4.3}. Let $I_1(t)$, $I_2(t)$,
and $I_3(t)$ be the currents flowing in the three legs of the circuit, which meet
at junctions $A$ and $B$.
 According to
{\em Kirchhoff's first circuital law}, the net current flowing into
each junction  is zero. It follows that $I_3=-(I_1+I_2)$. Hence, this
is a {\em two degree of freedom}\/ system whose instantaneous configuration is
specified by the two independent variables $I_1(t)$ and $I_2(t)$. It follows that there
are {\em two}\/ independent normal modes of oscillation. 
Now, the potential differences across the left, middle, and right legs of the circuit  are $Q_1/C+L\,\dot{I}_1$, $Q_3/C'$, and $Q_2/C+L\,\dot{I}_2$,
respectively, where $\dot{Q}_1=I_1$, $\dot{Q}_2=I_2$, and $Q_3=-(Q_1+Q_2)$. 
However, since the
three legs are connected in {\em parallel},  the potential differences must
all be equal, so that
\begin{eqnarray}
Q_1/C + L\,\dot{I}_1 &=& Q_3/C' = -(Q_1+Q_2)/C',\\[0.5ex]
Q_2/C+L\,\dot{I}_2 &=& Q_3/C' = -(Q_1+Q_2)/C'.
\end{eqnarray}
Differentiating with respect to $t$, and dividing by $L$,  we obtain the coupled time evolution
equations of the system:
\begin{eqnarray}\label{e4.36}
\ddot{I}_1 + \omega_0^{\,2}\,(1+\alpha)\,I_1+\omega_0^{\,2}\, \alpha\,I_2&=&0,\\[0.5ex]
\ddot{I}_2 + \omega_0^{\,2}\,(1+\alpha)\,I_2+ \omega_0^{\,2}\,\alpha\,I_1&=&0,\label{e4.37}
\end{eqnarray}
where $\omega_0=1/\sqrt{L\,C}$ and $\alpha=C/C'$. 

\begin{figure}
\epsfysize=2.in
\centerline{\epsffile{Chapter04/fig03.eps}}
\caption{\em Two degree of freedom $LC$ circuit.}\label{f4.3}   
\end{figure}

It is fairly easy to guess that the normal coordinates of the system are
\begin{eqnarray}
\eta_1 &=&(I_1+I_2)/2,\\[0.5ex]
\eta_2&=&(I_1-I_2)/2.
\end{eqnarray}
Forming the sum and difference of Equations~(\ref{e4.36})
and (\ref{e4.37}), we obtain the evolution equations for the two
independent normal modes of oscillation:
\begin{eqnarray}
\ddot{\eta}_1+\omega_0^{\,2}\,(1+2\,\alpha)\,\eta_1 &=&0,\\[0.5ex]
\ddot{\eta}_2+\omega_0^{\,2}\,\eta_2 &=&0.
\end{eqnarray}
These equations can readily  be solved to give
\begin{eqnarray}
\eta_1(t)&=&\hat{\eta}_1\,\cos(\omega_1\,t-\phi_1),\\[0.5ex]
\eta_2(t)&=&\hat{\eta}_2\,\cos(\omega_0\,t-\phi_2),
\end{eqnarray}
where $\omega_1=(1+2\,\alpha)^{1/2}\,\omega_0$. Here, $\hat{\eta}_1$, $\phi_1$, $\hat{\eta}_2$, and $\phi_2$ are 
constants determined by the initial conditions. 
It follows that 
\begin{eqnarray}\label{e4.44}
I_1(t)&=&\eta_1(t)+\eta_2(t) = \hat{\eta}_1\,\cos(\omega_1\,t-\phi_1)+\hat{\eta}_2\,\cos(\omega_0\,t-\phi_2),\\[0.5ex]
I_2(t) &=& \eta_1(t)-\eta_2(t) = \hat{\eta}_1\,\cos(\omega_1\,t-\phi_1)-\hat{\eta}_2\,\cos(\omega_0\,t-\phi_2).\label{e4.45}
\end{eqnarray}
As an example, suppose that $\phi_1=\phi_2=0$ and $\hat{\eta}_1=\hat{\eta}_2= I_0/2$. 
We obtain
\begin{eqnarray}
I_1(t) &=& I_0\,\cos(\omega_-\,t)\,\cos(\omega_+\,t),\\[0.5ex]
I_2(t) &=& I_0\,\sin (\omega_-\,t)\,\sin(\omega_+\,t),
\end{eqnarray}
where $\omega_\pm = (\omega_0\pm \omega_1)/2$. This solution is illustrated
in Figure~\ref{f4.4}. [Here, $T_0=2\pi/\omega_0$ and $\alpha=0.2$.
Thus, the two normal frequencies are $\omega_0$ and $1.18\,\omega_0$.] Note the
{\em beats}\/ generated by the superposition of two normal modes with similar
normal frequencies. 

\begin{figure}
\epsfysize=2.2in
\centerline{\epsffile{Chapter04/fig04.eps}}
\caption{\em Coupled oscillations in a two degree of freedom  $LC$ circuit.}\label{f4.4}   
\end{figure}

We can also solve the problem in a more systematic manner by specifically
searching for a normal mode of the form
\begin{eqnarray}
I_1(t) &=&\hat{I}_1\,\cos(\omega\,t-\phi),\\[0.5ex]
I_2(t) &=&\hat{I}_2\,\cos(\omega\,t-\phi).
\end{eqnarray}
Substitution into the time evolution equations (\ref{e4.36}) and (\ref{e4.37})
yields the matrix equation
\begin{equation}
\left(
\begin{array}{cc}
\hat{\omega}^2-(1+\alpha) & -\alpha\\[0.5ex]
-\alpha&\hat{\omega}^{\,2}-(1+\alpha)
\end{array}\right)\left(
\begin{array}{c}
\hat{I}_1\\[0.5ex] \hat{I}_2
\end{array}\right) = \left(
\begin{array}{c}
0\\[0.5ex] 0
\end{array}\right),\label{e4.50}
\end{equation}
where $\hat{\omega}=\omega/\omega_0$. The normal frequencies are
determined by setting the determinant of the matrix to zero. This gives
\begin{equation}
\left[\hat{\omega}^2-(1+\alpha)\right]^2-\alpha^2 = 0,
\end{equation}
or 
\begin{equation}
\hat{\omega}^{\,4} - 2\,(1+\alpha)\,\hat{\omega}^{\,2}+1+2\,\alpha
=\left(\hat{\omega}^{\,2}-1\right)\left(\hat{\omega}^{\,2}-[1+2\,\alpha]\right)=0.
\end{equation}
The roots of the above equation
are $\hat{\omega}=1$ and $\hat{\omega}=(1+2\,\alpha)^{1/2}$. (Again, we neglect the
negative frequency roots, since they generate the same patterns of motion as the
corresponding positive frequency roots.)
Hence, the two
normal frequencies are $\omega_0$ and $(1+2\,\alpha)^{1/2}\,\omega_0$.
The characteristic patterns of motion associated with the normal modes
can be calculated from the first row of the matrix equation (\ref{e4.50}),
which can be rearranged to give
\begin{equation}
\frac{\hat{I}_1}{\hat{I}_2}= \frac{\alpha}{\hat{\omega}^{\,2}-(1+\alpha)}.
\end{equation}
It follows that $\hat{I}_1=-\hat{I}_2$ for the normal mode with $\hat{\omega}=1$,
and $\hat{I}_1=\hat{I}_2$ for the normal mode with $\hat{\omega}=(1+2\,\alpha)^{1/2}$. We are thus led to Equations~(\ref{e4.44}) and (\ref{e4.45}), where
$\hat{\eta}_1$ and $\phi_1$ are the amplitude and phase of the higher frequency
normal mode, whereas $\hat{\eta}_2$ and $\phi_2$ are the amplitude and phase of the lower frequency
 mode.
 
\section{Three Spring Coupled Masses}
Consider a generalized version of the mechanical system discussed in Section~\ref{s4.1}
that consists of {\em three}\/ identical masses $m$ which slide over a
frictionless horizontal surface, and are  connected by identical
light horizontal springs of spring constant $k$. As before, the outermost masses are  attached to immovable
walls by springs of spring constant $k$. The instantaneous configuration of
the system is specified by the horizontal displacements of the
three masses from their equilibrium positions: {\em i.e.},  $x_1(t)$, $x_2(t)$, and $x_3(t)$.
Clearly, this is  a {\em three degree of freedom system}. We, therefore, 
expect it to possesses
{\em three}\/ independent normal modes of oscillation. Equations~(\ref{e4.1})--(\ref{e4.2})
generalize to
\begin{eqnarray}\label{e4.54}
m\,\ddot{x}_1&=&-k\,x_1 + k\,(x_2-x_1),\\[0.5ex]
m\,\ddot{x}_2&=&-k\,(x_2-x_1)+k\,(x_3-x_2),\\[0.5ex]
m\,\ddot{x}_3&=&-k\,(x_3-x_2)+k\,(-x_3).\label{e4.56}
\end{eqnarray}
These equations can be rewritten
\begin{eqnarray}\label{e4.57}
\ddot{x}_1&=&-2\,\omega_0^{\,2}\,x_1 + \omega_0^{\,2}\,x_2,\\[0.5ex]
\ddot{x}_2&=&\omega_0^{\,2}\,x_1-2\,\omega_0^{\,2}\,x_2+\omega_0^{\,2}\,x_3,\\[0.5ex]
\ddot{x}_3 &=&\omega_0^{\,2}\,x_2-2\,\omega_0^{\,2}\,x_3,\label{e4.60}
\end{eqnarray}
where $\omega_0=\sqrt{k/m}$.
Let us search for a normal mode  solution of the form
\begin{eqnarray}
x_1(t)&=&\hat{x}_1\,\cos(\omega\,t-\phi),\\[0.5ex]
x_2(t)&=&\hat{x}_2\,\cos(\omega\,t-\phi),\\[0.5ex]
x_3(t)&=&\hat{x}_3\,\cos(\omega\,t-\phi).\label{e4.62}
\end{eqnarray}
Equations~(\ref{e4.57})--(\ref{e4.62}) can be combined to give  the $3\times 3$ homogeneous matrix equation
\begin{equation}
\left(\begin{array}{ccc}
\hat{\omega}^{\,2}-2 & 1 & 0\\[0.5ex]
1 & \hat{\omega}^{\,2}-2 & 1\\[0.5ex]
0 & 1 & \hat{\omega}^{\,2}-2
\end{array}\right)
\left(\begin{array}{c}\hat{x}_1\\[0.5ex] \hat{x}_2 \\[0.5ex] \hat{x}_3\end{array}\right)
=\left(\begin{array}{c}0\\[0.5ex] 0 \\[0.5ex] 0\end{array}\right),\label{e4.63}
\end{equation}
where $\hat{\omega}=\omega/\omega_0$. 
The normal frequencies  are determined by setting the determinant of the matrix
to zero: {\em i.e.}, 
\begin{equation}
(\hat{\omega}^{\,2}-2)\left[(\hat{\omega}^{\,2}-2)^2-1\right]-(\hat{\omega}^{\,2}-2)=0,
\end{equation}
or
\begin{equation}
(\hat{\omega}^{\,2}-2)\,\left[\hat{\omega}^{\,2}-2-\sqrt{2}\right]\,\left[
\hat{\omega}^{\,2}-2+\sqrt{2}\right]=0.
\end{equation}
Thus, the normal frequencies are
$\hat{\omega}= \sqrt{2}\,(1-1/\sqrt{2})^{1/2}$, $\sqrt{2}$, and
$\sqrt{2}\,(1+1/\sqrt{2})^{1/2}$.  According to the first and third rows
of Equation~(\ref{e4.63}), 
\begin{equation}
\hat{x}_1:\hat{x}_2:\hat{x}_3 :: 1:2- \hat{\omega}^{\,2}:1,
\end{equation}
provided $\hat{\omega}^{\,2}\neq 2$. According to the second row,
\begin{equation}
\hat{x}_1:\hat{x}_2:\hat{x}_3 :: -1:0:1
\end{equation}
when $\hat{\omega}^{\,2} = 2$. 
Note that we can only determine the  {\em ratios}\/ of $\hat{x}_1$, $\hat{x}_2$, and
$\hat{x}_3$, rather
than the absolute values of these quantities. In other words, only the direction of the vector $\hat{\bf x}= (\hat{x}_1,\hat{x}_2,\hat{x}_3)$ is well-defined. [This follows
because the most general solution, (\ref{e4.70}), is undetermined to an arbitrary multiplicative constant.
That is, if ${\bf x}(t)=(x_1(t),x_2(t),x_3(t))$ is a solution to the dynamical equations (\ref{e4.57})--(\ref{e4.60}) then so is $a\,{\bf x}(t)$, where $a$ is an arbitrary constant. This, in turn,
follows because the dynamical equations are linear.]
Let us
arbitrarily set the magnitude of $\hat{\bf x}$ to unity. It follows that 
the normal mode  associated with the  normal frequency $\hat{\omega}_1=\sqrt{2}\,(1-1/\sqrt{2})^{1/2}$
is 
\begin{equation}
\hat{\bf x}_1 = \left(\frac{1}{2},\,\frac{1}{\sqrt{2}},\,\frac{1}{2}\right).
\end{equation}
Likewise, the normal mode  associated with the  normal frequency $\hat{\omega}_2=\sqrt{2}$
is 
\begin{equation}
\hat{\bf x}_2 = \left(-\frac{1}{\sqrt{2}},\,0,\,\frac{1}{\sqrt{2}}\right).
\end{equation}
Finally, the 
normal mode  associated with the  normal frequency $\hat{\omega}_3=\sqrt{2}(1+1/\sqrt{2})^{1/2}$
is 
\begin{equation}
\hat{\bf x}_3 =\left(\frac{1}{2},-\frac{1}{\sqrt{2}},\,\frac{1}{2}\right). 
\end{equation}
Note that the vectors $\hat{\bf x}_1$, $\hat{\bf x}_2$, and $\hat{\bf x}_3$ are {\em mutually perpendicular}: {\em i.e.},
they are {\em normal vectors.} Hence, the name ``normal'' mode. 


Let ${\bf x}=(x_1,x_2,x_2)$. It follows that the most general solution to the problem is
\begin{equation}\label{e4.70}
{\bf x}(t) = \eta_1(t)\,\hat{\bf x}_1 + \eta_2(t)\,\hat{\bf x}_2+\eta_3(t)\,\hat{\bf x}_3,
\end{equation}
where
\begin{eqnarray}
\eta_1(t) &=&\hat{\eta}_1\,\cos(\hat{\omega}_1\,t-\phi_1),\\[0.5ex]
\eta_2(t) &=&\hat{\eta}_2\,\cos(\hat{\omega}_2\,t-\phi_2),\\[0.5ex]
\eta_3(t) &=&\hat{\eta}_3\,\cos(\hat{\omega}_3\,t-\phi_3).
\end{eqnarray}
Here, $\hat{\eta}_{1,2,3}$ and $\phi_{1,2,3}$ are constants. Incidentally, we need to introduce the arbitrary amplitudes
$\hat{\eta}_{1,2,3}$ to make up for the fact that we arbitrarily set the magnitudes of the vectors $\hat{\bf x}_{1,2,3}$ to unity.
Equation~(\ref{e4.70}) yields
\begin{equation}
\left(\begin{array}{c}x_1\\[0.5ex]x_2\\[0.5ex]x_3\end{array}\right)
=
\left(\begin{array}{ccc}
1/2&-1/\sqrt{2}&1/2\\[0.5ex]
1/\sqrt{2}&0&-1/\sqrt{2}\\[0.5ex]
1/2&1/\sqrt{2}&1/2
\end{array}\right)\left(\begin{array}{c}\eta_1\\[0.5ex]\eta_2\\[0.5ex]\eta_3\end{array}\right)
\end{equation}
The above equation can easily be inverted by noting that the matrix
is {\em unitary}: {\em i.e.}, its transpose is equal to its inverse. 
Thus, we obtain
\begin{equation}
\left(\begin{array}{c}\eta_1\\[0.5ex]\eta_2\\[0.5ex]\eta_3\end{array}\right)
=
\left(\begin{array}{ccc}
1/2&1/\sqrt{2}&1/2\\[0.5ex]
-1/\sqrt{2}&0&1/\sqrt{2}\\[0.5ex]
1/2&-1/\sqrt{2}&1/2
\end{array}\right)\left(\begin{array}{c}x_1\\[0.5ex]x_2\\[0.5ex]x_3\end{array}\right)
\end{equation}
This equation determines the three normal coordinates, $\eta_1$, $\eta_2$, $\eta_3$,
in terms of the three conventional coordinates, $x_1$, $x_2$, $x_3$. Note that, in general, 
the normal coordinates are undetermined to arbitrary multiplicative constants.
Incidentally, the above matrix equation can also be obtained directly from (\ref{e4.70}), which yields
$\eta_{1,2,3}={\bf x}\cdot\hat{\bf x}_{1,2,3}$ (since $\hat{\bf x}_{1,2,3}$ are mutually perpendicular unit
vectors). 

\section{Exercises}
{\small
\begin{enumerate}
\item A particle of mass $m$ is attached to a rigid support by means of a spring of
spring constant $k$. At equilibrium, the spring hangs vertically
downward. An identical oscillator is added to this system, the
spring of the former being attached to the mass of the latter.
Calculate the normal frequencies for one-dimensional vertical
oscillations {\em about the equilibrium}, and describe the associated normal modes.

\item Consider a mass-spring system of the general form shown
in Figure~\ref{f4.1} in which the two masses are of mass $m$, the two outer springs have spring constant
$k$, and the middle spring has spring constant $k'$. Find the normal
frequencies and normal modes in terms of $\omega_0=\sqrt{k/m}$ and
$\alpha=k'/k$. 

\item Consider a mass-spring system of the general form shown
in Figure~\ref{f4.1} in which the springs all have spring constant $k$, and the
left and right masses are of mass $m$ and $m'$, respectively. Find the normal
frequencies and normal modes in terms of $\omega_0=\sqrt{k/m}$ and
$\alpha=m'/m$. 

\item Consider two simple pendula with the same
length, $l$, but different bob masses, $m_1$ and $m_2$. Suppose
that the pendula are connected by a spring of spring constant $k$. 
Let the spring be unextended when the two bobs are in their equilibrium
positions. Demonstrate that the equations of motion of the system (for small
amplitude oscillations) are
\begin{eqnarray}
m_1\,\ddot{\theta}_1&=&-m_1\,\frac{g}{l}\,\theta_1+ k\,(\theta_2-\theta_1),\nonumber\\[0.5ex]
m_2\,\ddot{\theta}_2&=&-m_2\,\frac{g}{l}\,\theta_2+ k\,(\theta_1-\theta_2),\nonumber
\end{eqnarray}
where $\theta_1$ and $\theta_2$ are the angular displacements of the respective
pendula from their equilibrium positions. Show that the
normal coordinates are $\eta_1=(m_1\,\theta_1+m_2\,\theta_2)/(m_1+m_2)$ and
$\eta_2=\theta_1-\theta_2$. Find the normal frequencies and normal
modes. Find a superposition of the two modes such that at $t=0$ the
two pendula are stationary, with  $\theta_1=\theta_0$, and $\theta_2=0$. 

\begin{figure}[h]
\epsfysize=1.6in
\centerline{\epsffile{Chapter04/fig05.eps}}
\end{figure}

\item Find the normal frequencies and normal modes of the coupled $LC$ circuit
shown above in terms of $\omega_0=1/\sqrt{L\,C}$ and $\alpha=L'/L$.
\end{enumerate}
}