\chapter{Wave Mechanics}

\section{Introduction}
According to {\em classical physics}\/ ({\em i.e.}, physics prior to the 20th century), {\em particles}\/ and {\em waves}\/
are two completely distinct classes of physical entity that possess markedly 
 different properties. 1)~Particles are {\em discrete}: {\em i.e.}, they cannot be arbitrarily divided. 
In other words,  it makes sense to talk about one electron, or two electrons, but not  about a third of an electron. Waves, on the
other hand, are {\em continuous}: {\em i.e.}, they can be arbitrarily divided. In other words, given a wave whose amplitude has a certain value,  it makes sense to talk about a similar wave whose amplitude is one third, or any other fraction  whatsoever, of this value. 2)~Particles are {\em highly
 localized}\/ in space.  For example, atomic nuclei  have very small radii of order $10^{-15}\,m$, whilst
electrons act like point particles: {\em i.e.}, they  have no discernible  spatial extent.
Waves, on the other hand, are {\em non-localized}\/ in space. In fact, a wave is defined to be a disturbance that
is periodic in space, with some finite periodicity length: {\em i.e.}, wavelength. Hence, it is fairly meaningless to
talk about a disturbance being a wave unless it extends over
a region of space that is at least a few wavelengths in dimension.

The classical scenario, described above, in which particles and waves are completely distinct from one another, had to be
significantly modified  in the early decades of the 20th century.
During this time period, physicists discovered, much to their surprise, that, under certain circumstances, waves
 act as particles, and particles act as waves. This bizarre phenomenon is known as
{\em wave-particle duality}. For instance, the {\em photoelectric effect}\/ (see Section~\ref{s12.2}) indicates that
electromagnetic waves  sometimes act like swarms of massless particles called {\em photons}. Moreover, the phenomenon of {\em electron diffraction}\/ by atomic lattices (see Section~\ref{s12.3})
implies that electrons  sometimes have wave-like properties.  Note, however, that wave-particle
duality usually only manifests itself on atomic and sub-atomic lengthscales ({\em i.e.}, on lengthscales less than,
or of order, 
$10^{-10}\,{\rm m}$---see Section~\ref{s12.3}).  The classical picture remains valid on significantly longer lengthscales. In other words,
on {\em macroscopic}\/ lengthscales,  
  waves only act like waves, particles only act like particles, and there is no wave-particle duality.  However, on
  {\em microscopic}\/ lengthscales, {\em classical mechanics}, which governs the macroscopic behavior of massive particles, and
{\em classical electrodynamics}, which governs the macroscopic behavior of electromagnetic fields---neither of
which take wave-particle duality into account---must be replaced by new theories. The theories in question are called {\em quantum mechanics}\/ and {\em quantum electrodynamics},
respectively. In the following, we shall discuss a simplified version of quantum mechanics in which the microscopic  dynamics of
 massive particles ({\em i.e.}, particles with finite mass) is described
 {\em entirely}\/ in terms of  wavefunctions. This
 particular 
theory is known as {\em wave mechanics}.  

\section{Photoelectric Effect}\label{s12.2}
The so-called {\em photoelectric effect}, by which a polished metal surface emits electrons
when illuminated by visible or ultra-violet light, was discovered by Heinrich Hertz in 1887.
The following facts regarding this effect can be established via careful
observation. First, a given surface only emits electrons when the {\em frequency}\/
of the light with which it is illuminated exceeds a certain threshold value,
which is a property of the metal. Second, the current of photoelectrons, when it
exists, is proportional to the {\em intensity}\/ of the light falling on the surface. 
Third, the energy of the photoelectrons is independent of the light intensity,
but varies {\em linearly}\/ with the light frequency. These facts are
inexplicable within the framework of classical physics.

In 1905, Albert Einstein proposed a radical new theory of light in order to
account for the photoelectric effect. According to this  theory, light
of  fixed angular frequency $\omega$ consists of a collection of indivisible discrete packages, called
{\em quanta},\footnote{Plural of {\em quantum}: Latin neuter
of {\em quantus}\/: how much.} whose energy is
\begin{equation}\label{e12.1}
E = \hbar\,\omega.
\end{equation}
Here, $\hbar =  1.055\times 10^{-34}\,{\rm J\,s}$ is a new constant of nature,
known as {\em Planck's constant}. (Strictly speaking, it is Planck's constant divided by $2\pi$). Incidentally, $\hbar$ is called Planck's constant, rather than Einstein's constant, because Max Planck first introduced the concept of the quantization of light, in 1900, whilst trying
to account for the  electromagnetic spectrum of a black body ({\em i.e.},
a perfect emitter and absorber of electromagnetic radiation). 

\begin{figure}
\epsfysize=3in
\centerline{\epsffile{Chapter12/fig01.eps}}
\caption{\em Variation of the kinetic energy $K$ of photoelectrons with the wave angular frequency $\omega$.}\label{f13.1}   
\end{figure}

Suppose that the electrons at the surface of a piece of metal lie in a potential well
of depth $W$. In other words, the electrons have to acquire an energy $W$
in order to be emitted from the surface. Here, $W$ is generally called
the {\em work-function}\/ of the surface, and is a property of the
metal. Suppose that an electron absorbs a single quantum of light, otherwise known as a {\em photon}. Its energy
therefore increases by $\hbar\,\omega$. If $\hbar\,\omega$ is greater than $W$ then the
electron is emitted from the surface with the residual kinetic energy
\begin{equation}
K = \hbar\,\omega - W.
\end{equation}
Otherwise, the electron remains trapped in the potential well, and is not emitted. Here, we are assuming that the probability of an electron absorbing
two or more photons is negligibly small compared to the probability of it
absorbing a single photon (as is, indeed, the case for
low intensity illumination). Incidentally, we can determine Planck's
constant, as well as the work-function of the metal, by plotting the kinetic
energy of the emitted photoelectrons as a function of the wave frequency,
as shown in Figure~\ref{f13.1}. This plot is a straight-line whose slope is $\hbar$,
and whose intercept with the $\omega$ axis is $W/\hbar$. Finally, the number
of emitted electrons increases with the intensity of the light because the
more intense the light the larger the flux of photons onto the surface.
Thus, Einstein's quantum theory of light is capable of accounting for all
three of the previously mentioned observational facts regarding the photoelectric
effect. In the following, we shall assume that  the central component of Einstein's theory---namely, Equation~(\ref{e12.1})---is a general result which applies to {\em all}\/ particles,   not
just photons. 

\section{Electron Diffraction}\label{s12.3}
In 1927, George Paget Thomson discovered that if a beam of electrons
is made to pass through a thin metal film then the regular atomic array in  the metal acts as a
sort of diffraction grating, so that when a photographic film, placed behind the metal, is developed an
{\em interference pattern}\/ is discernible. Of course, this implies that electrons have wave-like
properties. Moreover, the electron wavelength, $\lambda$,  or, alternatively, the wavenumber, $k =2\pi/\lambda$, can be deduced from the spacing
of the maxima in the interference pattern (see Chapter~\ref{c11}).
Thomson found that the momentum, $p$, of an electron is related to its wavenumber, $k$, according to the
following
simple relation:
\begin{equation}\label{e12.3}
p = \hbar\,k.
\end{equation}
The associated wavelength, $\lambda = 2\pi/k$, is known as the {\em de Broglie wavelength}, since the above
relation was first hypothesized   by Louis de Broglie in 1926.
In the following, we shall assume that Equation~(\ref{e12.3}) is a general result which applies to {\em all}\/ particles,  not just electrons.

It turns out that wave-particle duality only manifests itself on lengthscales less than,
or of order, the de Broglie wavelength. Note, however, that this wavelength is generally pretty small. For instance,
the de Broglie wavelength of an electron is
\begin{equation}
\lambda_e = 1.2\times 10^{-9}\,[E({\rm eV})]^{-1/2}\,{\rm m},
\end{equation}
where the electron energy is conveniently measured in units of electron-volts (eV). 
(An electron accelerated from rest through a potential difference of $1000$\,V
acquires an energy of $1000$\,eV, and so on. Electrons in atoms typically have energies in the range $10$ to $100$ eV.) Moreover, the de Broglie wavelength
of a proton is
\begin{equation}
\lambda_p = 2.9\times 10^{-11}\,[E({\rm eV})]^{-1/2}\,{\rm m}.
\end{equation}

\section{Representation of Waves via Complex Numbers}
In mathematics, the symbol ${\rm i}$ is conventionally used to represent the {\em square-root of minus one}: {\em i.e.}, the
solution of ${\rm i}^2 = -1$. Now, a {\em real number}, $x$ (say), can take any value in a continuum of different values lying between $-\infty$ and $+\infty$. 
On the other hand, an {\em imaginary number}\/ takes the general form ${\rm i}\,y$, where $y$ is a real number. It follows that the square of
a real number is a positive real number, whereas the square of an imaginary number is a negative real number. In addition, a general {\em complex number}\/ is written
\begin{equation}
z = x + {\rm i}\,y,
\end{equation}
where $x$ and $y$ are real numbers. In fact, $x$ is termed the {\em real part}\/ of $z$, and $y$ 
the {\em imaginary part}\/ of $z$. This is written mathematically as $x={\rm Re}(z)$ and $y={\rm Im}(z)$. 
Finally, the {\em complex conjugate}\/ of $z$ is defined $z^\ast = x-{\rm i}\,y$.

Now, just as we
can visualize a real number as a point on an infinite straight-line, we can visualize a complex number as
a point in an infinite plane. The coordinates of the point in question are the real and imaginary
parts of the number: {\em i.e.}, $z\equiv (x,\,y)$. This idea is illustrated in Figure~\ref{f13.2}. 
The distance, $r=\sqrt{x^2+y^2}$, of the representative point from the origin is termed the {\em modulus}\/
of the corresponding complex number, $z$. This is written mathematically as $|z|=\sqrt{x^2+y^2}$.  Incidentally, it follows that $z\,z^\ast = x^2 + y^2=|z|^2$. 
The angle, $\theta=\tan^{-1}(y/x)$,  that the straight-line joining the representative point to the origin subtends with the 
real axis is termed the {\em argument}\/ of the corresponding complex number, $z$. This is written mathematically
as ${\rm arg}(z)=\tan^{-1}(y/x)$. It follows from standard trigonometry that $x=r\,\cos\theta$, and $y=r\,\sin\theta$.
Hence, $z= r\,\cos\theta+ {\rm i}\,r\sin\theta$. 

\begin{figure}
\epsfysize=3.25in
\centerline{\epsffile{Chapter12/fig02.eps}}
\caption{\em Representation of a complex number as a point in a plane.}\label{f13.2}   
\end{figure}

Complex numbers are often used to represent waves, and wavefunctions. All such representations  depend ultimately on a fundamental mathematical identity, known as
{\em de Moivre's theorem}\/ (see Exercise~12.1), which takes the form
\begin{equation}
{\rm e}^{\,{\rm i}\,\phi} \equiv \cos\phi + {\rm i}\,\sin\phi,
\end{equation}
where $\phi$ is a  real number. Incidentally, given that $z=r\,\cos\theta + {\rm i}\,r\,\sin\theta= r\,[\cos\theta+{\rm i}\,\sin\theta]$, where $z$ is a general
complex number, $r=|z|$  its modulus, and $\theta={\rm arg}(z)$ its argument, it follows from de Moivre's theorem that any
complex number, $z$, can be written
\begin{equation}
z = r\,{\rm e}^{\,{\rm i}\,\theta},
\end{equation}
where $r=|z|$ and $\theta={\rm arg}(z)$ are real numbers. 

Now, a  one-dimensional wavefunction takes the general form
\begin{equation}\label{e12.8}
\psi(x,t) = A\,\cos(\phi+k\,x-\omega\,t),
\end{equation}
where $A>0$ is the wave amplitude, $\phi$ the phase angle, $k$ the wavenumber, and $\omega$ the angular
frequency.  Consider the complex wavefunction
\begin{equation}\label{e12.10}
\psi(x,t) = \psi_0\,{\rm e}^{\,{\rm i}\,(k\,x-\omega\,t)},
\end{equation}
where $\psi_0$ is a complex constant. We can write
\begin{equation}
\psi_0 = A\,{\rm e}^{\,{\rm i}\,\phi},
\end{equation}
where $A$ is the modulus, and $\phi$ the argument, of $\psi_0$.
Hence, we deduce that
\begin{eqnarray}
{\rm Re}\left[\psi_0\,{\rm e}^{\,{\rm i}\,(k\,x-\omega\,t)}\right] &=& {\rm Re}\left[A\,{\rm e}^{\,{\rm i}\,\phi}\,{\rm e}^{\,{\rm i}\,(k\,x-\omega\,t)}\right]\nonumber\\[0.5ex]
&=&{\rm Re}\left[A\,{\rm e}^{\,{\rm i}\,(\phi+k\,x-\omega\,t)}\right]\nonumber\\[0.5ex]&=&A\,{\rm Re}\left[{\rm e}^{\,{\rm i}\,(\phi+k\,x-\omega\,t)}\right].
\end{eqnarray}
Thus, it follows from de Moirve's theorem, and Equation~(\ref{e12.8}), that
\begin{equation}
{\rm Re}\left[\psi_0\,{\rm e}^{\,{\rm i}\,(k\,x-\omega\,t)}\right] =A\,\cos(\phi+k\,x-\omega\,t)=\psi(x,t).
\end{equation}
In other words, a  general one-dimensional real wavefunction, (\ref{e12.8}), can be
represented as the {\em real part}\/ of a complex wavefunction of the form (\ref{e12.10}).
For ease
of notation, the ``take the real part''  aspect of the above expression is usually omitted, and our general one-dimension wavefunction
is simply written
\begin{equation}\label{e12.13}
\psi(x,t) = \psi_0\,{\rm e}^{\,{\rm i}\,(k\,x-\omega\,t)}.
\end{equation}
 The
main advantage of the complex representation, (\ref{e12.13}), over the more straightforward
real representation, (\ref{e12.8}), is that the former enables us to combine the amplitude, $A$, and the
phase angle, $\phi$, of the wavefunction into a single complex amplitude, $\psi_0$. 

\section{Schr\"{o}dinger's Equation}\label{s12.5}
The basic premise of wave mechanics is that a massive particle of energy $E$ and linear momentum $p$, moving in the $x$-direction (say), 
can be represented by a one-dimensional {\em complex wavefunction}\/ of the form
\begin{equation}\label{e12.13a}
\psi(x,t) = \psi_0\,{\rm e}^{\,{\rm i}\,(k\,x-\omega\,t)},
\end{equation}
where the complex amplitude, $\psi_0$, is arbitrary, whilst the wavenumber, $k$, and the angular frequency, $\omega$,
are related to the particle momentum, $p$, and energy, $E$, via the fundamental
relations (\ref{e12.3}) and (\ref{e12.1}), respectively. Now, the above one-dimensional wavefunction is, presumably, the solution of
some one-dimensional wave equation that determines how the wavefunction evolves in time. 
As described below, we can guess the form of this wave equation by drawing an analogy with classical physics.

A classical particle of mass $m$, moving in a one-dimensional potential $U(x)$, satisfies the energy conservation
equation
\begin{equation}
E = K+ U,
\end{equation}
where
\begin{equation}
K = \frac{p^2}{2\,m}
\end{equation}
is the particle's kinetic energy. Hence, 
\begin{equation}\label{e12.16}
E\,\psi = (K+U)\,\psi
\end{equation}
is a valid, but not obviously useful,  wave equation. 

However, it follows from Equations~(\ref{e12.13a}) and (\ref{e12.1}) that
\begin{equation}
\frac{\partial \psi}{\partial t} = -{\rm i}\,\omega\,\psi_0\,{\rm e}^{\,{\rm i}\,(k\,x-\omega\,t)} =- {\rm i}\,\frac{E}{\hbar}\,\psi,
\end{equation}
which can be rearranged to give
\begin{equation}\label{e12.18}
E\,\psi= {\rm i}\,\hbar\,\frac{\partial\psi}{\partial t}.
\end{equation}
Likewise, from (\ref{e12.13a}) and (\ref{e12.3}), 
\begin{equation}
\frac{\partial^2\psi}{\partial x^2} = - k^2\,\psi_0 \,{\rm e}^{\,{\rm i}\,(k\,x-\omega\,t)} = - \frac{p^2}{\hbar^2}\,\psi,
\end{equation}
which can be rearranged to give
\begin{equation}\label{e12.20}
\frac{p^2}{2\,m}\,\psi = -\frac{\hbar^2}{2\,m}\,\frac{\partial^2\psi}{\partial x^2}.
\end{equation}
Thus, combining Equations~(\ref{e12.16}), (\ref{e12.18}), and (\ref{e12.20}), we obtain
\begin{equation}\label{e12.21}
{\rm i}\,\hbar\,\frac{\partial\psi}{\partial t} = -\frac{\hbar^2}{2\,m}\,\frac{\partial^2\psi}{\partial x^2} + U(x)\,\psi.
\end{equation}
This equation, which is known as {\em Schr\"{o}dinger's equation}---since it was first formulated  by Erwin Schr\"{o}dinder in 1926---is the fundamental equation of wave mechanics.

Now, for a massive particle moving  in free space ({\em i.e.}, $U=0$), the complex wavefunction (\ref{e12.13a}) is a
solution of Schr\"{o}dinger's equation, (\ref{e12.21}), provided that
\begin{equation}\label{e12.25r}
\omega = \frac{\hbar}{2\,m}\,k^2.
\end{equation}
The above expression can be thought of as the {\em dispersion relation}\/ (see Section~\ref{s5.1}) for matter waves in free space. The
associated {\em phase velocity}\/ (see Section~\ref{s7.2}) is
\begin{equation}\label{e12.26r}
v_p = \frac{\omega}{k} = \frac{\hbar\,k}{2\,m} = \frac{p}{2\,m},
\end{equation}
where use has been made of (\ref{e12.3}). Note that this phase velocity is only {\em half}\/ the classical velocity, $v=p/m$,
of a massive (non-relativistic) particle.

\section{Probability Interpretation of the Wavefunction}\label{s12.6}
After many false starts, physicists in the early 20th century came to the conclusion that the only self-consistent physical
interpretation of a particle wavefunction, which is consistent with experimental observations, is {\em probabilistic}\/ in nature. To be more exact, if $\psi(x,t)$ is the
complex wavefunction of a given particle, moving in one-dimension along the $x$-axis, then the {\em probability}\/ of finding the particle between $x$ and
$x+dx$ at time $t$ is
\begin{equation}\label{e12.25}
P(x,t) = |\psi(x,t)|^2\,dx.
\end{equation}
A probability is, of course, a real number lying in the range $0$ to $1$. An event which has a probability $0$ is impossible. On the
other hand, an event which has a probability $1$ is certain to occur. An event which has an probability $1/2$ (say) is such that in a
very large number of identical trials the event occurs in half of the trials. Now, we can interpret
\begin{equation}
P(t) = \int_{-\infty}^\infty |\psi(x,t)|^2\,dx
\end{equation}
as the probability of the particle being found {\em anywhere}\/ between $x=-\infty$ and $x=+\infty$ at time $t$. This follows, via induction,
from the fundamental result in probability theory that the probability of the occurrence of {\em one or other}\/ of two {\em mutually exclusive}\/ events (such as the particle being found in two
non-overlapping regions) is the {\em sum}\/ (or integral) of the probabilities of the individual events. (For example, the probability
of throwing a $1$ on a six-sided  die is $1/6$. Likewise, the probability of throwing a 2 is $1/6$. Hence, the
probability of throwing a $1$ {\em or}\/ a $2$ is $1/6+1/6=1/3$.)
Now, assuming that the
particle exists, it is {\em certain}\/ that it will be found somewhere between  $x=-\infty$ and $x=+\infty$ at time $t$. Since a certain event
has probability $1$, our probability interpretation of the wavefunction is only tenable provided that
\begin{equation}\label{e12.24r}
\int_{-\infty}^\infty |\psi(x,t)|^2\,dx=1
\end{equation}
at all times. 
A wavefunction which satisfies the above condition is said to be {\em properly normalized}.

Suppose that we have a wavefunction, $\psi(x,t)$, which is such that it satisfies the normalization condition (\ref{e12.24r})
at time $t=0$. Furthermore, let the wavefunction evolve in time according to Schr\"{o}dinger's equation, (\ref{e12.21}). 
Our probability interpretation of the wavefunction only makes sense if the normalization
condition remains satisfied at all subsequent times. This follows because if the particle is certain to be
found somewhere on the $x$-axis  (which is the interpretation put on the normalization condition) at time $t=0$ then
it is equally certain to be found somewhere on the $x$-axis at a later time (since we are not presently dealing with any
physical process by which particles can be created or destroyed). Thus, it is necessary for us to demonstrate that
Schr\"{o}dinger's equation preserves the normalization of the wavefunction.

Taking Schr\"{o}dinger's equation, and multiplying it by $\psi^\ast$ (the complex conjugate of the wavefunction), we
obtain
\begin{equation}
{\rm i}\,\hbar\,\frac{\partial\psi}{\partial t}\,\psi^\ast = -\frac{\hbar^2}{2\,m}\,\frac{\partial^2\psi}{\partial x^2}\,\psi^\ast + U(x)\,|\psi|^2.
\end{equation}
The complex conjugate of the above expression yields
\begin{equation}
-{\rm i}\,\hbar\,\frac{\partial\psi^\ast}{\partial t}\,\psi = -\frac{\hbar^2}{2\,m}\,\frac{\partial^2\psi^\ast}{\partial x^2}\,\psi+ U(x)\,|\psi|^2.
\end{equation}
Here, we have made use  of the easily demonstrated results $(\psi^\ast)^\ast=\psi$ and ${\rm i}^\ast=-{\rm i}$, as well as the fact that $U$ is real. 
Taking the difference between the above two expressions, we obtain
\begin{equation}
{\rm i}\,\hbar\left(\frac{\partial\psi}{\partial t}\,\psi^\ast + \frac{\partial\psi^\ast}{\partial t}\,\psi\right)
= -\frac{\hbar^2}{2\,m}\left(\frac{\partial^2\psi}{\partial x^2}\,\psi^\ast - \frac{\partial^2\psi^\ast}{\partial x^2}\,\psi\right),
\end{equation}
which can be written
\begin{equation}
{\rm i}\,\hbar\,\frac{\partial |\psi|^2}{\partial t} = -\frac{\hbar^2}{2\,m}\,\frac{\partial}{\partial x}\!\left(
\frac{\partial\psi}{\partial x}\,\psi^\ast - \frac{\partial\psi^\ast}{\partial x}\,\psi\right).
\end{equation}
Integrating in $x$, we get
\begin{equation}
{\rm i}\,\hbar\,\frac{d}{dt}\int_{-\infty}^\infty |\psi|^2\,dx= -\frac{\hbar^2}{2\,m}\left[
\frac{\partial\psi}{\partial x}\,\psi^\ast - \frac{\partial\psi^\ast}{\partial x}\,\psi\right]_{-\infty}^\infty.
\end{equation}
Finally, assuming that the wavefunction is {\em localized}\/ in space: {\em i.e.}, 
\begin{equation}
|\psi(x,t)|\rightarrow 0 \mbox{\hspace{0.5cm}as\hspace{0.5cm}} |x|\rightarrow\infty,
\end{equation}
we obtain
\begin{equation}
\frac{d}{dt}\int_{-\infty}^\infty |\psi|^2\,dx=0.
\end{equation}

It follows, from the above analysis, that if a localized wavefunction is properly normalized at $t=0$  ({\em i.e.}, if
$\int_{-\infty}^\infty |\psi(x,0)|^2\,dx =1$) then it will remain properly
normalized as it evolves in time according to Schr\"{o}dinger's equation. Incidentally,
a wavefunction which is not localized cannot be properly normalized, since its normalization integral $\int_{-\infty}^\infty|\psi|^2\,dx$
is necessarily {\em infinite}. For such a wavefunction, $|\psi(x,t)|^2\,dx$ gives the {\em relative probability}, rather than the
absolute probability, of finding the particle between $x$ and $x+dx$ at time $t$: {\em i.e.}, [{\em cf.}, Equation~(\ref{e12.25})]
\begin{equation}
P(x,t)\propto |\psi(x,t)|^2\,dx.
\end{equation}

\section{Wave Packets}
As we have seen, the wavefunction of a massive particle
of momentum $p$ and energy $E$, moving in free space  along the $x$-axis,  can be written
\begin{equation}\label{e12.38r}
\psi(x,t) = \bar{\psi}\,{\rm e}^{\,{\rm i}\,(k\,x-\omega\,t)},
\end{equation}
where $k= p/\hbar$, $\omega = E/\hbar$, and $\bar{\psi}$ is a complex constant. Here, $\omega$ and
$k$ are linked via the matter wave dispersion relation (\ref{e12.25r}). Expression (\ref{e12.38r}) represents a {\em plane wave}\/ which propagates in the $x$-direction
with the phase velocity $v_p=\omega/k$.  However, according to (\ref{e12.26r}), this phase velocity is only half of the classical velocity of a massive particle.

Now, according to the discussion in the previous section, the most reasonable physical interpretation of the wavefunction is that
$|\psi(x,t)|^{\,2}\,dx$ is proportional to (assuming that the wavefunction is not
properly normalized) the {\em probability}\/  of finding the particle
between $x$ and $x+dx$ at time $t$.  However, the modulus squared of the wavefunction (\ref{e12.38r}) is $|\bar{\psi}|^{\,2}$, which is a constant that depends on neither $x$ nor $t$. In other words, the above  wavefunction represents a particle
which is {\em equally likely}\/ to be found {\em anywhere}\/ on the $x$-axis {\em at all times}. 
Hence, the fact that this wavefunction propagates at 
a phase velocity which does not correspond to the classical particle velocity has no observable   consequences.

So, how can we write the wavefunction of a particle which is {\em localized}\/
in $x$: {\em i.e.}, a particle which is more likely to be found at some
positions on the $x$-axis than at others? It turns out that we can achieve this goal by forming
a {\em linear combination}\/ of plane waves of different wavenumbers:
{\em i.e.}, 
\begin{equation}\label{e12.37}
\psi(x,t) = \int_{-\infty}^{\infty} \bar{\psi}(k)\,{\rm e}^{\,{\rm i}\,(k\,x-\omega\,t)}\,dk.
\end{equation}
Here, $\bar{\psi}(k)$ represents the complex amplitude of plane waves of wavenumber $k$ within this combination. In writing the above expression,
we are relying on the assumption that matter waves are  {\em superposable}:
{\em i.e.}, it is possible to add two valid wave solutions to form a third valid wave solution.
The ultimate justification for this assumption is that matter waves
satisfy the {\em linear}\/ wave equation (\ref{e12.21}).

Now, there is a fundamental mathematical theorem, known as {\em Fourier's theorem}\/  (see Section~\ref{s8.1} and Exercise 12.2), which states that if
\begin{equation}\label{e12.38}
f(x) = \int_{-\infty}^{\infty} \bar{f}(k)\,{\rm e}^{\,{\rm i}\,k\,x}\,dk,
\end{equation}
then
\begin{equation}\label{e12.39}
\bar{f}(k) = \frac{1}{2\pi}\int_{-\infty}^\infty f(x)\,{\rm e}^{-{\rm i}\,k\,x}\,dx.
\end{equation}
Here, $\bar{f}(k)$ is known as the {\em Fourier transform}\/ of the
function $f(x)$. We can use Fourier's theorem to find the $k$-space function $\bar{\psi}(k)$ which generates any given $x$-space wavefunction $\psi(x)$
at a given time.

For instance, suppose that at $t=0$ the wavefunction of our particle takes the
form
\begin{equation}\label{e12.40}
\psi(x,0) \propto \exp\left[{\rm i}\,k_0\,x - \frac{(x-x_0)^{\,2}}{4\,({\mit\Delta}x)^{\,2}}\right].
\end{equation}
Thus, the initial probability distribution for the particle's $x$-coordinate is 
\begin{equation}\label{e12.41}
|\psi(x,0)|^{\,2} \propto \exp\left[- \frac{(x-x_0)^{\,2}}{2\,({\mit\Delta}x)^{\,2}}\right].
\end{equation}
This particular  distribution is called a {\em Gaussian}\/ distribution (see Section~\ref{s8.1}), and is plotted in Figure~\ref{f12.3}. 
It can be seen that a measurement of the particle's position is most
likely to yield the value $x_0$, and  very
unlikely to yield a value which differs from $x_0$ by more than
$3\,{\mit\Delta} x$. Thus, (\ref{e12.40}) is the wavefunction of a particle
which is initially localized  in some region of $x$-space, centered  on $x=x_0$, whose width is
of order ${\mit\Delta} x$. This type of wavefunction is
known as a {\em wave packet}. Of course, a wave packet is just another name for a wave pulse (see Chapter~\ref{c8}).

\begin{figure}
\epsfysize=3.in
\centerline{\epsffile{Chapter12/fig03.eps}}
\caption{\em A one-dimensional Gaussian probability distribution.}\label{f12.3}   
\end{figure}

Now, according to Equation~(\ref{e12.37}), 
\begin{equation}
\psi(x,0) = \int_{-\infty}^{\infty} \bar{\psi}(k)\,{\rm e}^{\,{\rm i}\,k\,x}\,dk.
\end{equation}
Hence, we can employ Fourier's theorem to invert this expression to give
\begin{equation}\label{e12.45}
\bar{\psi}(k)\propto \int_{-\infty}^{\infty} \psi(x,0)\,{\rm e}^{-{\rm i}\,k\,x}\,dx.
\end{equation}
Making use of Equation~(\ref{e12.40}),
we obtain
\begin{equation}
\bar{\psi}(k) \propto
{\rm e}^{-{\rm i}\,(k-k_0)\,x_0}\int_{-\infty}^{\infty} \exp\left[
-{\rm i}\,(k-k_0)\,(x-x_0) - \frac{(x-x_0)^2}{4\,({\mit\Delta}x)^2}\right]dx.
\end{equation}
Changing the variable of integration to $y=(x-x_0)/ (2\,{\mit\Delta} x)$, the above expression reduces to
\begin{equation}
\bar{\psi}(k) \propto {\rm e}^{-{\rm i}\,k\,x_0 - \beta^2/4}\int_{-\infty}^{\infty} {\rm e}^{-(y-y_0)^{\,2}}\,dy,
\end{equation}
where  $\beta = 2\,(k-k_0)\,{\mit\Delta}x$ and
$y_0 = - {\rm i}\,\beta/2$. The integral in the above equation is now just a number,
as can easily be seen by making the second change of variable $z=y-y_0$. 
Hence, we deduce that
\begin{equation}\label{e12.49}
\bar{\psi}(k) \propto \exp\left[-{\rm i}\,k\,x_0 - \frac{(k-k_0)^{\,2}}{4\,({\mit\Delta}k)^2}\right],
\end{equation}
where
\begin{equation}
{\mit\Delta} k = \frac{1}{2\,{\mit\Delta} x}.
\end{equation}

Now, if $|\psi(x,0)|^{\,2}\,dx$ is proportional to the probability  of a measurement of  the
particle's position yielding a value in the range $x$ to $x+dx$ at time $t=0$ then it stands to reason that $|\bar{\psi}(k)|^{\,2}\,dk$
is proportional to the probability  of a measurement of the
particle's wavenumber yielding a value in the range $k$ to $k+dk$. (Recall that $p = \hbar\,k$,
so a measurement of the particle's wavenumber, $k$, is equivalent to a measurement of the particle's
momentum, $p$). According to Equation~(\ref{e12.49}),
\begin{equation}\label{e12.51}
|\bar{\psi}(k)|^{\,2} \propto \exp\left[- \frac{(k-k_0)^{\,2}}{2\,({\mit\Delta}k)^{\,2}}\right].
\end{equation}
Note that this probability distribution is a {\em Gaussian}\/ in $k$-space---see
Equation~(\ref{e12.41}) and Figure~\ref{f12.3}. Hence, a measurement of $k$ is
most likely to yield the value $k_0$, and very unlikely to yield
a value which differs from $k_0$ by more than
$3\,{\mit\Delta}k$. Incidentally,  as was previously mentioned in Section~\ref{s8.1},  a Gaussian is the {\em only}\/ mathematical function
in $x$-space which has the same form as its Fourier transform in $k$-space.

We have just seen that a wave packet with a Gaussian probability distribution of characteristic
width ${\mit\Delta} x$ in $x$-space [see Equation~(\ref{e12.41})] is equivalent  to a wave packet with a Gaussian probability distribution of characteristic width
${\mit\Delta} k$ in $k$-space [see Equation~(\ref{e12.51})],
where
\begin{equation}
{\mit\Delta}x\,{\mit\Delta} k = \frac{1}{2}.
\end{equation}
This illustrates an important property of wave packets. Namely, in order to
construct a packet which is highly localized in $x$-space ({\em i.e.}, with small ${\mit\Delta}x$)  we need
to combine plane waves with a very wide range of different $k$-values
({\em i.e.}, with large ${\mit\Delta}k$). Conversely, if we only combine
plane waves whose wavenumbers differ by a small amount ({\em i.e.}, if
${\mit\Delta}k$ is small) then the resulting wave packet is highly 
extended in $x$-space ({\em i.e.}, ${\mit\Delta}x$ is large).

Now, according to Section~\ref{s9.1}, a wave packet made up of a superposition of
plane waves that is strongly peaked around some
central wavenumber $k_0$ propagates at the {\em group velocity},
\begin{equation}
v_g = \frac{d\omega(k_0)}{dk},
\end{equation}
rather than the {\em phase velocity}, $v_p = (\omega/k)_{k_0}$, 
assuming that all of the constituent plane waves satisfy a dispersion relation of the form $\omega=\omega(k)$. Now,
for the case of matter waves, the dispersion relation is (\ref{e12.25r}). Thus, the associated group velocity is
\begin{equation}
v_g = \frac{\hbar\,k_0}{m} = \frac{p}{m},
\end{equation}
where $p=\hbar\,k_0$. Note that this velocity is {\em identical}\/ to the classical
velocity of a (non-relativistic) massive particle. We conclude that the matter wave dispersion relation (\ref{e12.25r}) is perfectly consistent
with classical physics, as long as we recognize  that particles must be identified with
{\em wave packets}\/ (which propagate at the group velocity) rather than plane waves (which propagate at the phase velocity). 

In Section~\ref{s9.1}, it was also demonstrated that the spatial extent of a  wave packet of initial  extent $(\Delta x)_0$
grows, as the packet evolves in time, like
\begin{equation}
\Delta x \simeq (\Delta x)_0 + \frac{d^2\omega(k_0)}{dk^2}\,\frac{t}{(\Delta x)_0},
\end{equation}
where $k_0$ is the packet's central wavenumber. Thus, it follows from the matter wave dispersion relation, (\ref{e12.25r}), that the
width of a particle wave packet grows in time as
\begin{equation}\label{e12.55}
\Delta x \simeq (\Delta x)_0 + \frac{\hbar}{m}\,\frac{t}{(\Delta x)_0}.
\end{equation}
For example, if an electron wave packet is initially localized in a region of atomic dimensions ({\em i.e.}, $\Delta x\sim 10^{-10}\,{\rm m}$)
then the width of the packet doubles in about $10^{-16}\,{\rm s}$. Clearly, particle
wave packets  spread out very rapidly indeed (in free space).

\section{Heisenberg's Uncertainty Principle}\label{sun}
According to the analysis contained in the previous section, a particle
wave packet that is initially localized in $x$-space, with characteristic
width ${\mit\Delta}x$, is also localized in $k$-space, with characteristic
width ${\mit\Delta}k= 1/(2\,{\mit\Delta} x)$. However, as time progresses,
the width of the wave packet in $x$-space increases [see Equation~(\ref{e12.55})], whilst that of the packet in $k$-space stays the same [since $\bar{\psi}(k)$ is given by Equation~(\ref{e12.45}) at all times.] Hence,
in general, we can say that
\begin{equation}
{\mit\Delta}x\,{\mit\Delta} k\gtapp \frac{1}{2}.
\end{equation}
Furthermore, we can interpret ${\mit\Delta}x$ and ${\mit\Delta} k$ as
characterizing our {\em uncertainty}\/ regarding the values of the particle's
position and wavenumber, respectively.

Now, a measurement of a particle's wavenumber, $k$, is equivalent to
a measurement of its momentum, $p$, since $p=\hbar \,k$. Hence,
an uncertainty in $k$ of order ${\mit\Delta} k$ translates to
an uncertainty in $p$ of order ${\mit\Delta}p=\hbar\,{\mit\Delta}k$.
It follows, from the above inequality, that
\begin{equation}\label{e12.58}
{\mit\Delta}x\,{\mit\Delta} p \gtapp \frac{\hbar}{2}. 
\end{equation}
This is the famous {\em Heisenberg uncertainty principle},
first proposed by Werner Heisenberg in 1927.
According to this principle, it is impossible to {\em simultaneously}\/
measure the position and momentum of a particle (exactly). Indeed, a good knowledge
of the particle's position implies a poor knowledge of its momentum,
and {\em vice versa}. Note that the uncertainty principle is a direct consequence of representing particles as waves.

It is apparent, from expression (\ref{e12.55}),
that a particle wave packet of initial spatial extent $({\mit\Delta} x)_0$
  spreads out in such a manner  that its spatial extent becomes
\begin{equation}\label{espread}
\Delta x\sim \frac{\hbar\,t}{m\,({\mit\Delta}x)_0}
\end{equation}
at large $t$.
It is easily demonstrated that this spreading of the wave packet is a consequence of the
uncertainty principle. Indeed, since the initial uncertainty in the particle's
position is $({\mit\Delta}x)_0$, it follows that the uncertainty in its
momentum is of order $\hbar/({\mit\Delta}x)_0$. This translates to an uncertainty
in velocity of ${\mit\Delta}v = \hbar/[m\,({\mit\Delta}x)_0]$. Thus,
if we imagine that part of the wave packet propagates at $v_0+ {\mit\Delta}v/2$, and another part at $v_0-{\mit\Delta}v/2$, where $v_0$ is
the mean propagation velocity, then it is clear that the wave packet will
 spread out as time progresses. Indeed, at large $t$, we expect the
width of the wave packet to be
\begin{equation}
\Delta x \sim {\mit\Delta}v\,t \sim  \frac{\hbar\,t}{m\,({\mit\Delta}x)_0},
\end{equation}
which is identical to Equation~(\ref{espread}). Evidently, the spreading of
a particle wave packet, as time progresses, should be interpreted as representing an increase
in our {\em uncertainty}\/ regarding the particle's position, rather than
an increase in the spatial extent of the particle itself.

\section{Collapse of the Wavefunction}\label{scoll}
Consider a spatially extended wavefunction, $\psi(x,t)$. According to our
usual interpretation, $|\psi(x,t)|^{\,2}\,dx$ is proportional to the
probability  of a measurement of the particle's position yielding 
a value in the range  $x$ to $x+dx$ at time $t$. Thus, if the wavefunction is extended then there is a wide
range of likely values that such a  measurement could give. 
Suppose, however, that we make  a measurement of the particle's position, and obtain the value $x_0$.
We now know that the particle is located at $x=x_0$.  
If we make another measurement, immediately after the first one, then
what value would we expect to obtain? Well, common sense tells us that
we should obtain the {\em same}\/ value, $x_0$, since the particle
cannot have shifted position appreciably in an infinitesimal  time interval. 
Thus, immediately after the first measurement, a measurement of
the particle's position is {\em certain}\/ to give the value $x_0$, and has
no chance of giving any other value. This implies that the
wavefunction must have {\em collapsed}\/ to some sort of  ``spike'' function,
centered on $x=x_0$. This idea is illustrated in Figure~\ref{f12.4}.
Of course, as soon as the wavefunction collapses, it  starts to
expand again, as described in the previous section. Thus, the second measurement
must be made reasonably quickly after the first one, otherwise the
same result will not necessarily be obtained.

\begin{figure}
\epsfysize=3.5in
\centerline{\epsffile{Chapter12/fig04.eps}}
\caption{\em Collapse of the wavefunction upon measurement of $x$.}\label{f12.4}   
\end{figure}

The above discussion illustrates an important point in wave
mechanics. Namely, that the wavefunction of a massive particle
changes {\em discontinuously}\/ (in time) whenever a measurement of the particle's position is made. We conclude that there are two types of time
evolution of the wavefunction in wave mechanics. First, there is a {\em smooth}\/ evolution which is governed
by Schr\"{o}dinger's equation. This evolution takes place {\em between}\/ measurements. Second, there is a {\em discontinuous}\/ evolution which
takes place each time a measurement is made.

\section{Stationary States}
Consider {\em separable}\/ solutions to Schr\"{o}dinger's equation of the form
\begin{equation}
\psi(x,t) = \psi(x)\,{\rm e}^{-{\rm i}\,\omega\,t}.
\end{equation}
According to (\ref{e12.18}),  such solutions have definite energies $E=\hbar\,\omega$. For this reason,
they are usually written
\begin{equation}\label{e12.62}
\psi(x,t) = \psi(x)\,{\rm e}^{-{\rm i}\,(E/\hbar)\,t}.
\end{equation}
Now, the probability of finding the particle between $x$ and $x+dx$ at time $t$ is
\begin{equation}
P(x,t) = |\psi(x,t)|^2\,dx = |\psi(x)|^2\,dx.
\end{equation}
Note that this probability is {\em time independent}. For this reason, wavefunctions of the
form (\ref{e12.62}) are known as {\em stationary states}. Moreover, $\psi(x)$ is called a {\em stationary
wavefunction}. Substituting  (\ref{e12.62}) into Schr\"{o}dinger's equation, (\ref{e12.21}), we
obtain the following expression for $\psi(x)$:
\begin{equation}\label{e12.64}
-\frac{\hbar^2}{2\,m}\,\frac{d^2\psi}{d x^2} + U(x)\,\psi = E\,\psi.
\end{equation}
Not surprisingly, the above equation is called  the {\em time independent Schr\"{o}d\-inger equation}. 

Consider a particle trapped in a one-dimensional square potential well, of infinite depth, which is such that
\begin{equation}
U(x) = \left\{
\begin{array}{lll}
0&\mbox{\hspace{0.5cm}}&0\leq x \leq a\\[0.5ex]
\infty &&\mbox{otherwise}
\end{array}\right..
\end{equation}
The particle is obviously excluded from the region $x<0$ or $x>a$, so $\psi=0$ in this region ({\em i.e.}, there
is zero probability of finding the particle outside the well). Within the  well, a particle
of definite energy $E$ has a stationary wavefunction, $\psi(x)$,  which satisfies
\begin{equation}\label{e12.66}
-\frac{\hbar^2}{2\,m}\,\frac{d^2\psi}{d x^2}  = E\,\psi.
\end{equation}
The boundary conditions are
\begin{equation}\label{e12.67}
\psi(0) = \psi(a) = 0.
\end{equation}
This follows because $\psi=0$ in the region $x<0$ or $x>a$, and $\psi(x)$ must be {\em continuous}\/ [since a discontinuous
wavefunction would generate a singular term ({\em i.e.}, the term involving $d^2\psi/dx^2$) in the time independent Schr\"{o}dinger equation, (\ref{e12.64}),
which could not be balanced, even by an infinite potential]. 

Let us search for solutions to (\ref{e12.66}) of the form
\begin{equation}\label{e12.68}
\psi(x) = \psi_0\,\sin(k\,x),
\end{equation}
where $\psi_0$ is a constant. It follows that
\begin{equation}\label{e12.69x}
\frac{\hbar^2\,k^2}{2\,m} = E.
\end{equation}
The solution (\ref{e12.68}) automatically satisfies the boundary condition $\psi(0)=0$. The second boundary
condition, $\psi(a)=0$, leads to a quantization of the wavenumber: {\em i.e.}, 
\begin{equation}\label{e12.69}
k= n\,\frac{\pi}{a},
\end{equation}
where $n=1,\,2,\,3,$  {\em etc.} (Note that a ``quantized'' quantity is one which can only take discrete values.) According to (\ref{e12.69x}), the energy is also quantized.
In fact, $E=E_n$, where 
\begin{equation}
E_n = n^2\,\frac{\hbar^2\,\pi^2}{2\,m\,a^2}.
\end{equation}
Thus the allowed wavefunctions for a particle  trapped in a one-dimensional square potential well of infinite depth are
\begin{equation}\label{e12.72}
\psi_n(x,t) = A_n\,\sin\left(n\,\pi\,\frac{x}{a}\right)\,\exp\left(-{\rm i}\,n^2\,\frac{E_1}{\hbar}\,t\right),
\end{equation}
where $n$ is a positive integer, and $A_n$ a constant. Note that we cannot have $n=0$, since, in this case, we obtain
a null wavefunction: {\em i.e.}, $\psi=0$, everywhere. Furthermore, if $n$ takes a negative integer value
then it generates exactly the same wavefunction as the corresponding positive integer value  (assuming $\psi_{-n}=-\psi_n$). 

The constant $A_n$, appearing in the above wavefunction,  can be determined from the constraint that the
wavefunction be properly normalized. For the  problem presently under consideration, the normalization condition (\ref{e12.24r})
reduces to
\begin{equation}
\int_0^a\,|\psi(x)|^2\,dx = 1.
\end{equation}
It follows from (\ref{e12.72}) that $|A_n|^2=2/a$. Hence, a properly normalized version of the  wavefunction (\ref{e12.72})
is 
\begin{equation}
\psi_n(x,t) = \left(\frac{2}{a}\right)^{1/2}\,\sin\left(n\,\pi\,\frac{x}{a}\right)\,\exp\left(-{\rm i}\,n^2\,\frac{E_1}{\hbar}\,t\right).
\end{equation}
Figure~\ref{f12.5} shows the first four properly normalized stationary wavefunctions for a particle trapped in a one-dimensional
square potential well of infinite depth: {\em i.e.},  $\psi_n(x)= \sqrt{2/a}\,\sin(n\,\pi\,x/a)$,
for $n=1$ to $4$. 

\begin{figure}
\epsfysize=4in
\centerline{\epsffile{Chapter12/fig05.eps}}
\caption{\em First four stationary wavefunctions for  a particle trapped in a one-dimensional
square potential well of infinite depth.}\label{f12.5}   
\end{figure}

Note that the stationary wavefunctions that we have just found are, in essence, {\em standing wave}\/ solutions to Schr\"{o}dinger's equation. 
Indeed, the wavefunctions are very similar in form to the classical standing wave solutions discussed in Chapters~\ref{c5}
and \ref{c6}.  

At first sight, it seems rather strange that the lowest energy that a particle trapped in a one-dimensional
potential well can have is not zero, as would be the case in classical mechanics, but rather $E_1= \hbar^2\,\pi^2/(2\,m\,a^2)$. In fact,
as explained in the following, this residual energy is a direct consequence of {\em Heisenberg's uncertainty principle}. Now,  a
particle trapped in a one-dimensional well of width $a$ is likely to be found
anywhere inside the well. Thus, the uncertainty in the particle's position is $\Delta x\sim a$. It
follows from the uncertainty principle, (\ref{e12.58}), that
\begin{equation}
\Delta p \gtapp \frac{\hbar}{2\,\Delta x}\sim \frac{\hbar}{a}.
\end{equation}
In other words, the particle {\em cannot}\/ have zero momentum. In fact, the particle's momentum
must be at least $p\sim \hbar/a$. 
However, for a free particle, $E=p^2/2\,m$. Hence, the residual energy associated with the
particle's residual momentum is
\begin{equation}
E \sim \frac{p^2}{m}\sim \frac{\hbar^2}{m\,a^2}\sim E_1.
\end{equation}
This type of residual energy, which is often found in quantum mechanical systems, and has no equivalent in classical
mechanics, is generally known as {\em zero point energy}. 

\section{Three-Dimensional Wave Mechanics}
Up to now, we have only discussed wave mechanics for a particle moving in one dimension. However, the
generalization to a particle moving in three dimensions is fairly straightforward. 
 A massive particle moving in three dimensions
has a complex wavefunction of the form [{\em cf.}, (\ref{e12.13a})]
\begin{equation}
\psi(x,y,z,t) = \psi_0\,{\rm e}^{\,{\rm i}\,({\bf k}\cdot{\bf r}-\omega\,t)},
\end{equation}
where $\psi_0$ is a complex constant, and ${\bf r}= (x,\,y,\,z)$. Here, the wavevector, ${\bf k}$, and
the angular frequency, $\omega$, are related to the particle momentum, ${\bf p}$,  and energy, $E$, according
to [{\em cf.}, (\ref{e12.3})]
\begin{equation}
{\bf p} = \hbar\,{\bf k},
\end{equation}
and [{\em cf.}, (\ref{e12.1})]
\begin{equation}
E = \hbar\,\omega,
\end{equation}
 respectively. Generalizing the
analysis of Section~\ref{s12.5}, the three-dimensional version of Schr\"{o}dinger's
equation is easily shown to take the form [{\em cf.}, (\ref{e12.21})]
\begin{equation}\label{e12.78}
{\rm i}\,\hbar\,\frac{\partial\psi}{\partial t} = - \frac{\hbar^2}{2\,m}\,\nabla^2\psi + U({\bf r})\,\psi,
\end{equation}
where the differential operator
\begin{equation}
\nabla^2 \equiv \frac{\partial^2}{\partial x^2} + \frac{\partial^2 }{\partial y^2} + \frac{\partial^2}{\partial z^2}
\end{equation}
is known as the {\em Laplacian}. The interpretation of a three-dimensional wavefunction is that the
probability of finding the particle between $x$ and $x+dx$, between $y$ and $y+dy$, and
between $z$ and $z+dz$, at time $t$ is [{\em cf.}, (\ref{e12.25})]
\begin{equation}
P(x,y,z,t) = |\psi(x,y,z,t)|^2\,dx\,dy\,dz.
\end{equation}
Moreover, the normalization condition for the wavefunction becomes [{\em cf.}, (\ref{e12.24r})]
\begin{equation}\label{e12.81}
\int_{-\infty}^\infty\int_{-\infty}^\infty\int_{-\infty}^\infty|\psi(x,y,z,t)|^2\,dx\,dy\,dz =1.
\end{equation}
Incidentally, it is easily demonstrated that Schr\"{o}dinger's equation, (\ref{e12.78}), preserves the normalization
condition, (\ref{e12.81}), of a {\em localized}\/ wavefunction. 
Heisenberg's uncertainty principle generalizes to [{\em cf.}, (\ref{e12.58})]
\begin{eqnarray}
\Delta x\,\Delta p_x&\gtapp &\frac{\hbar}{2},\\[0.5ex]
\Delta y\,\Delta p_y&\gtapp &\frac{\hbar}{2},\\[0.5ex]
\Delta z\,\Delta p_z&\gtapp &\frac{\hbar}{2}.
\end{eqnarray}
Finally, a stationary state of energy $E$ is written [{\em cf.}, (\ref{e12.62})]
\begin{equation}
\psi(x,y,z,t) = \psi(x,y,z)\,{\rm e}^{-{\rm i}\,(E/\hbar)\,t},
\end{equation}
where the stationary wavefunction, $\psi(x,y,z)$, satisfies [{\em cf.}, (\ref{e12.64})]
\begin{equation}
 - \frac{\hbar^2}{2\,m}\,\nabla^2\psi + U({\bf r})\,\psi = E\,\psi.
\end{equation}

As an example of a three-dimensional problem in wave mechanics, consider a particle trapped in a  square potential well  of infinite depth which is
such that
\begin{equation}
U(x,y,z) = \left\{
\begin{array}{lll}
0&\mbox{\hspace{0.5cm}}&0\leq x \leq a,\, 0\leq y \leq a,\, 0\leq z\leq a\\[0.5ex]
\infty &&\mbox{otherwise}
\end{array}\right..
\end{equation}
Within the well, the stationary wavefunction, $\psi(x,y,z)$, satisfies
\begin{equation}\label{e12.87}
- \frac{\hbar^2}{2\,m}\,\nabla^2\psi  = E\,\psi,
\end{equation}
subject to the boundary conditions
\begin{equation}\label{e12.88}
\psi(0,y,z) = \psi(x,0,z)=\psi(x,y,0) =0,
\end{equation}
and
\begin{equation}\label{e12.89}
\psi(a,y,z) = \psi(x,a,z)=\psi(x,y,a) =0,
\end{equation}
since $\psi=0$ outside the well.
Let us try a seperable wavefunction of the form
\begin{equation}\label{e12.90}
\psi(x,y,z) =\psi_0\,\sin(k_x\,x)\,\sin(k_y\,y)\,\sin(k_z\,z).
\end{equation}
This expression automatically satisfies the boundary conditions (\ref{e12.88}). The
remaining boundary conditions, (\ref{e12.89}),  are satisfied provided
\begin{eqnarray}\label{e12.92}
k_x &=& n_x\,\frac{\pi}{a},\\[0.5ex]
k_y &=&n_y\,\frac{\pi}{a},\\[0.5ex]
k_z &=& n_z\,\frac{\pi}{a},\label{e12.94}
\end{eqnarray}
where $n_x$, $n_y$, and $n_z$ are (independent) {\em positive integers}. 
Substitution of the wavefunction (\ref{e12.90}) into Equation~(\ref{e12.87})
yields
\begin{equation}
E = \frac{\hbar^2}{2\,m}\,(k_x^{\,2} + k_y^{\,2}+k_z^{\,2}).
\end{equation}
Thus, it follows from Equations~(\ref{e12.92})--(\ref{e12.94}) that the particle energy is quantized, and that the
allowed {\em energy levels}\/ are
\begin{equation}\label{e12.95}
E_{n_x,n_y,n_z} = \frac{\hbar^2}{2\,m\,a^2}\,(n_x^{\,2}+n_y^{\,2}+n_z^{\,2}).
\end{equation}
The properly normalized [see Equation~(\ref{e12.81})] stationary wavefunctions corresponding to
these energy levels are
\begin{equation}
\psi_{n_x,n_y,n_z}(x,y,z) = \left(\frac{2}{a}\right)^{3/2}\,\sin\left(n_x\,\pi\,\frac{x}{a}\right)\,\sin\left(n_y\,\pi\,\frac{y}{a}\right)\,\sin\left(n_z\,\pi\,\frac{z}{a}\right).
\end{equation}

As is the case for a particle trapped in a one-dimensional  potential well, the lowest
energy level for a particle trapped in a three-dimensional  well is not zero, but rather
\begin{equation}
E_{1,1,1} = 3\,E_1.
\end{equation}
Here, 
\begin{equation}
E_1 =  \frac{\hbar^2}{2\,m\,a^2}.
\end{equation}
is the {\em ground state}\/ ({\em i.e.}, the lowest energy state) energy in the one-dimension\-al
case.  
Now, it is clear, from (\ref{e12.95}), that distinct permutations of $n_x$, $n_y$, and $n_z$ which do
not alter the value of $n_x^{\,2}+n_y^{\,2}+ n_z^{\,2}$ also do not alter the energy. In other words, in three dimensions
it is possible for distinct wavefunctions to be associated with the same energy level. In this
situation, the energy level is said to be {\em degenerate}. The ground state energy level, $3\,E_1$, is non-degenerate,
since the only combination of ($n_x$, $n_y$,  $n_z$) which gives this energy is ($1$, $1$, $1$). 
However, the next highest energy level, $6\,E_1$, is degenerate, since it is obtained when ($n_x$, $n_y$,  $n_y$) 
take the values ($2$, $1$, $1$),
or ($1$, $2$, $1$), or ($1$, $1$, $2$). In fact, it is not difficult to see that a non-degenerate energy
level corresponds to a case where the three {\em mode numbers}\/ ({\em i.e.}, $n_x$, $n_y$, and $n_z$) all have
the same value, whereas a three-fold degenerate energy level corresponds to a case where only
two of the mode numbers have the same value, and, finally,  a six-fold degenerate energy level corresponds to a
case where the mode numbers are all different. 

\section{Particle in a Finite Potential Well}
Consider, now, a particle of mass $m$ trapped in a one-dimensional  square potential
well of width $a$ and finite depth $V>0$. In fact, suppose that the potential takes the form
\begin{equation}
U(x) = \left\{
\begin{array}{lll}
-V &\mbox{\hspace{0.5cm}}&|x|\leq a/2\\[0.5ex]
0&&\mbox{otherwise}
\end{array}
\right..\label{e12.100}
\end{equation}
Here, we have adopted the standard convention that $U(x)\rightarrow 0$ as $|x|\rightarrow \infty$. 
This convention is useful because, just like in classical mechanics, a particle whose overall energy, $E$,
is negative is bound in the well ({\em i.e.}, it cannot escape to infinity), whereas a
particle whose overall energy is positive is unbound. Since we are interested in bound particles,
we shall assume that $E<0$. We shall also assume that $E+V>0$,  in order to allow the particle
to have a positive kinetic energy inside the well. 

Let us search for a stationary state 
\begin{equation}
\psi(x,t) = \psi(x)\,{\rm e}^{-{\rm i}\,(E/\hbar)\,t},
\end{equation}
whose stationary wavefunction, $\psi(x)$, satisfies the time independent Schr\"{o}d\-inger equation, 
(\ref{e12.64}). Now, it is easily appreciated that the solutions to (\ref{e12.64}) in the symmetric
 [{\em i.e.}, $U(-x)=U(x)$] potential (\ref{e12.100}) must be either totally
 symmetric [{\em i.e.}, $\psi(-x)=\psi(x)$] or totally antisymmetric [{\em i.e.}, $\psi(-x)=-\psi(x)$]. 
 Moreover, the solutions must satisfy the boundary condition
 \begin{equation}
 \psi\rightarrow 0\mbox{\hspace{0.5cm}as\hspace{0.5cm}$|x|\rightarrow\infty$},
 \end{equation}
 otherwise they would not correspond to bound states.
 
 Let us, first of all, search for a totally symmetric solution. In the region to the left of the well ({\em i.e.}, $x<-a/2$),
 the solution of the time independent Schr\"{o}dinger equation which satisfies the boundary condition
 $\psi\rightarrow 0$ as $x\rightarrow -\infty$ is
 \begin{equation}
 \psi(x) = A\,{\rm e}^{\,q\,x},
 \end{equation}
 where
 \begin{equation}
 q = \sqrt{\frac{2\,m\,(-E)}{\hbar^2}},
 \end{equation}
 and $A$ is a constant. By symmetry, the solution in the region to the right of the well ({\em i.e.}, $x>a/2$)
 is
 \begin{equation}\label{e12.105}
 \psi(x) = A\,{\rm e}^{- q\,x}.
 \end{equation}
 The solution inside the well ({\em i.e.}, $|x|\leq a/2$) which satisfies the symmetry
 constraint $\psi(-x)=\psi(x)$ is
 \begin{equation}\label{e12.106}
 \psi(x) = B\,\cos(k\,x),
 \end{equation}
 where
 \begin{equation}
 k= \sqrt{\frac{2\,m\,(V+E)}{\hbar^2}},
 \end{equation}
 and $B$ is a constant.
 The appropriate matching conditions at the edges of the well ({\em i.e.}, $x=\pm a/2$) are that $\psi(x)$ and $d\psi(x)/dx$
 both be {\em continuous}\/ [since  a discontinuity in the
wavefunction, or  its first derivative, would generate a singular term in the time independent Schr\"{o}dinger equation 
({\em i.e.}, the term involving $d^2\psi/dx^2$) which could not be balanced]. The matching conditions yield
\begin{equation}\label{e12.108}
q = k\,\tan(k\,a/2).
\end{equation}

Let $y=k\,a/2$. It follows that 
\begin{equation}
E = E_0\,y^2 - V,
\end{equation}
where
\begin{equation}
E_0 = \frac{2\,\hbar^2}{m\,a^2}.
\end{equation}
Moreover, Equation~(\ref{e12.108}) becomes
\begin{equation}\label{e12.111}
\frac{\sqrt{\lambda-y^2}}{y} = \tan y,
\end{equation}
with
\begin{equation}
\lambda = \frac{V}{E_0}.
\end{equation}
Here, $y$ must lie in the range $0<y<\sqrt{\lambda}$, in order to ensure that $E$
lies in the range $-V<E<0$.

\begin{figure}
\epsfysize=3in
\centerline{\epsffile{Chapter12/fig06.eps}}
\caption{\em The curves $\tan y$ (solid) and $\sqrt{\lambda - y^2}/y$ (dashed), calculated for $\lambda = 1.5\,\pi^2$. The latter curve takes the
value $0$ when $y>\sqrt{\lambda}$. }\label{f12.6}   
\end{figure}

Now, the solutions of Equation~(\ref{e12.111}) correspond to the
intersection of the curve $\sqrt{\lambda - y^2}/y$ with the curve
$\tan y$. Figure~\ref{f12.6} shows these two curves plotted for
a particular value of $\lambda$. In this case, the curves intersect
twice, indicating the existence of two totally symmetric bound states in the well.
Moreover, it is clear, from the figure, that as $\lambda$ increases ({\em i.e.}, as the well becomes
deeper) there are more and more bound states. However, it is also apparent  that there is
always at least one totally symmetric bound state, no matter how small $\lambda$
becomes ({\em i.e.}, no matter how shallow the well becomes). In the limit $\lambda\gg 1$
({\em i.e.}, the limit in which the well is very deep), the
solutions to Equation~(\ref{e12.111}) asymptote to the roots of $\tan y =\infty$.
This gives $y = (2\,n-1)\,\pi/2$, where $n$ is a positive integer, or
\begin{equation}
k = (2\,n-1)\,\frac{\pi}{a}.
\end{equation}
These solutions are equivalent to the odd-$n$ infinite-depth potential well solutions 
specified by Equation~(\ref{e12.69}).

\begin{figure}
\epsfysize=3in
\centerline{\epsffile{Chapter12/fig07.eps}}
\caption{\em The curves $\tan y$ (solid) and $-y/\sqrt{\lambda - y^2}$ (dashed), calculated for $\lambda = 1.5\,\pi^2$. }\label{f12.7}   
\end{figure}

For the case of a totally antisymmetric bound state, similar analysis to the
above yields (see Exercise 12.3)
\begin{equation}\label{e12.113}
-\frac{y}{\sqrt{\lambda-y^2}} = \tan y.
\end{equation}
The solutions of this equation correspond to the intersection of the
curve $\tan y$ with the curve  $-y/\sqrt{\lambda-y^2}$. Figure~\ref{f12.7} shows these two curves plotted for
the same value of $\lambda$ as that used in Figure~\ref{f12.6}. In this
case, the curves intersect once, indicating the existence of
a single totally antisymmetric bound state in the well. It is, again, clear, from the figure, that as $\lambda$ increases ({\em i.e.}, as the well becomes
deeper) there are more and more bound states. However, it is also apparent that
when $\lambda$ becomes sufficiently small [{\em i.e.}, $\lambda < (\pi/2)^2$] then there is no totally
antisymmetric bound state. In other words, a very shallow potential well
always possesses a totally symmetric bound state, but does not generally
possess a totally antisymmetric bound state. In the limit $\lambda\gg 1$
({\em i.e.}, the limit in which the well becomes very deep), the
solutions to Equation~(\ref{e12.113}) asymptote to the roots of $\tan y =0$.
This gives $y = n\,\pi$, where $n$ is a positive integer, or
\begin{equation}
k= 2\,n\,\frac{\pi}{a}.
\end{equation}
These solutions are equivalent to the even-$n$ infinite-depth potential well solutions 
specified by Equation~(\ref{e12.69}).

Probably the most surprising aspect of the bound states that we have just
described is the possibility  of finding the particle {\em outside}\/ the well: {\em
i.e.}, in the region $|x|>a/2$ where $U(x)>E$. This follows from Equation~(\ref{e12.105}) and (\ref{e12.106}) 
because the ratio $A/B= \exp(q\,a/2)\,\cos(k\,a/2)$ is not necessarily zero.
Such behavior is strictly forbidden
in classical mechanics, according to which  a  particle of energy $E$ is restricted to
regions of space where $E>U(x)$. In fact, in the case of the ground state ({\em i.e.}, the lowest
energy symmetric state) it is possible to demonstrate that the probability of a
measurement finding the particle outside the well is (see Exercise 12.4)
\begin{equation}
P_{\rm out}\simeq 1-2\,\lambda
\end{equation}
for a shallow well ({\em i.e.}, $\lambda\ll 1$), and
\begin{equation}
P_{\rm out}\simeq \frac{\pi^2}{4}\,\frac{1}{\lambda^{3/2}}
\end{equation}
for a deep well ({\em i.e.}, $\lambda\gg 1$). It follows that the particle is very likely to be found outside a shallow well, and there is a small, but finite, probability of it being found outside a deep well.
In fact, the probability of
finding the particle outside the well only goes to zero   in the case of an infinitely deep well ({\em i.e.}, $\lambda\rightarrow \infty$).

\section{Square Potential Barrier}
Consider a particle of mass $m$ and energy $E>0$ interacting with the
simple  potential barrier
\begin{equation}
U(x) = \left\{\begin{array}{lcl}
V&\mbox{\hspace{1cm}}&\mbox{for $0\leq x\leq a$}\\[0.5ex]
0&&\mbox{otherwise}
\end{array}
\right.,
\end{equation}
where $V>0$. In the regions to the left and to the right of the
barrier, the stationary wavefunction, $\psi(x)$, satisfies
\begin{equation}\label{e12.15f}
\frac{d^2 \psi}{d x^2} = - k^2\,\psi,
\end{equation}
where
\begin{equation}
k = \sqrt{\frac{2\,m\,E}{\hbar^2}}.
\end{equation}

 Let us adopt the following solution
of the above equation to the left of the barrier ({\em i.e.}, $x<0$):
\begin{equation}
\psi(x) = {\rm e}^{\,{\rm i}\,k\,x} + R\,{\rm e}^{-{\rm i}\,k\,x}.
\end{equation}
This solution consists of a plane wave of unit amplitude traveling to
the right [since the full  wavefunction is multiplied by a factor
$\exp(-{\rm i}\,E\,t/\hbar$)], and a plane wave of complex amplitude $R$ traveling to
the left. We interpret the first plane wave as an {\em incoming particle}, and
the second as a particle {\em reflected}\/ by the potential barrier. Hence, $|R|^{\,2}$ is
the probability of reflection (see Section~\ref{srefl}).

Let us adopt the following solution to Equation~(\ref{e12.15f}) to the right
of the barrier ({\em i.e.} $x>a$):
\begin{equation}
\psi(x) = T\,{\rm e}^{\,{\rm i}\,k\,x}.
\end{equation}
This solution consists of a plane wave of complex amplitude $T$
traveling to the right. We interpret this as a particle {\em transmitted}\/ through
the barrier. Hence, $|T|^{\,2}$ is the probability of transmission. 

Let us consider the situation in which  $E< V$. In this case, according to classical mechanics, the particle is
unable to penetrate the barrier, so the coefficient of reflection is unity, and the coefficient
of transmission  zero. 
However, this is not necessarily the case in wave mechanics.
In fact, inside the barrier ({\em i.e.}, $0\leq x \leq a$), $\psi(x)$ satisfies
\begin{equation}\label{e12.21f}
\frac{d^2 \psi}{d x^2} =  q^2\,\psi,
\end{equation}
where
\begin{equation}
q = \sqrt{\frac{2\,m\,(V-E)}{\hbar^2}}.
\end{equation}
The general
solution to Equation~(\ref{e12.21f})  takes the
form
\begin{equation}
\psi(x) = A\,{\rm e}^{\,q\,x} + B\,{\rm e}^{-q\,x}.
\end{equation}

Now, continuity of $\psi$ and $d \psi/d x$ at the left edge of
the barrier ({\em i.e.}, $x=0$) yields
\begin{eqnarray}\label{e12.126}
1 + R &=& A+B,\\[0.5ex]
{\rm i}\,k\,(1-R) &=& q\,(A-B).
\end{eqnarray}
Likewise, continuity of $\psi$ and $d\psi/d x$ at the right edge of
the barrier ({\em i.e.}, $x=a$) gives
\begin{eqnarray}
A\, {\rm e}^{\,q\,a}+ B \,{\rm e}^{-q\,a} &=& T\,{\rm e}^{\,{\rm i}\,k\,a},\\[0.5ex]
q\left(A\, {\rm e}^{\,q\,a}-B \,{\rm e}^{-q\,a}\right) &=& {\rm i}\,k\,T\,{\rm e}^{\,{\rm i}\,k\,a}.\label{e12.129}
\end{eqnarray}
After considerable algebra (see Exercise~12.5), the above four equations yield
\begin{equation}\label{e12.37f}
|T|^{\,2} = 1-|R|^{\,2}= \frac{4\,k^2\,q^2}{4\,k^2\,q^2 + (k^2+q^2)^{\,2}\,\sinh^2(q\,a)}.
\end{equation}
Here, $\sinh x\equiv (1/2)\,({\rm e}^{\,x}-{\rm e}^{-x})$. 
The fact that $|R|^{\,2}+|T|^{\,2}=1$ ensures that the probabilities of reflection and
transmission sum to unity, as must be the case, since reflection and transmission
are the only possible outcomes for a particle incident on the barrier.
Note that, according to Equation~(\ref{e12.37f}), the probability of transmission 
is not necessarily zero. This means that, in wave mechanics, there is a finite probability for a particle incident on
a potential barrier, of finite width, to penetrate through the barrier, and reach the other side, even when the barrier
is sufficiently high to completely reflect the particle according to the laws of classical mechanics. This
strange phenomenon is known as {\em tunneling}. For the case of a very high barrier, such that
$V\gg E$, the tunneling probability reduces to
\begin{equation}
|T|^{\,2}\simeq \frac{4\,E}{V}\,{\rm e}^{-2\,a/\lambda},
\end{equation}
where $\lambda = \sqrt{\hbar^2/2\,m\,V}$ is the de Broglie wavelength inside the barrier. Here, it
is assumed that $a\gg\lambda$. 
Note that, even in the limit in which  the barrier is very high, there is an exponentially small, but nevertheless {\em non-zero}, tunneling probability. Tunneling plays an important role in the physics
of $\alpha$-decay and electron field emission. 

\section{Exercises}
{\small
\begin{enumerate}
\item Use the standard power law expansions, 
\begin{eqnarray}
{\rm e}^x &=& 1 + x + \frac{x^2}{2!} + \frac{x^3}{3!}+\cdots,\nonumber\\[0.5ex]
\sin x &=& x - \frac{x^3}{3!} + \frac{x^5}{5!}-\frac{x^7}{7!}+\cdots,\nonumber\\[0.5ex]
\cos x &=&1-\frac{x^2}{2!} + \frac{x^4}{4!}-\frac{x^6}{6!}+\cdots,\nonumber
\end{eqnarray}
which are valid for complex $x$, 
to prove de Moivre's theorem,
$$
{\rm e}^{\,{\rm i}\,\theta} = \cos\theta + {\rm i}\,\sin\theta,
$$
where $\theta$ is real.
\item Equations~(\ref{e8.27}) and (\ref{e8.28}) can be combined with de Moivre's theorem to give
$$
\delta(k) = \frac{1}{2\pi}\int_{-\infty}^\infty {\rm e}^{\,{\rm i}\,k\,x}\,dx,
$$
where $\delta(k)$ is a Dirac delta function. 
Use this result to prove Fourier's theorem: {\em i.e.}, if
$$
f(x) = \int_{-\infty}^{\infty} \bar{f}(k)\,{\rm e}^{\,{\rm i}\,k\,x}\,dk,
$$
then
$$
\bar{f}(k) = \frac{1}{2\pi}\int_{-\infty}^\infty f(x)\,{\rm e}^{-{\rm i}\,k\,x}\,dx.
$$

\item Derive Equation~(\ref{e12.113}).

\item Consider a particle trapped in the finite potential well whose potential is given by Equation~(\ref{e12.100}). 
Demonstrate that for a totally symmetric state the ratio of the probability  of finding the particle outside to the
probability of finding the particle inside the well is
$$
\frac{P_{\rm out}}{P_{\rm in}}= \frac{\cos^3 y}{\sin y\,(y + \sin y\,\cos y)},
$$
where 
$\sqrt{\lambda-y^2} = y\,\tan y$, and 
$\lambda = V/E_0$. Hence, demonstrate that for a shallow well ({\em i.e.}, $\lambda\ll 1$) $P_{\rm out}\simeq 1 - 2\,\lambda$, 
whereas for a deep well ({\em i.e.}, $\lambda\gg 1$) $P_{\rm out}\simeq (\pi^2/4) / \lambda^{3/2}$.

\item Derive expression (\ref{e12.37f}) from Equations~(\ref{e12.126})--(\ref{e12.129}).

\item Show that the coefficient of transmission of a particle of mass $m$ and energy $E$, incident on a square
potential barrier of height $V<E$, and  width $a$, is
$$
|T|^{\,2} = \frac{4\,k^2\,q^2}{4\,k^2\,q^2 + (k^2-q^2)^{\,2}\,\sin^2(q\,a)},
$$
where $k=\sqrt{2\,m\,E/\hbar^2}$ and $q=\sqrt{2\,m\,(E-V)/\hbar^2}$. Demonstrate that the coefficient of transmission
is unity ({\em i.e.}, there is no reflection from the barrier) when $q\,a=n\,\pi$, where $n$ is positive
integer. 

\item A He-Ne laser emits radiation of wavelength $\lambda =633\,{\rm nm}$. How many photons are emitted per second
by a laser with a power of $1\,{\rm mW}$? What force does such a laser exert on a body which completely absorbs
its radiation?
\item The ionization energy of a hydrogen atom in its ground state is $E_{\rm ion}= 13.6\,{\rm eV}$. Calculate the
frequency (in Hertz), wavelength, and wavenumber of the electromagnetic radiation which will just ionize the atom.
\item The maximum energy of photoelectrons from aluminium is $2.3\,{\rm eV}$ for radiation of wavelength $200\,{\rm nm}$,
and $0.90\,{\rm eV}$ for radiation of wavelength $258\,{\rm nm}$. Use this data to calculate Planck's
constant (divided by $2\pi$) and the work function of aluminium.
\item Show that the de Broglie wavelength of an electron accelerated across a potential difference $V$
is given by
$$
\lambda = 1.29\times 10^{-9}\,V^{-1/2}\,{\rm m},
$$
where $V$ is measured in volts.
\item If the atoms in a regular crystal are separated by $3\times 10^{-10}\,{\rm m}$ demonstrate that an accelerating
voltage of about $3\,{\rm kV}$  would be required to produce an electron diffraction pattern from the crystal. 
\item A particle of mass $m$ has a wavefunction
$$
\psi(x,t)= A\,\exp\left[-a\,(m\,x^2/\hbar+ {\rm i}\,t)\right],
$$
where $A$ and $a$ are positive real constants. For what potential $U(x)$ does $\psi(x,t)$ satisfy
Schr\"{o}dinger's equation?
\item Show that the wavefunction of a particle of mass $m$ trapped in a  one-dimensio\-nal square potential well of 
of width $a$, and infinite depth, returns to its original form after a quantum revival time $T=4\,m\,a^2/\pi\,\hbar$. 
\item Show that the normalization constant for the stationary wavefunction
$$
\psi(x,y,z) = A\,\sin\left(n_x\,\pi\,\frac{x}{a}\right)\sin\left(n_y\,\pi\,\frac{y}{b}\right)\sin\left(n_z\,\pi\,\frac{z}{c}\right)
$$
describing an electron trapped in a three-dimensional rectangular potential well of dimensions $a$, $b$, $c$, and
infinite depth, is $A=(8/abc)^{1/2}$. Here, $n_x$, $n_y$, and
$n_z$ are positive integers. 
\item An electron of momentum $p$ passes through a slit of width $\Delta x$. Its diffraction as a wave can
be regarded in terms of a change of its momentum $\Delta p$ in a direction parallel to the plane of the
slit (the total momentum remaining constant). Show that the approximate position of the first maximum
of the diffraction pattern is in accordance with Heisenberg's uncertainty principle. 
\item The probability of a particle of mass $m$ penetrating a distance $x$ into a classically
forbidden region is proportional to ${\rm e}^{-2\,\alpha\,x}$, where
$$
\alpha^2 = 2\,m\,(V-E)/\hbar^2.
$$
If $x=2\times 10^{-10}\,{\rm m}$ and $V-E = 1\,{\rm eV}$ show that ${\rm e}^{-2\,\alpha\,x}$ is equal to $0.1$
for an electron, and $10^{-43}$ for a proton.
\end{enumerate}
}