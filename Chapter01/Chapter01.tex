\chapter{Introduction}
Oscillations and waves
are  ubiquitous phenomena that are observed in a wide range of different  physical systems.
An {\em oscillation}\/ is a disturbance in a physical system that is {\em repetitive in time}. A {\em wave}\/ is a disturbance in an extended physical system that
is both {\em repetitive in time}\/ and {\em periodic in space}. 
In general, an oscillation
involves a continual back and forth flow of energy between two different energy types: {\em e.g.},
kinetic and potential energy, in the case of a pendulum. A wave 
involves similar repetitive energy flows to an oscillation, but, in addition, is
capable of transmitting energy (and information) from place to place.  Now, although sound waves and electromagnetic
waves, for example, rely on quite distinct physical mechanisms, they, nevertheless, share
many common properties. This is also true of different types of oscillation. It turns
out that the common factor linking the various types of wave is that they are all described by the {\em same}\/
mathematical equations. Again, this is also true of  the various types of oscillation.

 The aim of this course is  to develop a unified mathematical theory of
oscillations and waves in physical systems.  Examples will be drawn from the dynamics of discrete mechanical systems;  continuous gases, fluids, and elastic solids; electronic
circuits; electromagnetic waves; and quantum mechanical systems. 

This course assumes a basic  familiarity with the laws of physics, such
as might be obtained from a two-semester introductory college-level survey course. Students
are also assumed to be familiar with standard mathematics, up to and including
trigonometry, linear algebra, differential calculus, integral calculus,  ordinary differential equations,
partial differential equations, and Fourier series. 

The textbooks   which were consulted most often during the development of the
course material are:
\begin{description}
\item {\em Waves}, Berkeley Physics Course, Vol.~3, F.S.~Crawford, Jr.\ (McGraw-Hill, New York NY, 1968).
\item {\em Vibrations and Waves}, A.P.~French (W.W.~Norton \& Co., New York NY, 1971).
\item {\em Introduction to Wave Phenomena}, A.~Hirose, and K.E.~Lonngren (John
Wiley \& Sons, New York NY, 1985).
\item {\em The Physics of Vibrations and Waves}, 5th Edition, H.J.~Pain (John Wiley \& Sons, Chichester UK, 1999). 
\end{description}