\chapter{Transverse Standing Waves}\label{c5}
\section{Normal Modes of a Beaded String}\label{s5.1}
Consider a mechanical system consisting of a taut string which is stretched
between two immovable walls. Suppose that $N$ identical beads of
mass $m$ are attached to the string in such a manner that they cannot slide along it. Let the beads be equally
spaced a  distance $a$ apart, and let the distance between the first and the last beads and
the neighboring walls also be $a$. See Figure~\ref{f5.1}. Consider {\em transverse}\/ 
oscillations of the string: {\em i.e.}, oscillations in which the string
moves from side to side ({\em i.e.}, in the $y$-direction). 
It is assumed that the inertia of the string
is negligible with respect to that of the beads. It follows that the sections of the string
between neighboring beads, and between the outermost beads and the walls,
are {\em straight}. (Otherwise, there would be a net tension force acting on the sections, and
they would consequently suffer an infinite acceleration.) In fact, we expect the instantaneous configuration of the string to be a
set of continuous straight-line segments of varying inclinations, as shown in the figure. Finally, assuming that
the transverse displacement of the string is {\em relatively small}, it is
reasonable to suppose that each section of the string possesses the
{\em same}\/ tension, $T$. 

\begin{figure}
\epsfysize=1.6in
\centerline{\epsffile{Chapter05/fig01.eps}}
\caption{\em A beaded string.}\label{f5.1}   
\end{figure}

It is convenient to introduce a Cartesian coordinate system such  that $x$
measure distance along the string from the left wall, and
$y$ measures the transverse displacement of the string from its
equilibrium position. See Figure~\ref{f5.1}. Thus, when the string is in its equilibrium position
it runs along the $x$-axis. We can define
\begin{equation}
x_i = i\,a,\label{e5.1}
\end{equation}
where $i=1,2,\cdots, N$. Here, $x_1$ is the $x$-coordinate of the
closest bead to the left wall, $x_2$ the
$x$-coordinate of the second closest bead, {\em etc.} The $x$-coordinates of the beads
are assumed to remain {\em constant}\/ during the transverse oscillations. 
We can also define $x_0=0$
and $x_{N+1}= (N+1)\,a$ as the $x$-coordinates of the left
and right ends of the string. Let the transverse displacement
of the $i$th bead be $y_i(t)$, for $i=1,N$. Since each displacement
can vary independently, we are clearly dealing with an $N$ degree of freedom system. 
We would, therefore, expect such a system to possess $N$ unique normal modes of
oscillation. 

\begin{figure}
\epsfysize=2.5in
\centerline{\epsffile{Chapter05/fig02.eps}}
\caption{\em A short section of a beaded string.}\label{f5.2}   
\end{figure}

Consider the section of the
string lying between the $i-1$th and $i+1$th beads, as shown in Figure~\ref{f5.2}.
Here, $x_{i-1}=x_i-a$, $x_i$, and $x_{i+1}=x_i+a$ are the distances of the $i-1$th, $i$th, and
$i+1$th beads, respectively, 
from the left  wall, whereas $y_{i-1}$, $y_i$, and
$y_{i+1}$ are the corresponding transverse displacements of these beads. 
The  two sections of the string which are attached to the $i$th
bead subtend angles $\theta_i$ and $\theta_{i+1}$ with the $x$-axis, as illustrated
in the figure. Simple trigonometry reveals that
\begin{equation}
\tan\theta_i = \frac{y_i-y_{i-1}}{x_i-x_{i-1}} = \frac{y_i-y_{i-1}}{a},
\end{equation}
and
\begin{equation}
\tan\theta_{i+1} =   \frac{y_{i+1}-y_{i}}{a}.
\end{equation}
However, if the transverse displacement of the string is
relatively small---{\em i.e.}, if $|y_i|\ll a$ for all $i$---which we shall assume to be
the case, then $\theta_i$ and $\theta_{i+1}$ are both {\em small}\/ angles. Now, it is
well known that $\tan\theta\simeq \theta$ when $|\theta|\ll 1$. It follows that
\begin{eqnarray}\label{e5.4}
\theta_i &\simeq & \frac{y_i-y_{i-1}}{a},\\[0.5ex]
\theta_{i+1} &\simeq &  \frac{y_{i+1}-y_{i}}{a}.\label{e5.5}
\end{eqnarray}

Let us find the {\em transverse}\/ equation of motion of the $i$th bead. This
bead is subject to two forces: {\em i.e.}, the tensions in the sections
of the string to the left and to the right of it. (Incidentally, we are neglecting any gravitational forces
acting on the beads compared to the tension forces.) These tensions are of
magnitude $T$, and are directed parallel to the associated string sections, as shown
in Figure~\ref{f5.2}. Thus, the transverse ({\em i.e.}, $y$-directed) components of
these two tensions are $-T\,\sin\theta_i$  and $T\,\sin\theta_{i+1}$, respectively. 
Hence, the transverse equation of motion of the $i$th bead becomes
\begin{equation}
m\,\ddot{y}_i = -T\,\sin\theta_i + T\,\sin\theta_{i+1}.
\end{equation}
However, since $\theta_i$ and $\theta_{i+1}$ are both small angles, we
can employ the small angle approximation $\sin\theta\simeq \theta$. 
It follows that
\begin{equation}
\ddot{y}_i\simeq \frac{T}{m}\left(\theta_{i+1}-\theta_i\right).
\end{equation}
Finally, making use of Equations~(\ref{e5.4}) and (\ref{e5.5}),
we obtain
\begin{equation}\label{e5.8}
\ddot{y}_i = \omega_0^{\,2}\left(y_{i-1}-2\,y_i+y_{i+1}\right),
\end{equation}
where $\omega_0 = \sqrt{T/ m\,a}$. Since there is nothing special
about the $i$th bead, we deduce that the above equation of motion
applies to all $N$ beads: {\em i.e.}, it is valid for $i=1,N$. Of course,
the first ($i=1$) and last ($i=N$) beads are special cases, since there
is no bead corresponding to $i=0$ or $i=N+1$.  In fact, $i=0$ and $i=N+1$
correspond to the left and  right ends of the string, respectively. However,
Equation~(\ref{e5.8}) still applies to the first and last beads as long as
we set $y_0=0$ and $y_{N+1}=0$. What we are effectively demanding
 is that the two ends of the string, which are attached to the left and right walls, must both have {\em zero transverse
displacement}. 

Incidentally, we can prove that the tensions in the two sections of the string shown in Figure~\ref{f5.2} must be
equal by considering the {\em longitudinal}\/ equation of motion of the $i$th bead. This equation takes the
form
\begin{equation}
m\,\ddot{x}_i=-T_i\,\cos\theta_i +T_{i+1}\,\cos\theta_{i+1},
\end{equation}
where $T_i$ and $T_{i+1}$ are the, supposedly different, tensions in the sections of the string to the
immediate left and right of the $i$th bead, respectively. Now, we are assuming that the motion
of the beads is purely transverse: {\em i.e.}, all of the $x_i$ are constants. Thus, it follows from the
above equation that
\begin{equation}
T_i\,\cos\theta_i = T_{i+1}\,\cos\theta_{i+1}.
\end{equation}
However, if the transverse displacement of the string is sufficiently small that all of the $\theta_i$ are small,
the we can make use of the small angle approximation $\cos\theta_i\simeq 1$. Hence, we
obtain
\begin{equation}
T_i\simeq T_{i+1}.
\end{equation}
A simple extension of this argument reveals that the tension is the same in all sections of the string. 

Let us search for a normal mode solution to Equation~(\ref{e5.8}) which
takes the form
\begin{equation}\label{e5.9}
y_i(t) = A\,\sin(k\,x_i)\,\cos(\omega\,t-\phi),
\end{equation}
where $A>0$, $k>0$, $\omega>0$, and $\phi$ are constants. 
This particular type of solution is such that all of the  beads execute transverse simple harmonic oscillations {\em in phase}\/
with one another. See Figure~\ref{f5.4}. Moreover, the oscillations have an amplitude $A\,\sin(k\,x_i)$
which {\em varies sinusoidally}\/ along the length of the string ({\em i.e.}, in the $x$-direction). The  pattern of oscillations is thus {\em periodic in space}. The spatial repetition period, which is usually known as the {\em wavelength}, is $\lambda=2\pi/k$. [This follows from (\ref{e5.9}) because $\sin\theta$ is a periodic function with period
$2\pi$: {\em i.e.}, $\sin(\theta+2\pi)\equiv \sin\theta$.] The constant $k$, which determines the wavelength, is usually referred to as the {\em wavenumber}.  Thus, a
small wavenumber corresponds to a long wavelength, and {\em vice versa}. 
The type of solution specified in (\ref{e5.9}) is generally
known as a {\em standing wave}.  It is a {\em wave}\/ because it is periodic
in both space and time. (An oscillation is periodic in time only.) It is
a {\em standing}\/ wave, rather than a traveling wave, because the points
of maximum and minimum amplitude oscillation are stationary (in $x$). See Figure~\ref{f5.4}.

Substituting (\ref{e5.9}) into (\ref{e5.8}), we obtain
\begin{eqnarray}
-\omega^2\,A\,\sin(k\,x_i)\,\cos(\omega\,t-\phi)
&=&\omega_0^{\,2}\,A\left[\sin(k\,x_{i-1})-2\,\sin(k\,x_i)\right.\nonumber\\[0.5ex]
&&\left.+\sin(k\,x_{i+1})\right]\,\cos(\omega\,t-\phi),
\end{eqnarray}
which yields
\begin{equation}
-\omega^2\,\sin(k\,x_i)=\omega_0^{\,2}\left(\sin[k\,(x_i-a)] -2\,\sin(k\,x_i)
+\sin[k\,(x_i+a)]\right).
\end{equation}
However, since $\sin(a+b)\equiv \sin a\,\cos b + \cos a\,\sin b$, we get
\begin{eqnarray}
-\omega^2\,\sin(k\,x_i)=\omega_0^{\,2}\left[\cos(k\,a)-2+\cos(k\,a)\right]\sin(k\,x_i),
\end{eqnarray}
which gives
\begin{equation}\label{e5.13}
\omega = 2\,\omega_0\,\sin(k\,a/2),
\end{equation}
where use has been made of the trigonometric identity $1-\cos\theta \equiv 2\,\sin^2(\theta/2)$. Note that an expression, such as (\ref{e5.13}), which determines the angular frequency, $\omega$, of
a wave in terms of its wavenumber, $k$, is generally known as a {\em dispersion relation}.

Now, the solution (\ref{e5.9}) is only physical provided $y_0=y_{N+1}=0$: {\em i.e.},
provided that the two ends of the string remain stationary. The first constraint
is automatically satisfied, since $x_0=0$ [see (\ref{e5.1})]. The second constraint
implies that
\begin{equation}
\sin(k\,x_{N+1}) = \sin[(N+1)\,k\,a] = 0.
\end{equation}
This condition can only be  satisfied if
\begin{equation}\label{e5.15}
k = \frac{n}{N+1}\,\frac{\pi}{a},
\end{equation}
where the integer $n$ is  known as the {\em mode number}. Clearly, a small mode number translates to a small wavenumber, and, hence, to a
long wavelength, and {\em vice versa}. We conclude that the possible wavenumbers, $k$, of the normal modes of the system are
{\em quantized}\/ such that they are integer multiples of $\pi/[(N+1)\,a]$. 
Thus, the $n$th normal mode  is
associated with the characteristic pattern of bead displacements
\begin{equation}\label{e5.16}
y_{n,i}(t)=A_n\,\sin\left(\frac{n\,i}{N+1}\,\pi\right)\,\cos(\omega_n\,t-\phi_n),
\end{equation}
where
\begin{equation}\label{e5.17}
\omega_n = 2\,\omega_0\,\sin\left(\frac{n}{N+1}\,\frac{\pi}{2}\right).
\end{equation}
Here, the integer $i=1,N$ indexes the beads, whereas the mode number $n$ indexes the normal modes. Furthermore,
$A_n$ and $\phi_n$ are arbitrary constants determined by the initial conditions. 
Of course, $A_n$ is the {\em peak amplitude}\/ of the $n$th normal mode, whereas $\phi_n$
is the associated {\em phase angle}.

So, how many unique normal modes does the system possess? At first sight, it
might seem that there are an infinite number of normal modes, corresponding to the
infinite number of possible values that the integer $n$ can take. However, this is not
the case. For instance, if $n=0$ or $n=N+1$ then all of the $y_{n,i}$ are zero.
Clearly, these cases are not real normal modes. Moreover, it is easily
demonstrated that
\begin{eqnarray}
\omega_{-n} &=&-\omega_n,\\[0.5ex]
y_{-n,i}(t) &=& y_{n,i}(t),
\end{eqnarray}
provided $A_{-n}=-A_n$ and $\phi_{-n}=-\phi_n$, as well as
\begin{eqnarray}
\omega_{N+1+n} &=&\omega_{N+1-n},\\[0.5ex]
y_{N+1+n,i}(t) &=&y_{N+1-n,i}(t),
\end{eqnarray}
provided $A_{N+1+n}=-A_{N+1-n}$ and $\phi_{N+1+n}=\phi_{N+1-n}$. We,
thus, conclude that only those normal modes which have $n$ in the
range $1$ to $N$ correspond to unique  modes. Modes with $n$ values lying outside
this range are either null modes, or modes that are identical to other  modes
with  $n$ values lying within  the prescribed range. It follows  that
there are $N$ unique normal modes of the form (\ref{e5.16}). Hence, given that we are dealing with an $N$ degree of
freedom system, which we would expect to only possess $N$ unique normal modes, we can be sure that we have found {\em all}\/ of the normal modes of the system. 

\begin{figure}
\epsfysize=7in
\centerline{\epsffile{Chapter05/fig03.eps}}
\caption{\em Normal modes of a beaded string with eight equally spaced
beads.}\label{f5.3}   
\end{figure}

Figure~\ref{f5.3} illustrates the  spatial variation of the normal modes of a beaded string possessing eight
beads: {\em i.e.}, an $N=8$ system. The modes are all
shown at the instances in time at which they attain their maximum amplitudes: {\em i.e.},
at $\omega_n\,t-\phi_n=0$. It can be seen that the low-$n$---{\em i.e.}, long wavelength---modes cause the string to oscillate in a
fairly smoothly varying (in $x$) sine wave pattern. On the other hand, the high-$n$---{\em i.e.}, short
wavelength---modes cause the string to oscillate in a
rapidly varying zig-zag pattern which bears little resemblance to a sine wave.
The crucial distinction between the two different types of mode is that the
wavelength of the oscillation (in the $x$-direction) is much larger
than the bead spacing in the former case, whilst it is similar to
the bead spacing in the latter. For instance, $\lambda=18\,a$ for the $n=1$ mode,
$\lambda = 9\,a$ for the $n=2$ mode, but $\lambda  = 2.25\,a$ for the $n=8$ mode.

\begin{figure}
\epsfysize=3in
\centerline{\epsffile{Chapter05/fig04.eps}}
\caption{\em Time evolution of the $n=2$ normal mode of a beaded string with eight equally spaced
beads.}\label{f5.4}   
\end{figure}

Figure~\ref{f5.5}  displays the temporal variation of the $n=2$ normal mode of
an $N=8$  beaded string. The mode is shown at $\omega_2\,t-\phi_2=0$, $\pi/8$, $\pi/4$, $3\pi/8$, $\pi/2$, $5\pi/8$, $3\pi/2$, $7\pi/8$ and $\pi$. It can be seen
that the beads oscillate {\em in phase}\/ with one another: {\em i.e.}, they all attain their maximal
transverse displacements, and pass through zero displacement, {\em simultaneously}. 
Note that the mid-way point of the string always remains stationary. Such a point is
known as a {\em node}. The $n=1$ normal mode has two nodes (counting the
stationary points at each end of the string as nodes), the $n=2$
mode has three nodes, the $n=2$ mode four nodes, {\em etc.} In fact, the existence
of nodes is one of the distinguishing features of a standing wave.

\begin{figure}
\epsfysize=3in
\centerline{\epsffile{Chapter05/fig05.eps}}
\caption{\em Normal frequencies of a beaded
string with eight equally spaced beads.}\label{f5.5}   
\end{figure}

Figure~\ref{f5.5} shows the normal frequencies  of an $N=8$
beaded string  plotted  as a function of the normalized wavenumber. Recall that, for an $N=8$
system,
the relationship between the normalized wavenumber, $k\,a$, and the
mode number, $n$, is $k\,a =  (n/9)\,\pi$.
It can be seen that the angular frequency  increases as the  wavenumber
 increases, which implies that shorter wavelength  modes  have
higher oscillation frequencies. Note that the dependence of the angular frequency
on the normalized wave\-number, $k\,a$, is approximately {\em linear}\/ when $k\,a\ll 1$. Indeed, it can be seen from  Equation~(\ref{e5.17}) that if $k\,a\ll 1$ then the small
angle approximation $\sin\theta\simeq \theta$ yields a linear dispersion relation
of the form
\begin{equation}\label{e5.22}
\omega_n \simeq (k\,a)\,\omega_0 = \left(\frac{n}{N+1}\right)\pi\,\omega_0.
\end{equation}
We, thus, conclude that those normal modes of a uniformly beaded string whose wavelengths greatly
exceed the bead spacing ({\em i.e.}, modes with $k\,a\ll 1$) have approximately {\em linear dispersion
relations}\/ in which their angular frequencies are directly proportional
to their mode numbers. However, it is evident from the figure that this linear relationship
breaks down as $k\,a\rightarrow 1$, and the mode wavelength consequently becomes
similar to the bead spacing. 

\section{Normal Modes of a Uniform String}\label{s5.2}
Consider a uniformly beaded string in the limit in which the number of beads, $N$,
becomes increasingly large, whilst the spacing, $a$, and the individual mass, $m$, of the beads becomes increasingly small.  Let the
limit be  taken in such 
a manner that the length, $l=(N+1)\,a$, and the average mass per unit length, $\rho=m/a$, 
of the string both remain constant. Clearly, as $N$ increases, and becomes very large, such a string will more and
more closely approximate a {\em uniform string}\/ of length $l$ and
mass per unit length $\rho$. What can we guess about the normal modes of
a uniform string from the analysis contained in the previous section? Well, first of all,
we would guess that a uniform string is an {\em infinite}\/ degree of freedom system,
with an infinite number of unique normal modes of oscillation. This follows because a uniform string is the $N\rightarrow\infty$ limit of a beaded string, and
a beaded string possesses $N$ unique normal modes.   Next, we  would
guess that the normal modes of a uniform string exhibit {\em smooth sinusoidal spatial variation}\/ in the $x$-direction, and that the angular frequency of the modes is {\em directly proportional}\/ to
their wavenumber. These last two conclusions follow because all of the normal modes of a beaded string are characterized by $k\,a\ll 1$ in the limit in which the spacing between the beads becomes zero. Let us now investigate whether these guesses are correct.

Consider the transverse oscillations of a uniform string of length $l$ and mass
per unit length $\rho$ which is stretched between two immovable walls.  
It is again convenient to define a Cartesian coordinate system in which $x$
measures distance along the string from the left  wall, and $y$ measures
the transverse displacement of the string. Thus, the instantaneous state of the
system at time $t$ is determined by the function $y(x,t)$ for $0\leq x\leq l$. 
Of course, this function consists of an infinite number of different $y$ values,
corresponding to the infinite number of different $x$ values in the range $0$ to $l$.
Moreover, all of these $y$ values are free to vary {\em independently}\/ of one another.
It follows that we are indeed dealing with a dynamical system possessing an {\em infinite}\/ number
of degrees of freedom.

Let us try to reuse some of the analysis of the previous section. We can reinterpret
$y_i(t)$ as $y(x,t)$, $y_{i-1}(t)$ as $y(x-\delta x,t)$, and $y_{i+1}(t)$ as
$y(x+\delta x,t)$, assuming that $x_i=x$ and $a=\delta x$. Moreover, $\ddot{y}_i(t)$
becomes $\partial^2 y(x,t)/\partial t^2$: {\em i.e.}, a second derivative of
$y(x,t)$ with respect to $t$ at {\em constant}\/ $x$. Finally, $\omega_0^{\,2}= T/(m\,a)$, where
$T$ is the tension in the string, can be rewritten as $T/[\rho\,(\delta x)^2]$, since $\rho=m/\delta x$. Incidentally, we are again assuming that the transverse displacement of the
string remains sufficiently small that the tension is approximately constant in $x$. 
Thus,  the equation of motion of the beaded string, (\ref{e5.8}),  transforms into
\begin{equation}\label{e5.23}
\frac{\partial^2 y(x,t)}{\partial t^2} = \frac{T}{\rho}\left[\frac{y(x-\delta x,t)-2\,y(x,t)+y(x+\delta x,t)}{(\delta x)^2}\right].
\end{equation}
However, Taylor expanding $y(x+\delta x,t)$ in $x$ at constant $t$, we obtain
\begin{equation}
y(x+\delta x,t) = y(x,t) + \frac{\partial y(x,t)}{\partial x}\,\delta x + \frac{1}{2}\,\frac{\partial^2 y(x,t)}{\partial x^2}\,(\delta x)^2 + {\cal O}(\delta x)^3.
\end{equation}
Likewise,
 \begin{equation}
y(x-\delta x,t) = y(x,t) - \frac{\partial y(x,t)}{\partial x}\,\delta x + \frac{1}{2}\,\frac{\partial^2 y(x,t)}{\partial x^2}\,(\delta x)^2 + {\cal O}(\delta x)^3.
\end{equation}
It follows that
\begin{equation}
\left[\frac{y(x-\delta x,t)-2\,y(x,t)+y(x+\delta x,t)}{(\delta x)^2}\right]
=\frac{\partial^2 y(x,t)}{\partial x^2} + {\cal O}(\delta x).
\end{equation}
Thus, in the limit that $\delta x\rightarrow 0$, Equation~(\ref{e5.23}) reduces
to
\begin{equation}\label{e5.28}
\frac{\partial^2 y}{\partial t^2} = v^2\,\frac{\partial^2 y}{\partial x^2},
\end{equation}
where
\begin{equation}
v = \sqrt{\frac{T}{\rho}}
\end{equation}
is a quantity having the dimensions of {\em velocity}. Equation~(\ref{e5.28}), which is
the transverse equation of motion of the string, is an example of a very famous partial differential
equation known as the
{\em wave equation}. The quantity $v$ turns out to the the propagation
velocity of transverse waves along the string. See Section~\ref{s7.1}.

By analogy with Equation~(\ref{e5.9}), let us search for a solution of the wave
equation of the form
\begin{equation}\label{e5.29}
y(x,t) = A\,\sin(k\,x)\,\cos(\omega\,t-\phi),
\end{equation}
where $A>0$, $k>0$, $\omega>0$, and $\phi$ are constants. We would interpret such
a solution as a {\em standing wave}\/ of wavenumber $k$, wavelength $\lambda=2\pi/k$, angular frequency $\omega$,
peak amplitude $A$, and phase angle $\phi$. Substitution of the above
expression into Equation~(\ref{e5.28}) yields the {\em dispersion relation}\/
[{\em cf.}, (\ref{e5.13})]
\begin{equation}
\omega = k\,v.
\end{equation}

Now, the standing wave solution (\ref{e5.29}) is subject to the physical constraint that the two ends of the string, which are attached to immovable walls, 
remain {\em stationary}.  This leads directly to the {\em boundary conditions}
\begin{eqnarray}
y(0, t)&=& 0,\label{e5.31}\\[0.5ex]
y(l,t) &=&0.\label{e5.32}
\end{eqnarray}
It can be seen that the solution (\ref{e5.29}) automatically satisfies the first boundary
condition. However, the second boundary condition is only satisfied when $\sin(k\,l)=0$,
which immediately implies that
\begin{equation}
k = n\,\frac{\pi}{l},
\end{equation}
where the mode number, $n$, is an integer. We, thus, conclude that the
possible normal modes of a taut string, of length $l$ and
fixed ends,  have wavenumbers which are {\em quantized}\/ such that they
are integer multiples of $\pi/l$. Moreover, it is clear  that this quantization is a direct
consequence of the imposition of the physical boundary
conditions at the two ends of the string.

It follows, from the above analysis, that the $n$th normal mode of the string
is associated with the pattern of motion
\begin{equation}\label{e5.34}
y_n(x,t) = A_n\,\sin\left(n\,\pi\,\frac{x}{l}\right)\,\cos(\omega_n\,t-\phi_n),
\end{equation}
where
\begin{equation}\label{e5.35}
\omega_n = n\,\frac{\pi\,v}{l}.
\end{equation}
Here, $A_n$ and $\phi_n$ are constants which are determined
by the initial conditions. See Section~\ref{s5.3}. So, how many unique normal modes are there? Well,
the choice $n=0$ yields $y_0(x,t)=0$ for all $x$ and $t$, so this is not a real normal mode. Moreover,
\begin{eqnarray}
\omega_{-n}&=&-\omega_n,\\[0.5ex]
y_{-n}(x,t) &=& y_n(x,t),
\end{eqnarray}
provided that $A_{-n}= - A_n$ and $\phi_{-n}=-\phi_n$. We, thus,  conclude that
modes with negative mode numbers give rise to the same patterns of motion
as modes with corresponding positive mode numbers. However, the modes with
positive mode numbers each correspond to unique patterns of motion which
oscillate at unique frequencies. It follows that the string possesses an {\em infinite}\/ number
of normal modes, corresponding to the mode numbers $n=1,2,3,$ {\em etc.}
Recall that we are dealing with an infinite degree of freedom system, which
we would  expect to possess an infinite number of  unique normal modes. The fact that
we have actually found an infinite number of such modes suggests that we have found {\em all}\/
of the normal modes of the system. 

\begin{figure}
\epsfysize=7in
\centerline{\epsffile{Chapter05/fig06.eps}}
\caption{\em First eight normal modes of a uniform string.}\label{f5.6}   
\end{figure}

Figure~\ref{f5.6} illustrates the  spatial variation of the first eight normal modes of 
a uniform string with fixed ends. The modes are all
shown at the instances in time at which they attain their maximum amplitudes: {\em i.e.},
at $\omega_n\,t-\phi_n=0$. It can be seen that  the modes are all smoothly
varying sine waves. The low mode number ({\em i.e.}, long wavelength) modes are actually quite similar in form
to the corresponding normal modes of a uniformly beaded string. See Figure~\ref{f5.3}.
However, the high mode number modes are substantially different.
We conclude that the normal modes of a beaded string are similar to those of
a uniform string, with the same length and mass per unit length, provided that the wavelength of the mode is much larger
than the spacing between the beads. 

\begin{figure}
\epsfysize=3in
\centerline{\epsffile{Chapter05/fig07.eps}}
\caption{\em Time evolution of the $n=4$ normal mode of a uniform string.}\label{f5.7}   
\end{figure}

Figure~\ref{f5.7}  illustrates the temporal variation of the $n=4$ normal mode of
a uniform string. The mode is shown at $\omega_4\,t-\phi_4=0$, $\pi/8$, $\pi/4$, $3\pi/8$, $\pi/2$, $5\pi/8$, $3\pi/2$, $7\pi/8$ and $\pi$. It can be seen
that all points on the string attain their maximal
transverse displacements, and pass through zero displacement, {\em simultaneously}. 
Note that the $n=4$ mode possesses {\em five}\/ nodes, at which the string remains stationary. Two of these
are located at the ends of the string, and three  in the middle. In fact, it is
clear from Equation~(\ref{e5.34}) that the nodes correspond to points at which
$\sin [n\,(x/l)\,\pi]=0$. Hence, the nodes are located at
\begin{equation}
x_{n,j} = \left(\frac{j}{n}\right)l,
\end{equation}
where $j$ is an integer lying in the range $0$ to $n$. Here, $n$ indexes the normal mode,
and $j$ the node. Thus, the $j=0$ node lies at the left end of the string, the
$j=1$ node is the next node to the right, {\em etc.} It is apparent,  from the above
formula, that the $n$th normal mode has $n+1$ nodes which are {\em uniformly
spaced}\/ a distance $l/n$ apart.

\begin{figure}
\epsfysize=3in
\centerline{\epsffile{Chapter05/fig08.eps}}
\caption{\em Normal frequencies of the first eight normal modes of a uniform string.}\label{f5.8}   
\end{figure}

Finally, Figure~\ref{f5.8} shows the first eight normal frequencies  of a uniform string  with fixed ends, plotted  as a function of the mode number. 
It can be seen that the angular frequency  of oscillation increases {\em linearly}\/
with the mode number.  Recall that the low mode number ({\em i.e.}, long wavelength)
normal modes of a beaded string also exhibit a linear relationship between normal
frequency and mode number of the form [see Equation~(\ref{e5.22})]
\begin{equation}
\omega_n = \frac{n\,\pi}{N+1}\,\omega_0= \frac{n\,\pi}{N+1}\left(\frac{T}{m\,a}\right)^{1/2}.
\end{equation}
However, $m=\rho\,a$ and $l=(N+1)\,a$, so we obtain
\begin{equation}
\omega_n = \frac{n\,\pi}{l}\left(\frac{T}{\rho}\right)^{1/2} =n\, \frac{\pi\,v}{l},
\end{equation}
which is identical to Equation~(\ref{e5.35}). We, thus, conclude that the normal
frequencies of a uniformly beaded string are similar to those of a  uniform
string, with the same length and mass per unit length, as long as the wavelength of the associated normal mode is much larger than the spacing
between the beads.

\section{General Time Evolution of a Uniform String}\label{s5.3}
In the previous section, we found the normal modes of a uniform string of length $l$, both ends of
which are attached to immovable walls. These modes are spatially periodic solutions of the
wave equation (\ref{e5.28}) which oscillate at unique frequencies and
satisfy the boundary conditions (\ref{e5.31}) and (\ref{e5.32}). Since the wave equation is obviously {\em linear}\/ [{\em i.e.},
if $y(x,t)$ is a solution then so is $a\,y(x,t)$, where $a$ is an arbitrary constant], 
it follows that its most general solution is a {\em linear combination}\/ of all of the normal modes. 
In other words, 
\begin{equation}\label{e5.41}
y(x,t) = \sum_{n'=1,\infty}y_{n'}(x,t) = \sum_{n'=1,\infty} A_{n'}\,\sin\left(n'\,\pi\,\frac{x}{l}\right)\,\cos\left(n'\,\pi\,\frac{v\,t}{l}-\phi_{n'}\right),
\end{equation}
where use has been made of (\ref{e5.34}) and (\ref{e5.35}).
Note that this expression is obviously a solution of (\ref{e5.28}), and also
automatically satisfies the boundary conditions (\ref{e5.31}) and (\ref{e5.32}).
As we have already mentioned, the constants $A_n$ and $\phi_n$ are determined
by the {\em initial conditions}. Let us see how this comes about in more detail.

Suppose that the initial displacement of the string at $t=0$ is
\begin{equation}
y_0(x) \equiv y(x,0).
\end{equation}
Likewise, let the initial velocity of the string be
\begin{equation}
v_0(x) \equiv \frac{\partial y(x,0)}{\partial t}.
\end{equation}
Obviously, for consistency with the boundary conditions, we must have
$y_0(0)=y_0(l)=v_0(0)=v_0(l)=0$. 
It follows from Equation~(\ref{e5.41}) that
\begin{eqnarray}
y_0(x) &=& \sum_{n'=1,\infty} A_{n'}\,\cos\phi_{n'}\,\sin\left(n'\,\pi\,\frac{x}{l}\right)\label{e5.44},\\[0.5ex]
v_0(x) &=& \frac{v}{l}\sum_{n'=1,\infty} n'\,\pi\,A_{n'}\,\sin\phi_{n'}\,\sin\left(n'\,\pi\,\frac{x}{l}\right).\label{e5.45}
\end{eqnarray}

Now, it is readily demonstrated that
\begin{eqnarray}
\frac{2}{l}\int_0^l\sin\left(n\,\pi\,\frac{x}{l}\right)\,\sin\left(n'\,\pi\,\frac{x}{l}\right)dx&=&\frac{2}{\pi}\int_0^\pi \sin\left(n\,\theta\right)\,\sin\left(n'\,\theta\right)d\theta\nonumber\\[0.5ex]
&=&
\frac{1}{\pi}\int_0^\pi \cos\left[(n-n')\,\theta\right]d\theta-\frac{1}{\pi}\int_0^\pi\cos\left[(n+n')\,\theta\right] d\theta\nonumber\\[0.5ex]
&=&\frac{1}{\pi}\left[\frac{\sin\left[(n-n')\,\theta\right]}{n-n'}\right.-\left.
\frac{\sin\left[(n+n')\,\theta\right]}{n+n'}\right]_0^{\pi}\nonumber\\[0.5ex]
&=&\frac{\sin\left[(n-n')\,\pi\right]}{(n-n')\,\pi}-
\frac{\sin\left[(n+n')\,\pi\right]}{(n+n')\,\pi},\label{e5.46}
\end{eqnarray}
where $n$ and $n'$ are (possibly different) positive integers, $\theta=\pi\,x/l$,
and use has been made of the trigonometric identity $2\,\sin a\,\sin b \equiv \cos(a-b)-\cos(a+b)$.   Furthermore, it is easily seen that if $k$ is a {\em non-zero}\/ integer then
\begin{equation}
\frac{\sin(k\,\pi)}{k\,\pi} =0.
\end{equation}
On the other hand, $k=0$ is a special case, since both the
numerator and the denominator  in the above expression become zero simultaneously.  However, application of  {\em l`Hopital's rule}\/ yields
\begin{equation}
\lim_{x\rightarrow 0} \frac{\sin x}{x} = \lim_{x\rightarrow 0} \frac{d(\sin x)/dx}{dx/dx}
= \lim_{x\rightarrow 0} \frac{\cos x}{1} = 1.
\end{equation}
It follows that
\begin{equation}
\frac{\sin(k\,\pi)}{k\,\pi} = \left\{
\begin{array}{ccc}
1 &\mbox{\hspace{0.5cm}}&k=0\\[0.5ex]
0&&k\neq 0
\end{array}\right.,
\end{equation}
where $k$ is {\em any}\/ integer. This result can be combined with Equation~(\ref{e5.46}),
recalling that $n$ and $n'$ are both {\em positive}\/ integers, to give
\begin{equation}\label{e5.50}
\frac{2}{l}\int_0^l\sin\left(n\,\pi\,\frac{x}{l}\right)\,\sin\left(n'\,\pi\,\frac{x}{l}\right)dx
= \delta_{n,n'}.
\end{equation}
Here, the quantity
\begin{equation}\label{e5.51}
\delta_{n,n'} = \left\{
\begin{array}{ccc}
1 &\mbox{\hspace{0.5cm}}&n=n'\\[0.5ex]
0&&n\neq n'
\end{array}\right.,
\end{equation}
where $n$ and $n'$ are integers, is known as the {\em Kronecker delta function}.

Let us multiply Equation~(\ref{e5.44}) by $(2/l)\,\sin (n\,\pi\,x/l)$ and integrate over $x$
from $0$ to $l$. We obtain
\begin{eqnarray}
\frac{2}{l}\int_0^l y_0(x)\,\sin\left(n\,\pi\,\frac{x}{l}\right)dx
&=&\sum_{n'=1,\infty} \!\!\!A_{n'}\,\cos\phi_{n'}\,\frac{2}{l}\int_0^\pi\sin\left(n'\,\pi\,\frac{x}{l}\right)\,\sin\left(n\,\pi\,\frac{x}{l}\right)dx\nonumber\\[0.5ex]
&=&\sum_{n'=1,\infty} A_{n'}\,\cos\phi_{n'}\,\delta_{n,n'}
= A_n\cos\phi_n,
\end{eqnarray}
where use has been made of Equations~(\ref{e5.50}) and (\ref{e5.51}). 
Similarly, Equation (\ref{e5.45}) yields
\begin{equation}
\frac{2}{v}\int_0^l v_0(x)\,\sin\left(n\,\pi\,\frac{x}{l}\right)dx = n\,\pi \,A_n\,\sin\phi_n.
\end{equation}
Thus, defining the integrals
\begin{eqnarray}\label{e5.54}
C_n &=& \frac{2}{l}\int_0^l y_0(x)\,\sin\left(n\,\pi\,\frac{x}{l}\right)dx,\\[0.5ex]
S_n &=& \frac{2}{n\,\pi\,v}\int_0^l v_0(x)\,\sin\left(n\,\pi\,\frac{x}{l}\right)dx,\label{e5.55}
\end{eqnarray}
for $n=1,\infty$, we obtain
\begin{eqnarray}
C_n &=& A_n\,\cos\phi_n,\\[0.5ex]
S_n &=& A_n\,\sin\phi_n,
\end{eqnarray}
and, hence, 
\begin{eqnarray}\label{e5.58}
A_n &=& (C_n^{\,2} + S_n^{\,2})^{1/2},\\[0.5ex]
\phi_n &=& \tan^{-1}(S_n/C_n).\label{e5.59}
\end{eqnarray}
Thus, the constants $A_n$ and $\phi_n$, appearing in the general expression (\ref{e5.41}) for the time evolution of a uniform string with fixed ends, are
ultimately
determined by integrals over the string's initial displacement and velocity which are of the form
(\ref{e5.54}) and (\ref{e5.55}).

As an example, suppose that the string is initially {\em at rest}, so that 
\begin{equation}
v_0(x) = 0,\label{e5.60}
\end{equation}
 but
has the initial displacement
\begin{equation}
y_0(x) = 2\,A\left\{
\begin{array}{ccc}
x/l&\mbox{\hspace{0.5cm}}&0\leq x < l/2\\[0.5ex]
1-x/l &&l/2\leq x\leq l
\end{array}
\right. .\label{e5.61}
\end{equation}
This {\em triangular}\/  pattern is zero at both ends of the string,  rising {\em linearly}\/ to a
peak value of $A$ halfway along the string, and is designed to
mimic the initial displacement of a guitar string which is plucked at its
mid-point. See Figure~\ref{f5.10}.
A comparison of Equations~(\ref{e5.55}) and (\ref{e5.60}) reveals that, in this particular example, all of the $S_n$ coefficients
are zero. Hence, from (\ref{e5.58}) and (\ref{e5.59}), $A_n = C_n$ and $\phi_n=0$
for all $n$. Thus, making use of Equations~(\ref{e5.41}),  (\ref{e5.54}), and  (\ref{e5.61}), the
time evolution of the string is governed by
\begin{equation}\label{e5.62}
y(x,t) = \sum_{n=1,\infty}A_n\,\sin\left(n\,\pi\,\frac{x}{l}\right)\,\cos\left(n\,2\pi\,\frac{t}{\tau}\right),
\end{equation}
where $\tau= 2\,l/v$ is the oscillation period of the $n=1$ normal mode, and
\begin{equation}
A_n = \frac{2}{l}\int_0^{l/2}2\,A\,\frac{x}{l}\,\sin\left(n\,\pi\,\frac{x}{l}\right)\,dx
+\frac{2}{l}\int_{l/2}^{l}2\,A\left(1-\frac{x}{l}\right)\sin\left(n\,\pi\,\frac{x}{l}\right)\,dx.
\end{equation}
The above expression transforms to
\begin{equation}
A_n = A\left(\frac{2}{\pi}\right)^2\left\{
\int_0^{\pi/2}\theta\,\sin(n\,\theta)\,d\theta + \int_{\pi/2}^\pi(\pi-\theta)\,\sin(n\,\theta)\,d\theta\right\},
\end{equation}
where $\theta=\pi\,x/l$. Integration by parts yields
\begin{equation}\label{e5.65}
A_n = 2\,A\,\frac{\sin(n\,\pi/2)}{(n\,\pi/2)^2}.
\end{equation}
Note that $A_n=0$ whenever $n$ is even. We conclude that the triangular initial displacement pattern (\ref{e5.61})  only excites normal modes with {\em odd}\/
mode numbers. 

Now, when a stringed
instrument, such as a guitar, is  sounded a characteristic pattern of normal
mode oscillations is excited on the plucked string. These oscillations then generate 
sound waves of the same frequency, which are audible to a listener. 
The  normal mode (of appreciable amplitude) with the {\em lowest oscillation
frequency}\/ is called the {\em fundamental harmonic}, and  determines the
{\em pitch}\/ of the musical note which is heard by the listener. For instance, a fundamental
harmonic which oscillates at $261.6$ Hz would correspond to ``middle C''. Those normal
modes (of appreciable amplitude) with higher oscillation frequencies than the
fundamental harmonic are called {\em overtone harmonics}, since their
 frequencies are {\em integer multiples}\/ of the fundamental frequency.  In general, the
amplitudes of the overtone harmonics are much smaller than the amplitude of the fundamental. Nevertheless, when a stringed instrument is sounded, the particular mix of overtone harmonics which accompanies the
fundamental  determines
the {\em timbre}\/ of the musical note heard by the listener. For instance, when middle C
is played on a piano and a harpsichord the same frequency fundamental harmonic is excited 
in both cases. However, the mix of excited overtone harmonics  is quite different. This
accounts for the fact that middle C played on a piano is easily distinguishable from middle C
played on a harpsichord. 


\begin{figure}
\epsfysize=3in
\centerline{\epsffile{Chapter05/fig09.eps}}
\caption{\em Relative amplitudes of the overtone harmonics of a uniform guitar string
plucked at its mid-point.}\label{f5.9}   
\end{figure}

Figure~\ref{f5.9} shows the ratio $A_n/A_1$ for the first ten overtone harmonics
of a uniform guitar string plucked at its midpoint: {\em i.e.}, the ratio $A_n/A_1$
for odd-$n$ modes with $n>1$, calculated from Equation~(\ref{e5.65}).  It can be seen that the amplitudes
of the overtone harmonics are all small compared to the amplitude of the
fundamental. Moreover, the amplitudes  decrease rapidly in magnitude with increasing
mode number, $n$.

\begin{figure}
\epsfysize=3in
\centerline{\epsffile{Chapter05/fig10.eps}}
\caption{\em Reconstruction of the initial displacement of a uniform guitar
string plucked at its mid-point.}\label{f5.10}   
\end{figure}

In principle, we must include all of the normal modes in the sum on the right-hand side of Equation~(\ref{e5.62}). In practice, given that the amplitudes of the normal
modes decrease rapidly in magnitude as $n$ increases, we can {\em truncate}\/ the 
sum, by neglecting high-$n$ normal modes, without introducing significant error
into our calculation. Figure~\ref{f5.11} illustrates the effect of such a truncation. 
In fact, this figure shows the reconstruction of $y_0(x)$, obtained by setting $t=0$
in Equation~(\ref{e5.62}), made with various different numbers of normal modes.
The long-dashed line shows a reconstruction made with only the largest amplitude normal mode, the
short dashed-line shows a reconstruction made with the four largest amplitude
normal modes, and the solid line shows a reconstruction made with the
sixteen largest amplitude normal modes. It can be seen that sixteen normal
modes is sufficient to very accurately reconstruct the triangular
initial displacement pattern. Indeed, a reconstruction made with only four
normal modes is surprisingly accurate. On the other hand, a reconstruction made
with only one normal mode is fairly inaccurate. 

\begin{figure}
\epsfysize=3in
\centerline{\epsffile{Chapter05/fig11.eps}}
\caption{\em Time evolution of a uniform guitar
string plucked at its mid-point.}\label{f5.11}   
\end{figure}

Figure~\ref{f5.11} shows the time evolution of a uniform guitar
string plucked at its mid-point. This evolution is reconstructed from expression (\ref{e5.62}) using the sixteen
largest amplitude normal modes of the string. The upper solid, upper short-dashed, upper long-dashed, upper dot-short-dashed,
dot-long-dashed, lower dot-short-dashed, lower long-dashed, lower short-dashed,
and lower solid curves correspond to $t/\tau = 0$, $1/16$, $1/8$, $3/16$, $1/4$, $5/16$, 
$3/8$, $7/16$, and $1/2$, respectively. It can be seen that the string oscillates in
a rather strange fashion. The initial kink in the string at $x=l/2$ splits into two
equal kinks which propagate in opposite directions along the string at the velocity $v$.
The string remains straight and parallel to the $x$-axis between the kinks, and straight and inclined to the $x$-axis between each kink and the closest wall. When the two
kinks reach the wall the string is instantaneously found in its undisturbed position. The
kinks then reflect off the two walls, with a phase change of $\pi$ radians. When the
two kinks meet again at $x=l/2$ the string is instantaneously found in a state which is an inverted form of its
initial state. The kinks subsequently pass through one another, reflect off the walls, with another phase change of $\pi$ radians, and meet for a second time at $x=l/2$. At this
instant, the string is again found in its initial position. The pattern of motion then repeats itself {\em ad infinitum}. The period of the oscillation is the time required for a kink to propagate
two string lengths, which is $\tau = 2\,l/v$. This, of course, is also the oscillation period of the
$n=1$ normal mode.

\section{Exercises}
{\small
\begin{enumerate}
\item Consider a uniformly beaded string with $N$ beads which is similar
to that pictured in Figure~\ref{f5.1}, except that each end of the string
is attached to a massless ring which slides (in the $y$-direction)
 on a frictionless rod.
Demonstrate that the normal modes of the system take the form
$$
y_{n,i}(t) = A_n\,\cos\left[\frac{n\,(i-1/2)}{N}\,\pi\right]\,\cos(\omega_n\,t-\phi_n),
$$
where
$$
\omega_n = 2\,\omega_0\,\sin\left(\frac{n}{N}\,\frac{\pi}{2}\right),
$$
$\omega_0$ is as defined in Section~\ref{s5.1},  $A_n$ and $\phi_n$ are constants, the integer $i=1,N$ indexes the beads, and the mode number $n$ indexes the
modes.  How
many unique normal modes does the system possess, and what are their mode numbers?
Show that the lowest frequency mode has an infinite wavelength and zero frequency. 
Explain this peculiar result. Plot the normal modes and normal frequencies
of an $N=8$ beaded string in a similar fashion to Figures~\ref{f5.3} and \ref{f5.5}.

\item Consider a uniformly beaded string with $N$ beads which is similar
to that pictured in Figure~\ref{f5.1}, except that the left end of the string is
fixed, and the right  end  is attached to a massless ring which slides (in the $y$-direction)
on a frictionless rod.  Find the normal
modes and normal frequencies of the system. Plot the normal modes and normal frequencies
of an $N=8$ beaded string in a similar fashion to Figures~\ref{f5.3} and \ref{f5.5}.

\begin{figure}[h]
\epsfysize=0.9in
\centerline{\epsffile{Chapter05/fig12.eps}}
\end{figure}
\item The above figure  shows the left and right extremities of a linear $LC$ network consisting of $N$ identical inductors of inductance $L$, and $N+1$
identical capacitors of capacitance $C$. Let the instantaneous current
flowing through the $i$th inductor be $I_i(t)$, for $i=1,N$. Demonstrate from
Kirchhoff's circuital laws that the currents evolve in time according to the
coupled equations
$$
\ddot{I}_i = \omega_0^{\,2}\,(I_{i-1}-2\,I_i+I_{i+1}),
$$
for $i=1,N$, where $\omega_0=1/\sqrt{L\,C}$, and $I_0=I_{N+1}=0$. 
Find the normal frequencies of the system.

\item Suppose that the outermost two capacitors in the circuit considered in the previous exercise  are short-circuited. Find the new normal frequencies of the system.

\item A uniform string of length $l$, tension $T$, and mass per unit length
$\rho$, is stretched between two immovable walls. Suppose that the string is
initially in its equilibrium state. At $t=0$ it is
struck by a hammer in such a manner as to impart an impulsive
velocity $u_0$ to a small segment of length $a<l$ centered on the mid-point.
Find an expression for the subsequent motion of the string. Plot the motion
as a function of time  in a similar fashion to Figure~\ref{f5.11}, assuming that
$a=l/10$. 

\item A uniform string of length $l$, tension $T$, and mass per unit length
$\rho$, is stretched between two massless rings, attached to its ends, which
slide (in the $y$-direction) along frictionless rods.  Demonstrate that, in this case,
the most general solution to the wave equation takes the form
$$
y(x,t) = Y_0 + V_0\,t + \sum_{n>0} A_n\,\cos\left(n\,\pi\,\frac{x}{l}\right)\cos\left(n\,\pi\,\frac{v\,t}{l}-\phi_n\right),
$$
where $v=\sqrt{T/\rho}$, and $Y_0$, $V_0$, $A_n$, and $\phi_n$ are arbitrary constants. 
Show that
$$
\frac{2}{l}\int_0^l\cos\left(n\,\pi\,\frac{x}{l}\right)\cos\left(n'\,\pi\frac{x}{l}\right)dx=\delta_{n,n'},
$$
where $n$ and $n'$ are integers. Use this result to demonstrate that the arbitrary
constants in the above solution can be determined from the initial conditions 
as follows:
\begin{eqnarray}
Y_0 &=&\frac{2}{l}\int_0^l y_0(x)\,dx,\nonumber\\[0.5ex]
V_0 &=&\frac{2}{l}\int_0^l v_0(x)\,dx,\nonumber\\[0.5ex]
A_n &=& (C_n^{\,2}+S_n^{\,2})^{1/2},\nonumber\\[0.5ex]
\phi_n &=&\tan^{-1}(S_n/C_n),\nonumber
\end{eqnarray}
where $y_0(x)\equiv y(x,0)$, $v_0(x)\equiv \partial y(x,0)/\partial t$, and
\begin{eqnarray}
C_n&=& \frac{2}{l}\int_0^l y_0(x)\,\cos\left(n\,\pi\,\frac{x}{l}\right) dx,\nonumber\\[0.5ex]
S_n &=&  \frac{2}{l}\int_0^l v_0(x)\,\cos\left(n\,\pi\,\frac{x}{l}\right) dx.\nonumber
\end{eqnarray}
Suppose that the string is
initially in its equilibrium state. At $t=0$ it is
struck by a hammer in such a manner as to impart an impulsive
velocity $u_0$ to a small segment of length $a<l$ centered on the mid-point.
Find an expression for the subsequent motion of the string. Plot the motion
as a function of time  in a similar fashion to Figure~\ref{f5.11}, assuming that
$a=l/10$. 

\item The linear $LC$ circuit considered in Exercise 3 can be thought of as a discrete 
model of a uniform lossless transmission line: {\em e.g.}, a co-axial cable. In this interpretation,
$I_i(t)$ represents $I(x_i,t)$, where $x_i=i\,\delta x$. Moreover, $C={\cal C}\,\delta x$, and
$L={\cal L}\,\delta x$, where ${\cal C}$ and ${\cal L}$ are the capacitance per unit length and the
inductance per unit length of the line, respectively. Show that, in the limit $\delta x\rightarrow 0$,
the evolution equation for the coupled currents given in Exercise 3 reduces to
the wave equation
$$
\frac{\partial^2 I}{\partial t^2} = v^2\,\frac{\partial^2 I}{\partial x^2},
$$
where $I=I(x,t)$, $x$ measures distance along the line, and $v=1/\sqrt{{\cal L}\,{\cal C}}$. 
If $V_i(t)$ is the potential difference (measured from the top to the bottom) across the $i+1$th capacitor (from the left) in the
circuit shown in Exercise 3, and $V(x,t)$ is the corresponding voltage in the transmission line, show that the discrete circuit equations relating the $I_i(t)$ and $V_i(t)$
reduce to 
\begin{eqnarray}
\frac{\partial V}{\partial t} &=&-\frac{1}{{\cal C}}\,\frac{\partial I}{\partial x},\nonumber\\[0.5ex]
\frac{\partial I}{\partial t} &=& - \frac{1}{{\cal L}}\,\frac{\partial V}{\partial x},\nonumber
\end{eqnarray}
in the transmission line limit. Hence, demonstrate that the voltage in a transmission
line satisfies the wave equation
$$
\frac{\partial^2 V}{\partial t^2} = v^2\,\frac{\partial^2 V}{\partial x^2}.
$$

\item Consider a uniform string of length $l$, tension $T$, and mass per unit length $\rho$ which is stretched between two immovable walls. Show that the total energy
of the string, which is the sum of its kinetic and potential energies, is
$$
E = \frac{1}{2}\int_0^l\left[\rho\left(\frac{\partial y}{\partial t}\right)^2
+ T\left(\frac{\partial y}{\partial x}\right)^2\right] dx,
$$
where $y(x,t)$ is the string's (relatively small) transverse displacement. Now, the general motion of
the string can be represented as a linear superposition of the normal modes:
$$
y(x,t) = \sum_{n=1,\infty}A_n\,\sin\left(n\,\pi\,\frac{x}{l}\right)\cos\left(n\,\pi\,\frac{v\,t}{l}-\phi_n\right),
$$
where $v=\sqrt{T/\rho}$. Demonstrate that
$$
E = \sum_{n=1,\infty} E_n,
$$
where
$$
E_n = \frac{1}{4}\,m\,\omega_n^{\,2}\,A_n^{\,2}
$$
is the energy of the $n$th normal mode. Here, $m=\rho\,l$ is the mass of the
string, and $\omega_n = n\,\pi\,v/l$ the angular frequency of the $n$th normal
mode.

\end{enumerate}}
