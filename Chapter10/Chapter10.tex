\chapter{Multi-Dimensional Waves}
\section{Plane Waves}\label{s10.1}
As we have already seen, a sinusoidal wave of amplitude $\psi_0>0$, wavenumber $k>0$, and  angular frequency $\omega>0$, 
propagating in the positive $x$-direction,  can be  represented by a {\em wavefunction}\/ of the form
\begin{equation}\label{e10.1}
\psi(x,t)=\psi_0\,\cos(k\,x-\omega\,t).
\end{equation}
Now, the  above type of wave is conventionally termed  a {\em one-dimen\-sio\-nal plane wave}. It is {\em one-dimensional}\/
because its associated wavefunction only depends on a single Cartesian coordinate. 
Furthermore, it is a {\em plane wave}\/ because the wave maxima, which are located at
\begin{equation}\label{e10.2}
k\,x-\omega\,t  = j\,2\pi,
\end{equation}
where $j$ is an integer, consist of a series of {\em parallel planes},  normal to the $x$-axis, which are equally spaced a distance
$\lambda=2\pi/k$ apart, and propagate along the
$x$-axis at the fixed speed $v=\omega/k$. 
These conclusions follow because Equation (\ref{e10.2}) can be re-written in the form
\begin{equation}\label{e10.3}
x= d,
\end{equation}
where $d=j\,\lambda + v\,t$. Moreover, (\ref{e10.3})
is clearly the equation of a plane, normal to the $x$-axis,  whose distance of closest approach to the
origin is $d$. 

\begin{figure}
\epsfysize=2.5in
\centerline{\epsffile{Chapter10/fig01.eps}}
\caption{\em The solution of ${\bf n}\cdot{\bf r} = d$ is a plane.}\label{f10.1}   
\end{figure}

The previous equation can also be written in the coordinate-free form
\begin{equation}\label{e10.4}
 {\bf n}\cdot{\bf r} = d,
\end{equation}
where  ${\bf n} = (1,\,0,\,0)$ is a unit
vector directed along the $x$-axis, and ${\bf r}=(x,\,y,\,z)$ represents the vector displacement of a general point from the origin. Since there is nothing special about the $x$-direction, it follows that if ${\bf n}$ is re-interpreted as a 
unit vector pointing in an {\em arbitrary}\/ direction then (\ref{e10.4}) can be re-interpreted as the general equation of a plane.
As before, the plane is normal to
${\bf n}$, and its distance of closest approach to the origin is $d$. See Figure~\ref{f10.1}. This observation allows us to write the three-dimensional
equivalent to the wavefunction (\ref{e10.1}) as
\begin{equation}\label{e10.5}
\psi(x,y,z,t)=\psi_0\,\cos({\bf k}\cdot{\bf r}-\omega\,t),
\end{equation}
where the constant vector ${\bf k} = (k_x,\,k_y,\,k_z)=k\,{\bf n}$ is called the {\em wavevector}. The  wave represented above is conventionally termed 
a {\em three-dimensional plane wave}. It is three-dimensio\-nal because its  wavefunction, $\psi(x,y,z,t)$, depends on all
three Cartesian coordinates. Moreover, it is a plane wave because the wave maxima are located at
\begin{equation}
{\bf k}\cdot{\bf r} -\omega\,t = j\,2\pi,
\end{equation}
or
\begin{equation}
{\bf n}\cdot{\bf r} = j\,\lambda + v\,t,
\end{equation}
where  $\lambda=2\pi/k$,   and $v=\omega/k$. Note that the wavenumber, $k$, is the
{\em magnitude}\/ of the wavevector, ${\bf k}$: {\em i.e.}, $k\equiv |{\bf k}|$. 
It follows, by comparison with Equation~(\ref{e10.4}), that the
wave maxima consist of a series of parallel planes,  normal to the wavevector, which are equally spaced a distance $\lambda$ apart, and propagate in the ${\bf k}$-direction  at the fixed speed $v$. See Figure~\ref{f10.2}. Hence, the direction of the wavevector specifies the  wave propagation direction, whereas its magnitude  determines the wavenumber, $k$, and, thus, the wavelength, $\lambda=2\pi/k$. 
Actually, the most general expression for the wavefunction of a  plane wave is $\psi = \psi_0\,\cos(\phi+{\bf k}\cdot{\bf r}-\omega\,t)$, where
$\phi$ is a constant {\em phase angle}. As is easily appreciated, the inclusion  of a non-zero phase angle in the wavefunction
merely shifts all the wave maxima a distance $-(\phi/2\pi)\,\lambda$  in the ${\bf k}$-direction. In the following, whenever possible, $\phi$ is
set to zero, for the sake of simplicity. 

\begin{figure}
\epsfysize=2.5in
\centerline{\epsffile{Chapter10/fig02.eps}}
\caption{\em Wave maxima associated with a plane wave.}\label{f10.2}   
\end{figure}

\section{Three-Dimensional Wave Equation}\label{s10.2}
As is readily demonstrated (see Exercise 1), the one-dimensional plane wave solution (\ref{e10.1}) satisfies the {\em one-dimen\-sional wave equation},
\begin{equation}\label{e10.8}
\frac{\partial^2\psi}{\partial t^2} = v^2\,\frac{\partial^2\psi}{\partial x^2}.
\end{equation}
Likewise, the three-dimensional plane wave solution (\ref{e10.5}) satisfies the {\em three-dimensional
wave equation}\/  (see Exercise 1),
\begin{equation}\label{e10.9}
\frac{\partial^2\psi}{\partial t^2} = v^2\left(\frac{\partial^2}{\partial x^2}+\frac{\partial^2}{\partial y^2}+\frac{\partial^2}{\partial z^2}\right)\psi.
\end{equation}
Note that both of these equations are {\em linear}, and, thus, have {\em superposable solutions}. 

\section{Laws of Geometric Optics}\label{sgeo}
Suppose that the region $z<0$ is occupied by a transparent dielectric medium of refractive index $n_1$, whereas
the region $z>0$ is occupied by a second transparent dielectric medium of refractive index $n_2$. Let a plane light wave be launched, toward positive $z$,  from a
light source of angular frequency $\omega$ located at  large negative $z$. Suppose that  this so-called {\em incident}\/ wave has a
wavevector ${\bf k}_i$. Now, we would expect the incident wave to be {\em partially reflected}\/ and {\em partially
transmitted}\/ at the interface between the two dielectric media. 
Let the reflected and transmitted waves have the wavevectors 
${\bf k}_r$ and ${\bf k}_t$, respectively. See Figure~\ref{f10.3}.  Hence, we can write
\begin{equation}
\psi(x,y,z,t)= \psi_i \,\cos({\bf k}_i\cdot{\bf r}-\omega\,t) + \psi_r\,\cos({\bf k}_r\cdot{\bf r}-\omega\,t)
\end{equation}
in the region $z<0$, and
\begin{equation}
\psi(x,y,z,t)= \psi_t\,\cos({\bf k}_t\cdot{\bf r}-\omega\,t)
\end{equation}
in the region $z>0$. Here, $\psi(x,y,z,t)$ represents the {\em magnetic}\/ component of the resultant light wave, $\psi_i$ the
amplitude of the incident wave, $\psi_r$ the amplitude of the reflected wave, and $\psi_t$ the amplitude of the
transmitted wave. Note that all of the component waves have the same angular frequency, $\omega$, since this property is
ultimately determined by the wave source. Note, further, that, according to standard electromagnetic
theory, if the magnetic component of an electromagnetic wave is specified then the electric
component of the wave is fully determined, and can easily be calculated, and {\em vice versa}. 

In general, the wavefunction, $\psi$, must be {\em continuous}\/ at $z=0$, since, according to standard electromagnetic theory,  there cannot be a discontinuity in either the
normal or the tangential component of a magnetic field across an interface between two (non-magnetic) dielectric media. (Incidentally,
the same is not true of an electric field, which can have a normal discontinuity across
an interface between two dielectric media. This explains why we have chosen  $\psi$ to represent the magnetic, rather than the electric,
component of the resultant light wave.) Thus, the matching condition at $z=0$ takes the form
\begin{eqnarray}
\psi_i\,\cos(k_{i\,x}\,x+k_{i\,y}\,y-\omega\,t)&&\nonumber\\[0.5ex] + \psi_r\,\cos(k_{r\,x}\,x+k_{r\,y}\,y-\omega\,t)&=&\psi_t\,\cos(k_{t\,x}\,x+k_{t\,y}\,y-\omega\,t).
\end{eqnarray}
Moreover, this condition must be satisfied at {\em all}\/ values of $x$, $y$, and $t$. Clearly, this is only possible if
\begin{equation}\label{e10.13}
k_{i\,x} = k_{r\,x} = k_{t\,x},
\end{equation}
and
\begin{equation}\label{e10.14}
k_{i\,y} = k_{r\,y} = k_{t\,y}.
\end{equation}

Suppose that the direction of propagation of the incident wave lies in the $x$-$z$ plane, so that $k_{i\,y}=0$. It immediately
follows, from (\ref{e10.14}), that $k_{r\,y}=k_{t\,y}=0$. In other words, the directions of propagation of the reflected
and the transmitted waves also lie in the $x$-$z$ plane.  This means that ${\bf k}_i$, ${\bf k}_r$ and ${\bf k}_t$ are 
{\em co-planar}\/ vectors. Note that  this constraint is implicit in the well-known laws of
geometric optics. 

\begin{figure}
\epsfysize=3in
\centerline{\epsffile{Chapter10/fig03.eps}}
\caption{\em Reflection and refraction of a plane wave at a plane boundary.}\label{f10.3}   
\end{figure}

Assuming that the above mentioned constraint is satisfied, let the incident, reflected, and transmitted waves subtend 
angles $\theta_i$, $\theta_r$, and $\theta_t$ with the $z$-axis, respectively. See Figure~\ref{f10.3}. It follows that
\begin{eqnarray}
{\bf k}_i &=& n_1\,k_0\,(\sin\theta_i,\,0,\,\cos\theta_i),\\[0.5ex]
{\bf k}_r &=&n_1\, k_0\,(\sin\theta_r,\,0,-\cos\theta_r),\\[0.5ex]
{\bf k}_t &=&n_2\,k_0\,(\sin\theta_t,\,0,\,\cos\theta_t),
\end{eqnarray}
where $k_0=\omega/c$ is the vacuum wavenumber, and $c$ the velocity of light in vacuum.
Here, we have made use of the fact that wavenumber ({\em i.e.}, the magnitude of the wavevector) of a light wave propagating through a dielectric medium of
refractive index $n$ is $n\,k_0$.

Now, according to Equation~(\ref{e10.13}), $k_{i\,x}= k_{r\,x}$, which yields
\begin{equation}
\sin\theta_i=\sin\theta_r,
\end{equation}
and $k_{i\,x}=k_{t\,x}$, which reduces to
\begin{equation}
n_1\,\sin\theta_i = n_2\,\sin\theta_t.
\end{equation}
The first of these relations states that the angle of incidence, $\theta_i$, is equal to the angle of reflection, $\theta_r$.  Of course, this is the
familiar {\em law of reflection}. Moreover, the second relation corresponds to the equally familiar
{\em law of refraction}, otherwise known as {\em Snell's law}. 

Incidentally, the fact that a plane wave propagates through a uniform
medium with a {\em constant}\/ wavevector, and, therefore, a {\em constant}\/ direction of propagation, is equivalent to the well known {\em law of rectilinear propagation}, which
states that light propagates through a uniform medium in a {\em straight-line}. 

It is clear, from the above discussion, that  the  laws of geometric optics ({\em i.e.}, the law
of rectilinear propagation, the law of reflection, and the law of refraction) are  fully consistent with the 
wave properties of light, despite the fact that they do not  appear to explicitly depend on these properties. 

\section{Waveguides}
As we saw in Section~\ref{s7.5}, transmission lines ({\em e.g.}, ethernet cables)  are used to carry high frequency electromagnetic signals over distances which are long compared to the signal wavelength, $\lambda = c/f$, where $c$ is the velocity of light and $f$ the signal frequency (in Hertz). 
Unfortunately, conventional transmission lines are subject to {\em radiative losses}\/ (since the lines effectively act as antennas) 
which increase as the {\em fourth power}\/ of the signal frequency. Above a certain critical  frequency, which typically
lies in the microwave band, the radiative losses become intolerably large. Under these circumstances, the transmission line must be
replaced by a device known as a {\em waveguide}. A {\em waveguide}\/ is basically a long  hollow metal box  within
which electromagnetic signals propagate. Moreover, if the walls of the box are much thicker than the skin-depth (see Section~\ref{s9.3}) in the wall material then
the signal is essentially isolated from the outside world, and the radiative losses are consequently negligible. 

Consider an evacuated  waveguide of rectangular cross-section which runs along the $z$-direction, and is enclosed by perfectly conducting 
({\em i.e.}, infinite conductivity) metal walls located at $x=0$, $x=a$, $y=0$, and $y=b$. Suppose that an electromagnetic
wave propagates along the waveguide  in the $z$-direction. For the sake of simplicity, let there be no $y$-variation of the wave electric or magnetic fields.
 Now, the wave propagation inside the waveguide is governed by the two-dimensional wave equation [{\em cf.}, Equation~(\ref{e10.9})]
\begin{equation}\label{e10.20}
\frac{\partial^2\psi}{\partial t^2} = c^2\left(\frac{\partial^2}{\partial x^2}+ \frac{\partial^2}{\partial z^2}\right)\psi,
\end{equation}
where $\psi(x,z,t)$ represents the {\em electric}\/ component of the wave, which is assumed to be everywhere parallel to the $y$-axis, 
and $c$ is the velocity of light in vaccum. The appropriate boundary conditions are 
\begin{eqnarray}\label{e10.21}
\psi(0,z,t) &=&0,\\[0.5ex]
\psi(a,z,t) &=&0,\label{e10.22}
\end{eqnarray}
since the electric field inside a perfect conductor is zero (otherwise, an infinite current would flow), and, according to standard electromagnetic
theory,  there cannot be a  tangential discontinuity in the electric
field at a conductor/vacuum boundary. (There can, however, be a normal discontinuity. This allows $\psi$ to be non-zero at $y=0$ and $y=b$.)

Let us search for a separable solution of (\ref{e10.20})  of the form
\begin{equation}\label{e10.23}
\psi(x,z,t)= \psi_0\,\sin(k_x\,x)\,\cos(k\,z-\omega\,t),
\end{equation}
where $k$ represents the $z$-component of the wavevector (rather than its magnitude), and is the
effective wavenumber  for propagation along the waveguide.
The above solution automatically satisfies the boundary condition (\ref{e10.21}). The second boundary
condition (\ref{e10.22}) is satisfied provided
\begin{equation}
k_x = j\,\frac{\pi}{a},
\end{equation}
where $j$ is a positive integer. Suppose that $j$ takes its smallest possible value $1$.
(Of course, $j$ cannot be zero, since, in this case, $\psi=0$ everywhere.)
Substitution of expression (\ref{e10.23}) into the wave equation (\ref{e10.20}) yields the dispersion relation
\begin{equation}\label{e10.25}
\omega^2 = k^2\,c^2+\omega_0^{\,2},
\end{equation}
where
\begin{equation}
\omega_0 = \frac{\pi\,c}{a}.
\end{equation}
Note that this dispersion relation is analogous in form to the dispersion relation (\ref{e9.26})
for an electromagnetic wave propagating through a plasma, with the {\em cut-off frequency}, $\omega_0$,
playing the role of the plasma frequency, $\omega_p$. The cut-off frequency is so-called because
for $\omega<\omega_0$ the wavenumber is imaginary ({\em i.e.}, $k^2<0$), which implies that the
wave does not propagate along the waveguide, but, instead, decays exponentially with increasing $z$. 
On the other hand, for wave frequencies above the cut-off frequency the phase velocity, 
\begin{equation}
v_p = \frac{\omega}{k} = \frac{c}{\sqrt{1-\omega_0^{\,2}/\omega^2}},
\end{equation}
is superluminal. This is not a problem, however, since the group velocity, 
\begin{equation}
v_g= \frac{d\omega}{dk} = c\,\sqrt{1-\omega_0^{\,2}/\omega^2},
\end{equation}
which is the true signal velocity, remains subluminal. (Recall, from Section~\ref{s9.2}, that a high frequency electromagnetic
wave propagating through a plasma exhibits similar behavior.) 
Not surprisingly, the signal velocity
goes to zero as $\omega\rightarrow \omega_0$, since the signal ceases to
propagate at all when $\omega=\omega_0$. 

It turns out that waveguides support many distinct modes of propagation. The type of mode
discussed above is termed a TE (for transverse electric-field) mode, since the electric
field is transverse to the direction of propagation. There are many different sorts of TE mode, corresponding, for instance, to
different choices of the mode number, $j$. However, the $j=1$ mode has the lowest cut-off frequency.
There are also
TM (for transverse magnetic-field) modes, and TEM (for transverse electric- and magnetic-field) modes. 
TM modes also only propagate when the wave frequency exceeds a cut-off frequency. On the other hand, TEM modes (which are the same type of mode as that supported by
a conventional transmission line) propagate at all frequencies. Note, however, that TEM modes are only possible when the waveguide possesses an internal
conductor running along its length. 

\begin{figure}
\epsfysize=3in
\centerline{\epsffile{Chapter10/fig04.eps}}
\caption{\em A cylindrical wave.}\label{f10.4}   
\end{figure}

\section{Cylindrical Waves}\label{s10.4}
Consider a cylindrically symmetric wavefunction $\psi(\rho,t)$, where $\rho= \sqrt{x^2+y^2}$ is a conventional
cylindrical polar coordinate. Assuming that this function satisfies the three-dimensional wave equation (\ref{e10.9}), which
can be re-written (see Exercise 3)
\begin{equation}
\frac{\partial^2\psi}{\partial t^2} = v^2\left(\frac{\partial^2\psi}{\partial \rho^2} + \frac{1}{\rho}\,\frac{\partial\psi}{\partial\rho}\right),
\end{equation}
it is easily demonstrated that a sinusoidal cylindrical wave of phase angle $\phi$, wavenumber $k$, and angular frequency $\omega=k\,v$, takes the form (see Exercise 3)
\begin{equation}\label{e10.30}
\psi(\rho,t) \simeq \frac{\psi_0}{\rho^{1/2}}\,\cos(\phi+k\,\rho-\omega\,t)
\end{equation}
in the limit $k\,\rho\gg 1$. Here, $\psi_0/\rho^{1/2}$ is the amplitude of the wave. 
In this case, the associated wavefronts ({\em i.e.}, surfaces of constant phase) are a set of concentric cylinders which
propagate radially outward, from their common axis ($\rho=0$), at the phase velocity $v=\omega/k$. See Figure~\ref{f10.4}. Note that the wave amplitude attenuates as $\rho^{-1/2}$.
Such behavior  can be understood as a  consequence of {\em energy conservation}, which requires the power flowing across the various $\rho={\rm const.}$ surfaces to be
constant. (The areas of such surfaces scale as $A\propto \rho$.
Moreover, the power flowing across them  is proportional to $\psi^2\,A$, since the  energy flux associated with
a wave is generally proportional to $\psi^2$, and is directed normal to the wavefronts.)
 The cylindrical wave specified in expression (\ref{e10.30}) is such as would be generated by
a uniform {\em line source}\/ located at $\rho=0$. See Figure~\ref{f10.4}.

\section{Exercises}
{\small \begin{enumerate}
\item Show that the one-dimensional plane wave (\ref{e10.1}) is a solution of the one-dimensional wave equation (\ref{e10.8})
provided that 
$$
\omega=k\,v.
$$
Likewise, demonstrate that the three-dimensional plane wave (\ref{e10.5}) is a solution of the three-dimensional wave
equation (\ref{e10.9}) as long as
$$
\omega = |{\bf k}|\,v.
$$

\item Consider a square waveguide of internal dimensions $5\times 10\,{\rm cm}$. What is the frequency
(in MHz) of the lowest frequency TE mode which will propagate along the waveguide without attenuation?
What are the phase and group velocities (expressed as multiples of $c$) for a TE mode whose
frequency is $5/4$ times this cut-off frequency?

\item Demonstrate that for a cylindrically symmetric wavefunction $\psi(\rho,t)$, where $\rho= \sqrt{x^2+y^2}$, the 
three-dimensional wave equation (\ref{e10.9}) 
can be re-written 
$$
\frac{\partial^2\psi}{\partial t^2} = v^2\left(\frac{\partial^2\psi}{\partial \rho^2} + \frac{1}{\rho}\,\frac{\partial\psi}{\partial\rho}\right).
$$
Show that 
$$
\psi(\rho,t) \simeq \frac{\psi_0}{\rho^{1/2}}\,\cos(\phi+k\,\rho-\omega\,t)
$$
is an approximate solution of this equation in  the limit $k\,\rho\gg 1$, where $v=\omega/k$. 

\item Demonstrate that for a spherically symmetric wavefunction $\psi(r,t)$, where $r=\sqrt{x^2+y^2+z^2}$, the 
three-dimensional wave equation (\ref{e10.9}) 
can be re-written 
$$
\frac{\partial^2\psi}{\partial t^2} = v^2\left(\frac{\partial^2\psi}{\partial r^2} + \frac{2}{r}\,\frac{\partial\psi}{\partial r}\right).
$$
Show that
$$
\psi(r,t) =\frac{ \psi_0}{r}\,\cos(\phi+k\,r-\omega\,t)
$$
is a solution of this equation, where $v=\omega/k$. Explain why the wave amplitude attenuates as $r^{-1}$. What sort of wave source would be most likely to
generate the above type of wave solution?
\end{enumerate}}
