\chapter{Simple Harmonic Oscillation}
\section{Mass on a Spring}\label{s2.1}
Consider a compact mass $m$ which slides over a  frictionless horizontal surface. Suppose that
the mass is attached
to one end of a light horizontal spring whose other end is anchored in an immovable wall. See
Figure~\ref{f2.1}. At time $t$, let $x(t)$ be the extension of the spring: {\em i.e.}, the difference between
the spring's actual length and its unstretched length. Obviously, $x(t)$ can also be used as
a coordinate to determine the instantaneous horizontal displacement of the mass. 

\begin{figure}
\epsfysize=2.in
\centerline{\epsffile{Chapter02/fig01.eps}}
\caption{\em Mass on a spring}\label{f2.1}   
\end{figure}

The equilibrium state of the system corresponds to the situation in which
the mass is at rest, and the spring is unextended ({\em i.e.}, $x=\dot{x}=0$, where $\dot{}\equiv d/dt$).
In this state, zero horizontal force acts on the mass, and so there is no reason for it to start to move.
However, if the system is perturbed from its equilibrium state ({\em i.e.}, if the mass is displaced, so that the
spring becomes extended) then the mass experiences a horizontal {\em restoring force}\/ given by {\em Hooke's law}:
\begin{equation}
f (x)= -k\,x.
\end{equation}
Here, $k>0$ is the so-called {\em force constant}\/ of the spring. The negative sign indicates that $f(x)$ is indeed a restoring force ({\em i.e.}, if the displacement is
positive then the force is negative, and {\em vice versa}). 
Note that the magnitude of the restoring
force is {\em directly proportional}\/ to the displacement of the mass from its equilibrium 
position 
({\em i.e.}, $|f|\propto x$). Of course, Hooke's law only holds for relatively {\em small}\/ spring extensions.
Hence, the displacement of the mass cannot be made too large.
Incidentally, the motion of this particular dynamical system is representative of the
motion of a wide variety of mechanical systems when they are {\em slightly}\/ disturbed from a stable equilibrium state (see Section~\ref{spen}).

{\em Newton's second law of motion}\/ leads to the following time evolution equation for the system:
\begin{equation}
m\,\ddot{x} = - k\,x,\label{eshm}
\end{equation}
where $\ddot{}  \equiv d^2/dt^2$. 
This {\em differential equation}\/ is known as the {\em simple harmonic oscillator equation}, and its solution has been known
for centuries. In fact, the solution is
\begin{equation}
x(t) = a\,\cos(\omega\,t-\phi),\label{shm}
\end{equation}
where $a>0$, $\omega>0$, and $\phi$ are constants. We can demonstrate that Equation~(\ref{shm}) is indeed a
solution of Equation~(\ref{eshm}) by direct substitution. Plugging the right-hand side of (\ref{shm}) into 
Equation~(\ref{eshm}), and recalling from standard calculus that $d(\cos\theta)/d\theta = -\sin\theta$ and
$d (\sin\theta)/d\theta = \cos\theta$, so that $\dot{x} = -\omega\,a\,\sin(\omega\,t-\phi)$ and $\ddot{x} = -\omega^2\,a\,\cos(\omega\,t-\phi)$, where use has been made of
the {\em chain rule}, 
\begin{equation}
\frac{d}{dx}\left(f\left[g(x)\right]\right)\equiv \frac{df}{dg}\,\frac{dg}{dx},
\end{equation}
we obtain
\begin{equation}
-m\,\omega^2\,a\,\cos(\omega\,t-\phi) =- k\,a\,\cos(\omega\,t-\phi).
\end{equation}
It follows that Equation~(\ref{shm}) is the correct solution provided
\begin{equation}\label{eomega}
\omega = \sqrt{\frac{k}{m}}.
\end{equation}

\begin{figure}
\epsfysize=3in
\centerline{\epsffile{Chapter02/fig02.eps}}
\caption{\em Simple harmonic oscillation.}\label{f2.2}  
\end{figure}

Figure~\ref{f2.2} shows a graph of $x$ versus $t$ obtained from Equation~(\ref{shm}). The type of behavior  displayed here is
called {\em simple harmonic oscillation}.
It can be seen that
the displacement $x$ {\em oscillates}\/ between $x=-a$ and $x=+a$. Here, $a$ is termed the {\em amplitude}\/
of the oscillation. Moreover, the motion is {\em repetitive}\/ in time ({\em i.e.}, it repeats exactly after
a certain time period has elapsed). In fact, the {\em repetition period}\/ is
\begin{equation}
T = \frac{2\pi}{\omega}.
\end{equation}
This result is easily obtained from Equation~(\ref{shm}) by noting that $\cos\theta$ is a {\em periodic function}\/
of $\theta$ with
period $2\pi$: {\em i.e.}, $\cos(\theta+2\pi) \equiv \cos\theta$.  It follows that
the motion repeats every time $\omega\,t$ increases by $2\pi$: {\em i.e.}, every time $t$ increases by $2\pi/\omega$.
The {\em frequency}\/ of the motion ({\em i.e.}, the number of oscillations completed per
second) is
\begin{equation}
f = \frac{1}{T} = \frac{\omega}{2\pi}.
\end{equation}
It can be seen that $\omega$ is the motion's {\em angular frequency}; {\em i.e.}, the frequency
$f$ converted into radians per second. Of course, $f$ is measured in {\em Hertz}---otherwise
known as {\em cycles per second}.
Finally, the {\em phase angle}, $\phi$, determines the times at which the oscillation attains its maximum displacement,
$x=a$. In fact, since the maxima of $\cos\theta$ occur at $\theta=n\,2\pi$, where
$n$ is an arbitrary integer, the times of maximum displacement are
\begin{equation}
t_{\rm max} = T \left(n + \frac{\phi}{2\pi}\right).
\end{equation}
So, varying the phase angle simply shifts the pattern of oscillation backward and forward in time. See Figure~\ref{f2.3}. 

\begin{figure}
\epsfysize=3in
\centerline{\epsffile{Chapter02/fig03.eps}}
\caption{\em Simple harmonic oscillation. The
solid, short-dashed, and long dashed-curves correspond to $\phi = 0$, $+\pi/4$, and $-\pi/4$, respectively.}\label{f2.3}   
\end{figure}

\begin{table}[b]\centering
\begin{tabular}{c|cccc}\hline
$\omega\,t-\phi$ & $0$ & $\pi/2$ & $\pi$ & $3\pi/2$\\[0.5ex]\hline
$x$              & $+a$     & 0         & $-a$       & 0 \\[0.5ex]
$\dot{x}$        & 0        & $-\omega\,a$ & 0 & $+\omega\,a$ \\[0.5ex]
$\ddot{x}$       & $-\omega^2\,a$ &0& $+\omega^2\,a$ & 0
\end{tabular}
\caption{\em Simple harmonic oscillation.}\label{tshm}
\end{table}

Table~\ref{tshm} lists the displacement, velocity, and acceleration of the mass at various different phases of the
simple harmonic oscillation cycle. The information contained in this table can easily be derived from Equation~(\ref{shm}). Note that all of the non-zero values
shown in this table represent either the maximum or the minimum value taken by   the quantity in question during the
oscillation  cycle.

We have seen that when a mass on a spring is disturbed  it executes {\em simple harmonic
oscillation}\/ about its equilibrium position. In physical terms, if the mass's initial displacement is positive ($x>0$) then the
 restoring force is negative, and pulls the mass toward the equilibrium point ($x=0$). However,
 when the  mass reaches this point it is moving, and its inertia thus carries it onward,
 so that it acquires a negative displacement ($x<0$). The restoring force then becomes positive, and again pulls the mass toward the equilibrium point. However,  inertia again carries it past this point, and the mass   acquires a positive displacement.
The motion subsequently repeats itself {\em ad infinitum}. 
The angular frequency of the oscillation is determined by the spring stiffness, $k$,  and the system
inertia, $m$,  via Equation~(\ref{eomega}). 
On the other hand, the amplitude and phase angle of the oscillation are determined by the {\em initial conditions}. 
To be more exact, suppose that the instantaneous displacement and velocity of the mass at $t=0$ are $x_0$ and $v_0$,
respectively. It follows from Equation~(\ref{shm}) that
\begin{eqnarray}
x_0 &=& x(t=0) = a\,\cos\phi,\label{e2.9}\\[0.5ex]
v_0 &=& \dot{x}(t=0) =a\,\omega\,\sin\phi.
\end{eqnarray}
Here, use has been made of the trigonometric identities $\cos(-\theta) \equiv\cos\theta$ and $\sin(-\theta)
\equiv-\sin\theta$. Hence, we deduce that
\begin{equation}
a = \sqrt{x_0^{\,2} + (v_0/\omega)^2},
\end{equation}
and
\begin{equation}\label{e2.12}
\phi =\tan^{-1}\!\left(\frac{v_0}{\omega\,x_0}\right),
\end{equation}
since $\sin^2\theta+\cos^2\theta \equiv 1$ and $\tan\theta \equiv \sin\theta/\cos\theta$.

The kinetic energy of the system, which is the same as the kinetic energy of the mass,  is written
\begin{equation}
K = \frac{1}{2}\,m\,\dot{x}^2 = \frac{1}{2}\,m\,a^2\,\omega^2\,\sin^2(\omega\,t-\phi).
\end{equation}
 The potential energy of the system, which is the same as the potential energy of the
 spring, takes the form
\begin{equation}
U = \frac{1}{2}\,k\,x^2= \frac{1}{2}\,k\,a^2\,\cos^2(\omega\,t-\phi).
\end{equation}
Hence, the total energy is
\begin{equation}\label{eosce}
E = K + U =\frac{1}{2} \,k\,a^2= \frac{1}{2}\,m\,\omega^2\,a^2,
\end{equation}
since $m\,\omega^2 = k$ and $\sin^2\theta+\cos^2\theta \equiv 1$. Note that the
total energy is a {\em constant of the motion}. Moreover,
the energy is proportional to the {\em amplitude squared}\/ of the oscillation.
It follows, from the above expressions, that the simple harmonic oscillation of a
mass on a spring is characterized
by a continual back and forth flow of energy between kinetic and potential components.
The kinetic energy attains its maximum value, and the potential energy its minimum value,  when the displacement is zero ({\em i.e.}, when $x=0$). Likewise,
the potential energy attains its maximum value, and the kinetic energy 
its minimum value, when the displacement is maximal ({\em i.e.}, when $x=\pm a$). 
Note that the minimum value of $K$ is {\em zero}, since the system is instantaneously at rest
when the displacement is maximal.

\section{Simple Harmonic Oscillator Equation}
Suppose that a physical system possesing {\em one degree of freedom}---{\em i.e.}, a
system whose instantaneous state at time $t$ is fully described by a {\em single dependent variable}, $s(t)$---obeys the following time evolution equation [{\em cf.}, Equation~(\ref{eshm})]:
\begin{equation}\label{e2.16}
\ddot{s}+ \omega^2\,s=0,
\end{equation}
where $\omega>0$ is a  constant.  As we have seen, this differential
equation is called the {\em simple harmonic oscillator equation}, and has the
following  solution
\begin{equation}\label{e2.17}
s(t) = a\,\cos(\omega\,t-\phi),
\end{equation}
where $a>0$ and $\phi$ are constants. Moreover, the above equation describes a type of oscillation
characterized by a constant {\em amplitude}, $a$, and a constant {\em angular frequency}, $\omega$.
The {\em phase angle}, $\phi$, determines the times at which the oscillation attains its
maximum value. Finally, the frequency of the oscillation (in Hertz) is $f=\omega/2\pi$,
and the period is $T=2\pi/\omega$. Note that the frequency and period of the
oscillation are both determined by the constant $\omega$, which appears in the simple harmonic oscillator equation, whereas the amplitude, $a$, and phase angle, $\phi$, are both determined by the {\em initial conditions}---see Equations~(\ref{e2.9})--(\ref{e2.12}). In fact, $a$ and $\phi$ are the two {\em constants of integration}\/ of the
{\em second-order ordinary differential equation}\/ (\ref{e2.16}). Recall, from standard differential equation theory, that the
most general solution of an $n$th-order ordinary differential equation ({\em i.e.},
an equation involving a single independent variable, and a single dependent variable, in which the highest derivative of the dependent  with respect to the
independent variable is
$n$th-order, and the lowest zeroth-order) involves $n$ arbitrary constants of integration. (Essentially, this is because
we have to integrate the equation $n$ times with respect to the independent variable in order to reduce it to zeroth-order, and so
obtain the solution. Furthermore, each integration introduces an arbitrary constant: {\em e.g.},
the integral of $\dot{s}=a$, where $a$ is a known constant, is $s=a\,t+b$, where $b$ is an arbitrary constant.)

Multiplying Equation~(\ref{e2.16}) by $\dot{s}$, we obtain
\begin{equation}
\dot{s}\,\ddot{s} + \omega^2\,\dot{s}\,s=0.
\end{equation}
However, this can also be written in the form
\begin{equation}
\frac{d}{dt}\!\left(\frac{1}{2}\,\dot{s}^{\,2}\right) +\frac{d}{dt}\!\left(
\frac{1}{2}\,\omega^2\,s^2\right)=0,
\end{equation}
or
\begin{equation}
\frac{d{\cal E}}{dt} = 0,
\end{equation}
where
\begin{equation}\label{e2.21}
{\cal E} = \frac{1}{2}\,\dot{s}^{\,2} + \frac{1}{2}\,\omega^2\,s^2.
\end{equation}
Clearly, ${\cal E}$ is a {\em conserved quantity}: {\em i.e.}, it does not vary with time. In fact, this quantity is
generally proportional to the overall energy of the system. For instance, ${\cal E}$ would be the energy divided by the mass in
the mass-spring system discussed in Section~\ref{s2.1}. Note that ${\cal E}$ is either
zero or positive, since neither of the terms on the right-hand side of Equation~(\ref{e2.21}) can be negative. Let us search for an {\em equilibrium state}. Such a state is
characterized by $s= {\rm constant}$, so that $\dot{s}=\ddot{s}=0$. It follows
from (\ref{e2.16}) that $s=0$, and from (\ref{e2.21}) that ${\cal E}=0$. We conclude that   the system  can only remain permanently at
rest when ${\cal E}=0$.
 Conversely,  the system can
never permanently come to rest when ${\cal E}>0$,  and must, therefore, keep moving for ever. Furthermore, since the equilibrium state is characterized by $s=0$, it follows that
$s$ represents a kind of  ``displacement'' of the system from this state.
It is also apparent, from (\ref{e2.21}), that $s$ attains it maximum value when $\dot{s}=0$.
In fact,
\begin{equation}
s_{\rm max} = \frac{\sqrt{2\,{\cal E}}}{\omega}.
\end{equation}
This, of course, is the amplitude of the oscillation: {\em i.e.}, $s_{\rm max}=a$. 
Likewise, $\dot{s}$ attains its maximum value when $x=0$, and
\begin{equation}
\dot{s}_{\rm max} = \sqrt{2\,{\cal E}}.
\end{equation}

Note that the simple harmonic oscillation (\ref{e2.17}) can also
be written in the form
\begin{equation}\label{e2.24}
s(t) = A\,\cos(\omega\,t) + B\,\sin(\omega\,t),
\end{equation}
where $A= a\,\cos\phi$ and $B=a\,\sin\phi$. Here, we have employed the trigonometric identity $\cos(x-y) \equiv \cos x\,\cos y+\sin x\,\sin y$. 
Alternatively, (\ref{e2.17}) can be written
\begin{equation}\label{e2.25}
s(t) = a\,\sin(\omega\,t-\phi'),
\end{equation}
where $\phi'=\phi-\pi/2$, and use has been made of the trigonometric
identity $\cos\theta \equiv \sin(\pi/2+\theta)$. Clearly, there are many different
ways of representing a simple harmonic oscillation, but they all involve
{\em linear}\/ combinations of sine and cosine functions whose arguments
take the form $\omega\,t+c$, where $c$ is some constant. Note, however, that,
whatever form it takes, 
a {\em general}\/ solution to the simple harmonic oscillator equation must always contain {\em two}\/ arbitrary constants: {\em i.e.}, $A$ and $B$ in (\ref{e2.24}) or
$a$ and $\phi'$ in (\ref{e2.25}).

The simple harmonic oscillator equation, (\ref{e2.16}), is a {\em linear}\/ differential equation,
which  means that
if $s(t)$ is a solution  then so is $a\,s(t)$, where $a$ is
an arbitrary constant. This can be verified by multiplying the equation by $a$,
and then making use of the fact that $a\,d^2s/dt^2=d^2(a\,s)/dt^2$. Now, linear
differential equations have a very important and useful property: {\em i.e.}, their
solutions are {\em superposable}. This means that if $s_1(t)$ is a
solution to Equation~(\ref{e2.16}), so
that
\begin{equation}
\ddot{s}_1=-\omega^2\,s_1,
\end{equation}
and
$s_2(t)$ is a different solution, so that
\begin{equation}
\ddot{s}_2=-\omega^2\,s_2,
\end{equation}
then $s_1(t)+s_2(t)$ is also a solution. This can be verified by adding the previous
two equations, and making use of the fact that $d^2s_1/dt^2+d^2 s_2/dt^2=d^2(s_1+s_2)/dt^2$. Furthermore, it is easily demonstrated that {\em any linear combination}\/ of $s_1$ and $s_2$,
such as $a\,s_1+b\,s_2$, where $a$ and $b$ are constants, is also a solution.
It is very helpful to know this fact. 
For instance, the special solution to the simple harmonic oscillator equation (\ref{e2.16}) with the simple initial
conditions $s(0) = 1$ and $\dot{s}(0) = 0$ is easily shown to be
\begin{equation}
s_1(t) = \cos(\omega\,t).
\end{equation}
Likewise, the special solution with the simple initial conditions $s(0)=0$ and $\dot{s}(0)=1$ is clearly
\begin{equation}
s_2(t) = \omega^{-1}\,\sin(\omega\,t).
\end{equation}
Thus, since the solutions to the simple harmonic oscillator equation are superposable, the
solution with the general initial conditions $s(0)=s_0$ and $\dot{s}(0)=\dot{s}_0$ is
\begin{equation}
 s(t)=s_0\,s_1(t) + \dot{s}_0\,s_2(t), 
 \end{equation}
 or
\begin{equation}
s(t) =  s_0\,\cos(\omega\,t)+ \frac{\dot{s}_0}{\omega}\,\sin(\omega\,t).
\end{equation}

\section{$LC$ Circuit}\label{slc}
Consider an electrical circuit consisting of an inductor, of inductance $L$, connected
in series with a capacitor, of capacitance $C$. See Figure~\ref{f2.4}. Such
a circuit is known as an $LC$ circuit, for obvious reasons. Suppose that
$I(t)$ is the instantaneous current flowing around the circuit. 
 According to
standard electrical circuit theory, the 
potential difference across the inductor is $L\,\dot{I}$. 
Again,  from standard electrical circuit theory,  the potential difference across the capacitor is $V=Q/C$, where
$Q$ is the charge stored on the capacitor's positive plate. However,
since electric charge is {\em conserved}, 
the current flowing around the circuit is equal to the rate at which charge accumulates on the capacitor's 
 positive plate: {\em i.e.}, $I = \dot{Q}$. 
 Now, according to
{\em Kirchhoff's second circuital law}, the sum of the potential differences across the
various components of a closed circuit loop is equal to zero. In other words,
\begin{equation}\label{e2.31}
 L\,\dot{I}+Q/C = 0.
\end{equation}
Dividing by $L$, and differentiating with respect to $t$, we obtain
\begin{equation}\label{e2.32}
\ddot{I} +  \omega^2\,I = 0,
\end{equation}
where
\begin{equation}\label{e2.33}
\omega = \frac{1}{\sqrt{L\,C}}.
\end{equation}
Comparison with Equation~(\ref{e2.16}) reveals that (\ref{e2.32}) is a {\em simple harmonic oscillator equation}\/ with the associated angular oscillation frequency $\omega$. 
We conclude that the current in an $LC$ circuit executes simple harmonic oscillations of the form
\begin{equation}\label{e2.34}
I(t) = I_0\,\cos(\omega\,t-\phi),
\end{equation}
where $I_0>0$ and $\phi$ are constants. 
Now, according to Equation~(\ref{e2.31}), the potential difference, $V=Q/C$,  across the capacitor is minus that across the inductor, so that $V= -L\,\dot{I}$, giving
\begin{equation}\label{e2.35}
V(t) = \sqrt{\frac{L}{C}}\,I_0\,\sin (\omega\,t-\phi) = \sqrt{\frac{L}{C}}\,I_0\,\cos(\omega\,t-\phi-\pi/2).
\end{equation}
Here, use has been made of the trigonometric identity $\sin\theta\equiv \cos(\theta-\pi/2)$. 
It follows that the voltage in an $LC$ circuit  oscillates at the {\em same
frequency}\/ as the current, but with a {\em phase shift}\/ of $\pi/2$. In other words, the
voltage is maximal when the current is zero, and {\em vice versa}. 
The amplitude of the voltage oscillation is that of the current oscillation
multiplied by $\sqrt{L/C}$. Thus, we can also write
\begin{equation}\label{e2.36}
V(t) = \sqrt{\frac{L}{C}}\,I(t-\omega^{-1}\,\pi/2).
\end{equation}

\begin{figure}
\epsfysize=2.5in
\centerline{\epsffile{Chapter02/fig04.eps}}
\caption{\em An $LC$ circuit.}\label{f2.4}   
\end{figure}

Comparing with Equation~(\ref{e2.21}),  it is
clear that
\begin{equation}
{\cal E} = \frac{1}{2}\,\dot{I}^{\,2} +\frac{1}{2}\, \omega^2\,I^2
\end{equation}
is a conserved quantity. However, $\omega^2=1/L\,C$, and $\dot{I} = -V/L$.  Thus, multiplying the
above expression by $C\,L^2$, we obtain
\begin{equation}\label{e2.37}
E = \frac{1}{2}\,C\,V^2 + \frac{1}{2}\,L\,I^2.
\end{equation}
The first and second terms on the right-hand side of the above expression can be recognized as the instantaneous energies
stored in the capacitor and the inductor, respectively. The former energy is stored in the
electric field generated when the capacitor is charged, whereas the latter is stored in the
magnetic field induced when current flows through the inductor. It follows that
(\ref{e2.37}) is the total energy of the circuit, and that this
energy is a {\em conserved quantity}.  Clearly, the oscillations
of an $LC$ circuit can be understood as a cyclic interchange between
electric energy stored in the capacitor and magnetic energy stored in the inductor, much as the oscillations of the mass-spring
system studied in Section~\ref{s2.1} can be understood as a cyclic interchange
between kinetic energy stored by the mass and  potential energy stored by the spring. 

Suppose that at $t=0$  the capacitor is charged
to a voltage $V_0$, and there is no current flowing through the inductor. In other
words, the initial state is one in which all of the circuit energy resides in the
capacitor. The initial conditions are $V(0)=-L\,\dot{I}(0)=V_0$ and $I(0) = 0$.
It is easily demonstrated that the current evolves in time as
\begin{equation}
I(t) = -\frac{V_0}{\sqrt{L/C}}\,\sin(\omega\,t).
\end{equation}
Suppose that at $t=0$ the capacitor is fully discharged, and there is a current
$I_0$ flowing through the inductor. In other words, the initial state is one in
which all of the circuit energy resides in the inductor. The initial
conditions are $V(0)=-L\,\dot{I}(0)=0$ and $I(0)=I_0$. It is easily demonstrated
that the current evolves in time as
\begin{equation}
I(t) = I_0\,\cos(\omega\,t).
\end{equation}
Suppose, finally, that at $t=0$ the capacitor is charged to a voltage $V_0$,
and the current flowing through the inductor is $I_0$. Since the solutions
of the simple harmonic oscillator equation are superposable, it follows  that the
current evolves in time as
\begin{equation}\label{e2.41}
I(t) =  -\frac{V_0}{\sqrt{L/C}}\,\sin(\omega\,t) +  I_0\,\cos(\omega\,t).
\end{equation}
Furthermore, according to Equation~(\ref{e2.36}),  the voltage evolves in time
as
\begin{equation}
V(t)= - V_0\,\sin(\omega\,t-\pi/2) + \sqrt{\frac{L}{C}}\,I_0\,\cos(\omega\,t-\pi/2),
\end{equation}
or
\begin{equation}\label{e2.43}
V(t) = V_0\,\cos(\omega\,t) + \sqrt{\frac{L}{C}}\,I_0\,\sin(\omega\,t).
\end{equation}
Here, use has been made of the trigonometric identities $\sin (\theta-\pi/2)\equiv -\cos\theta$ and  $\cos(\theta-\pi/2)\equiv \sin\theta$. 

The instantaneous electrical power absorption by the capacitor, which can easily be
shown to be minus the instantaneous power absorption by the inductor,  is
\begin{equation}
P(t)=I(t)\,V(t) = I_0\,V_0\,\cos(2\,\omega\,t) + \frac{1}{2}\left(I_0^{\,2}\,\sqrt{\frac{L}{C}}- \frac{V_0^{\,2}}{\sqrt{L/C}}\right)\sin(2\,\omega\,t),
\end{equation}
where use has been made of Equations~(\ref{e2.41}) and (\ref{e2.43}), as well as the trigonometric identities $\cos(2\,\theta)\equiv \cos^2\theta-\sin^2\theta$ and $\sin(2\,\theta)\equiv 2\,\sin\theta\,\cos\theta$. Hence, the
average power absorption during a cycle of the oscillation,
\begin{equation}
\langle P\rangle \equiv \frac{1}{T}\int_0^T P(t)\,dt,
\end{equation}
is  {\em zero}, since it is easily demonstrated that $\langle\cos(2\,\omega\,t)\rangle=\langle \sin(2\,\omega\,t)\rangle=0$.  In other words, any energy which the capacitor absorbs from the
circuit during one part of the oscillation cycle is returned to the circuit without loss
during another. The same goes for the inductor.

\begin{figure}
\epsfysize=2.75in
\centerline{\epsffile{Chapter02/fig05.eps}}
\caption{\em A simple pendulum.}\label{f2.5}  
\end{figure}

\section{Simple Pendulum}\label{spen}
Consider a compact mass $m$ suspended from a light inextensible string of length $l$, such that the
mass is free to swing from side to side in a vertical plane, as shown in 
Figure~\ref{f2.5}.
This setup is known as a {\em simple pendulum}. 
 Let $\theta$ be the angle subtended between the string and
the downward vertical. Obviously, the stable equilibrium state of the system corresponds to
the situation in which the mass is stationary, and hangs vertically down ({\em 
i.e.}, $\theta=\dot{\theta}=0$).
The angular equation of motion of the pendulum is 
\begin{equation}
I\,\ddot{\theta} = \tau,
\end{equation}
where $I$ is the moment of inertia of the mass, and $\tau$ the torque acting 
about the suspension point.
For the
case in hand, given that the mass is essentially a point particle, and is situated a distance $l$ from
the axis of rotation ({\em i.e.}, the suspension point), it is easily seen that 
$I=m\,l^2$. 

The two forces acting on the mass are the downward gravitational force, $m\,g$, where $g$ is the acceleration due to gravity, 
 and the tension, $T$, in the string.
Note, however, that the tension makes no contribution to the torque, 
since its line of action clearly passes
through the suspension point. From elementary trigonometry, 
the line of action of the gravitational force passes a perpendicular distance $l\,\sin\theta$ 
from the
suspension point. Hence, the magnitude of the gravitational torque is $m\,g\,l\,
\sin\theta$.
Moreover, the gravitational torque is  a {\em restoring torque}: {\em i.e.}, if 
the mass is
displaced slightly from its equilibrium position ({\em i.e.}, $\theta =0$) then the
 gravitational torque clearly acts
 to push the mass back towards that position. Thus, we can write
\begin{equation}
\tau = - m\,g\,l\,\sin\theta.
\end{equation}
Combining the previous two equations, we obtain the following  angular equation 
of motion of the pendulum:
\begin{equation}\label{epend}
l\,\ddot{\theta} + g\,\sin\theta=0.
\end{equation}
Note that, unlike all of the other time evolution equations which we have
examined so far in this chapter, the above equation is {\em nonlinear} [since $\sin(a\,\theta)\neq a\,\sin\theta$], which means that it is generally
very difficult to solve.

Suppose, however,  that the system does not stray very far from
its equilibrium position ($\theta=0$). If this is the case then we
can expand $\sin\theta$ in a Taylor series about $\theta=0$. We obtain
\begin{equation}
\sin\theta = \theta - \frac{\theta^3}{6} + \frac{\theta^5}{120} + {\cal O}(\theta^7).
\end{equation}
Now, if $|\theta|$ is sufficiently small then the series is dominated by its
first term, and we can write $\sin\theta\simeq \theta$. This is known
as the {\em small angle approximation}.
Making use of this approximation, 
the  equation of motion (\ref{epend}) simplifies to
\begin{equation}\label{e2.47}
\ddot{\theta} +\omega^2\,\theta\simeq 0,
\end{equation}
where 
\begin{equation}
\omega = \sqrt{\frac{g}{l}}.
\end{equation}
 Of course, (\ref{e2.47}) is just the
{\em simple harmonic oscillator equation}. Hence, we can immediately write its solution
in the form
\begin{equation}\label{e4.64}
\theta(t) = \theta_0\,\cos(\omega\,t-\phi),
\end{equation}
where $\theta_0>0$ and $\phi$ are constants. 
We conclude that the pendulum swings back and forth at a fixed frequency, $\omega$, which depends on $l$ and $g$, but is {\em independent}\/ of the amplitude,
$\theta_0$, of the motion. Actually, this result only holds as long as
the small angle approximation remains valid. It turns out that $\sin\theta\simeq \theta$
is a good approximation provided  $|\theta|\ltapp 6^\circ$. Hence, the period
of a simple pendulum is only amplitude independent when the amplitude of the motion
is less than about $6^\circ$. 

\section{Exercises}
{\small
\begin{enumerate}

\item A mass stands on a platform which executes simple harmonic oscillation in a vertical
direction at a frequency of $5\,{\rm Hz}$. Show that the mass loses contact with the platform
when the displacement exceeds $10^{-2}\,{\rm m}$. 

\item A small body rests on a horizontal diaphragm of a loudspeaker
which is supplies with an alternating current of constant amplitude but variable
frequency. If diaphragm executes simple harmonic oscillation in the vertical
direction of amplitude $10\,\mu{\rm m}$, at all frequencies, find the greatest
frequency for which the small body stays in contact with the diaphragm. 

\item Two light springs have spring constants $k_1$ and $k_2$, respectively, and are used in a vertical
orientation to support an object of mass $m$. Show that the angular frequency of small amplitude oscillations about the equilibrium state
is $[(k_1+k_2)/m]^{1/2}$ if the springs are in parallel, and $[k_1\,k_2/(k_1+k_2)\,m]^{1/2}$
if the springs are in series.

\item A body of uniform cross-sectional area $A$ and mass density $\rho$ floats in a liquid
of density $\rho_0$ (where $\rho<\rho_0$), and at equilibrium displaces a volume $V$. Making use of Archimedes principle (that the buoyancy force acting on a partially
submerged body is equal to the mass of the displaced liquid), show
that the period of small amplitude oscillations about the equilibrium position is
$$
T = 2\pi\,\sqrt{\frac{V}{g\,A}}.
$$

\item A particle of mass $m$ slides in a frictionless semi-circular depression in the
ground
of radius $R$. Find the angular frequency of small amplitude oscillations
about the particle's equilibrium position, assuming that the oscillations
are essentially one dimensional, so that the particle  passes through
the lowest point of the depression during each oscillation cycle.

\item If a thin wire is twisted through an angle $\theta$ then a restoring
torque $\tau = - k\,\theta$ develops, where $k>0$ is known as the {\em torsional
force constant}. Consider a so-called {\em torsional pendulum},
which consists of a horizontal disk of mass $M$, and moment of inertia $I$, suspended at its
center from a thin vertical wire of negligible mass and length $l$, whose other end is attached to a fixed
support. The disk is free to rotate about a vertical axis passing through the suspension point, but such rotation twists the wire. Find the frequency of torsional oscillations of the disk about its
equilibrium position.

\item Suppose that a hole is drilled through a laminar ({\em i.e.}, flat) object
of mass $M$, which is then suspended in a frictionless manner from a horizontal
axis passing through the hole, such that it is free to rotate in a vertical plane.
 Suppose that the moment of inertia of the object about the
axis is $I$, and that the distance of the hole from the object's center of mass is $d$. 
Find the frequency of small angle oscillations of the object about its
equilibrium position. Hence, find the frequency of small angle oscillations of a {\em compound pendulum}\/
consisting of a uniform
rod of mass $M$ and length $l$ suspended vertically from a horizontal axis
passing through one of its ends. 

\item A pendulum consists of a uniform circular disk of radius $r$ which is
free to turn about a horizontal axis perpendicular to its plane. Find the position
of the axis for which the periodic time is a minimum. 

\item A particle of mass $m$ executes one-dimensional simple harmonic oscillation  under the action of a
conservative force such that its instantaneous $x$ coordinate is
$$
x(t) = a\,\cos(\omega\,t-\phi).
$$
Find the average values of $x$, $x^2$, $\dot{x}$, and $\dot{x}^2$ over a single cycle of the
oscillation. Find the average values of the kinetic and potential energies of the
particle over a single cycle of the oscillation.

\item A particle executes two-dimensional simple harmonic oscillation such that its instantaneous coordinates in the $x$-$y$
plane are
\begin{eqnarray}
x(t) &=&a\,\cos(\omega\,t),\nonumber\\[0.5ex]
y(t) &=&a\,\cos(\omega\,t-\phi).\nonumber
\end{eqnarray}
Describe the motion when (a) $\phi=0$, (b) $\phi=\pi/2$, and (c) $\phi=-\pi/2$.
In each case, plot the trajectory of the particle in the $x$-$y$ plane.

\item An $LC$ circuit is such that at $t=0$ the capacitor is uncharged and a
current $I_0$ flows through the inductor. Find an expression for the
charge $Q$ stored on the positive plate of the capacitor as a function of time.

\item A simple pendulum of mass $m$ and length $l$ is such that $\theta(0)=0$
and $\dot{\theta}(0) = \omega_0$. Find the subsequent motion, $\theta(t)$, assuming that 
its amplitude  remains small. Suppose, instead, that  $\theta(0)=\theta_0$ and
$\dot{\theta}(0) = 0$. Find the subsequent motion. Suppose, finally, that
$\theta(0)=\theta_0$ and $\dot{\theta}(0) = \omega_0$. Find the subsequent motion.

\item Demonstrate that 
$$
E = \frac{1}{2}\,m\,l^2\,\dot{\theta}^{\,2} + m\,g\,l\,(\cos\theta-1)
$$
is a constant of the motion of a simple pendulum whose time evolution equation
is given by (\ref{epend}). (Do not make the small angle approximation.) Hence, show
that the amplitude of the motion, $\theta_0$, can be written
$$
\theta_0 = 2\,\sin^{-1}\left(\frac{E}{2\,m\,g\,l}\right)^{1/2}.
$$
Finally, demonstrate that the period of the motion is determined by
$$
\frac{T}{T_0} = \frac{1}{\pi}\int_0^{\theta_0}\frac{d\theta}{\sqrt{\sin^2(\theta_0/2)-\sin^2(\theta/2)}},
$$
where $T_0$ is the period of small angle oscillations. Verify that $T/T_0\rightarrow 1$
as $\theta_0\rightarrow 0$. Does the period increase, or decrease, as the amplitude
of the motion increases?
\end{enumerate}
}
